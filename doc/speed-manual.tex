\documentclass{easychair}

% https://en.wikibooks.org/wiki/LaTeX/Source_Code_Listings
\usepackage{listings}

% For inline citations
\usepackage{bibentry}
\nobibliography*

% Folders with images
\makeatletter
\providecommand*{\input@path}{}
\g@addto@macro\input@path{{../src/}{src/}}% append
\makeatother

% My own commands, commands adapted from Joey and Peter

% Cross-reference commands.
% Per Dr. Grogono and my own self.
\newcommand{\xf}[1]{Figure~\ref{#1}}
\newcommand{\xp}[1]{page~\pageref{#1}}
\newcommand{\xs}[1]{Section~\ref{#1}}
\newcommand{\xa}[1]{Appendix~\ref{#1}}
\newcommand{\xc}[1]{Chapter~\ref{#1}}
\newcommand{\xt}[1]{Table~\ref{#1}}
\newcommand{\xl}[1]{Listing~\ref{#1}}

%
% Abbrs
%

\newcommand{\rpc}{{RPC\index{RPC}}}
\newcommand{\rmi}{{RMI\index{RMI}}}
\newcommand{\clp}{{CLP\index{CLP}}}
\newcommand{\tlp}{{TLP\index{TLP}}}
\newcommand{\slp}{{SLP\index{SLP}}}
\newcommand{\complus}{{DCOM+\index{DCOM+}}}
\newcommand{\corba}{{CORBA\index{CORBA}}}
\newcommand{\jini}{{Jini\index{Jini}}}
\newcommand{\jms}{{JMS\index{JMS}}}
\newcommand{\dotnet}{{.NET Remoting\index{.NET Remoting}}}
\newcommand{\gnu}{{GNU\index{GNU}}}
\newcommand{\tcpip}{{TCP/IP\index{TCP/IP}}}
\newcommand{\AST}{{AST\index{AST}}}


%
% The GIPSY
%

\newcommand{\gipc}{{GIPC\index{GIPC}\index{Frameworks!GIPC}}}
\newcommand{\gicf}{{GICF\index{GICF}\index{Frameworks!GICF}}}
\newcommand{\iplcf}{{IPLCF\index{IPLCF}\index{Frameworks!IPLCF}}}
\newcommand{\gee}{{GEE\index{GEE}\index{Frameworks!GEE}}}
\newcommand{\geer}{{GEER\index{GEER}}}
\newcommand{\gipsy}{{GIPSY\index{GIPSY}}}
\newcommand{\agipsy}{{AGIPSY\index{AGIPSY}}}
\newcommand{\ripe}{{RIPE\index{RIPE}\index{Frameworks!RIPE}}}
\newcommand{\dpr}{{DPR\index{DPR}}}
\newcommand{\dms}{{DMS\index{DMS}}}
\newcommand{\dmf}{{DMF\index{DMF}\index{Frameworks!DMF}}}
\newcommand{\dfg}{{DFG\index{DFG}}}


%
% The Lucids Family
%

\newcommand{\glu}{{GLU\index{GLU}}}
\newcommand{\glusharp}{{GLU\#\index{GLU\#}}}
\newcommand{\gipl}{{GIPL\index{GIPL}}}
\newcommand{\sipl}{{SIPL\index{SIPL}}}
\newcommand{\ipl}{{IPL\index{IPL}}}
\newcommand{\lucid}{{Lucid\index{Lucid}}}
\newcommand{\ilucid}{{Indexical Lucid\index{Indexical Lucid}}}
\newcommand{\jlucid}{{JLucid\index{JLucid}}}
\newcommand{\olucid}{{Objective Lucid\index{Tensor Lucid}}}
\newcommand{\tlucid}{{Tensor Lucid\index{Tensor Lucid}}}
\newcommand{\plucid}{{Partial Lucid\index{Partial Lucid}}}
\newcommand{\flucid}{{Forensic Lucid\index{Forensic Lucid}}}
\newcommand{\onyx}{{Onyx\index{Onyx}}}
\newcommand{\lucx}{{Lucx\index{Lucx}}}
\newcommand{\ooip}{{OOIP\index{OOIP}}}
\newcommand{\ioop}{{IOOP\index{IOOP}}}
\newcommand{\jooip}{{JOOIP\index{JOOIP}}}

%
% The Other Intensionals
%

\newcommand{\marfl}{{MARFL\index{MARFL}}}


%
% The Imperatives
%

\newcommand{\C}{{C\index{C}}}
\newcommand{\cpp}{{C++\index{C++}}}
\newcommand{\perl}{{Perl\index{Perl}}}
\newcommand{\java}{{Java\index{Java}}}
\newcommand{\python}{{Python\index{Python}}}
\newcommand{\fortran}{{Fortran\index{Fortran}}}
\newcommand{\aspectj}{{AspectJ\index{AspectJ}}}
\newcommand{\php}{{PHP\index{PHP}}}


%
% The Functionals
%

\newcommand{\lisp}{{LISP\index{LISP}}}
\newcommand{\scheme}{{Scheme\index{Scheme}}}
\newcommand{\haskell}{{Haskell\index{Haskell}}}
\newcommand{\mllessequal}{{ML$_{\le}$\index{ML$_{\le}$}}}
\newcommand{\fcpp}{{FC++\index{FC++}}}


%
% Lucid Operators: The Original and The New
%

\newcommand{\olucidop}[1]{{\bf \texttt{\textmd{\textsc{#1}}}}}
\newcommand{\lucidop}[1]{{\bf \texttt{#1}}}


%
% Forensic terms
%

% Forward transition
\newcommand{\trans}{$\psi$}
\newcommand{\transeq}[2]{$\psi(#1) = #2$}
% Inverse transition function
\newcommand{\invtrans}{$\Psi^{-1}$}
\newcommand{\invtranseq}[2]{$\Psi^{-1}(#1) = #2$}


%
% Util
%

\newcommand{\tab}[1]{\hspace{#1pt}}

\newcommand{\shrule}[0]{\vspace{3pt}\hrule\vspace{6pt}}
\newcommand{\ehrule}[0]{\vspace{6pt}\hrule\vspace{3pt}}

\newcommand{\nonterminal}[1]{$\mathtt{<\!\!#1\!\!>}$}

\newcommand{\source}[1]
{
	{\shrule}
	\scriptsize
	#1
	\normalsize
	\hrule
}

\newcommand{\sourcefloat}[3]
{
	\begin{figure}[!hp]
	\begin{centering}
	\begin{minipage}{0.5\textwidth}
	\source{#1}
	\end{minipage}
	\caption{\small{#3}}
	\label{#2}
	\end{centering}
	\end{figure}
}

\newcommand{\todo}[0]
{
	\begin{center}{\Large [TODO]}\index{TODO}\end{center}
}

\newcommand{\file}[1]{\url{#1}\index{Files!#1}}
\newcommand{\tool}[1]{\texttt{#1}\index{Tools!#1}}
\newcommand{\option}[1]{\texttt{#1}\index{Options!#1}}
\newcommand{\api}[1]{\texttt{#1}\index{API!#1}}
\newcommand{\apipackage}[1]{\url{#1}\index{API!Packages!#1}\index{Packages!#1}}
\newcommand{\datatype}[1]{\texttt{#1}\index{Type!#1}}
\newcommand{\codesegment}[1]{\texttt{\##1}\index{Segments!\##1}}


%
% Tools
%

\newcommand{\javacc}[0]{JavaCC\index{Tools!JavaCC}}
\newcommand{\junit}[0]{JUnit\index{Tools!JUnit}}


%
% Frameworks, APIs, Libraries
%

\newcommand{\marf}[0]{MARF\index{MARF}\index{Frameworks!MARF}\index{Libraries!MARF}}
\newcommand{\dmarf}[0]{DMARF\index{MARF!Distributed}\index{Frameworks!Distributed MARF}\index{Libraries!Distributed MARF}}
\newcommand{\admarf}
	[0]
	{ADMARF%
	\index{ADMARF}%
	\index{MARF!Autonomic}%
	\index{DMARF!Autonomic}%
	\index{Frameworks!Autonomic Distributed MARF}%
	\index{Libraries!Autonomic Distributed MARF}%
	}
\newcommand{\jdsf}[0]{JDSF\index{Frameworks!JDSF}\index{Libraries!JDSF}}
\newcommand{\sqlrand}[0]{SQLrand\index{SQLrand}}
\newcommand{\hsqldb}[0]{HSQLDB\index{HSQLDB}\index{Tools!HSQLDB}\index{Databases!HSQLDB}}
\newcommand{\cryptolysis}[0]{Cryptolysis\index{Frameworks!Cryptolysis}}
\newcommand{\assl}{ASSL\index{ASSL}\index{Autonomic Systems Specification Language}}


%
% Def
%

\newcommand{\statement}[2]
{
	\vspace{7pt}
	\shrule
	{\bf #1}

	#2
	\ehrule
	\vspace{7pt}
}

% \newcommand{\proposition}[2]
\newcommand{\sproposition}[2]
{
	\statement{Proposition #1}{#2}
}

% \newcommand{\definition}[2]
\newcommand{\sdefinition}[2]
{
	\statement{Definition #1}{#2}
}

% \newcommand{\axiom}[2]
\newcommand{\saxiom}[2]
{
	\statement{Axiom #1}{#2}
}

% \newcommand{\theorem}[2]
\newcommand{\stheorem}[2]
{
	\statement{Theorem #1}{#2}
}

%
% OS
%

\newcommand{\unix}{\index{Unix@{\sc{Unix}}}{\sc{Unix}}}
\newcommand{\macos}[1]{\index{Mac OS #1@{\sc{Mac OS #1}}}{\sc{Mac OS #1}}}
\newcommand{\linux}{\index{Linux@{\sc{Linux}}}{\sc{Linux}}}
\newcommand{\rhl}[1]{\index{Red Hat Linux #1@{\sc{Red Hat Linux #1}}}{\sc{Red Hat Linux #1}}}
\newcommand{\fcore}[1]{\index{Fedora Core #1@{\sc{Fedora Core #1}}}{\sc{Fedora Core #1}}}
\newcommand{\ubuntu}[1]{\index{Ubuntu #1@{\sc{Ubuntu #1}}}{\sc{Ubuntu #1}}}
\newcommand{\debian}[1]{\index{Debian #1@{\sc{Debian #1}}}{\sc{Debian #1}}}
\newcommand{\solaris}[1]{\index{Solaris #1@{\sc{Solaris #1}}}{\sc{Solaris #1}}}
\newcommand{\win}[1]{\index{Windows #1@{\sc{Windows #1}}}{\sc{Windows #1}}}


% Joey:

\newtheorem{defn}{Definition}
\newtheorem{axioms}{Axiom}
% \newtheorem{lemma}{Lemma}
\newtheorem{lemmas}{Lemma}
% \newcommand{\web}{{WWW}}
\newcommand{\wwweb}{{WWW}}
\newcommand{\bic}{{\index{BIC}BIC}}
\newcommand{\mni}{{\index{MNI}MNI}}
\newcommand{\nfs}{{\index{NFS}NFS}}
\newcommand{\crim}{{\index{CRIM}CRIM}}
\newcommand{\animal}{\index{Animal@{\sc{Animal}}}{\sc{Animal}}}
\newcommand{\paranimal}{\index{Paranimal@{\sc{ParAnimal}}}{\sc{ParAnimal}}}
\newcommand{\minc}{{\sc{MINC}}}
\newcommand{\netcdf}{{\sc{NetCDF}}}
\newcommand{\sgi}{{\index{SGI}}SGI}
\newcommand{\vv}{{\tt{*var}}}
\newcommand{\vd}{{\tt{?var}}}
\newcommand{\tv}{{\tt{*term}}}
\newcommand{\td}{{\tt{?term}}}
\newcommand{\fv}{{\tt{*fn}}}
\newcommand{\fd}{{\tt{?fn}}}
\newcommand{\home}{{\tt{home}}}
\newcommand{\light}{{\tt{light}}}
\newcommand{\heavy}{{\tt{heavy}}}
\newcommand{\lucidA}[1]{${\mathit{Lucid}}(#1)$}
\newcommand{\lucidL}[1]{{$\mathit{Lucid}$}($L$) }
\newcommand{\tristan}{\index{Tristan}Tristan}
\newcommand{\commercial}[1]{#1}
\newcommand{\al}{\mbox{$\alpha$}}
\newcommand{\be}{\mbox{$\beta$}}
\newcommand{\ga}{\mbox{$\gamma$}}
\newcommand{\vx}[1]{\mbox{$\overrightarrow{#1}$}}
\newcommand{\lvx}[1]{\mbox{$\mid\!\!\overrightarrow{#1}\!\!\mid$}}
\newcommand{\svx}[1]{{\small \mbox{$\overrightarrow{#1}$}}}
\newcommand{\curl}[1]{\nabla\times\;\mathbf{#1}}
\newcommand{\components}[3]{{_{#3}}{#2}_{#1}}
\newcommand{\componentsp}[3]{{_{#3}}{#2}'_{#1}}
\newcommand{\mypageheader}[1]{\vspace*{22mm}{\Huge \bf #1}\vspace*{5mm}}
\newcommand{\myfig}[1]{\center{\makebox[\textwidth]{\hbox{\vbox{\epsfbox{#1}}}}}}
%\newcommand{\myfig}[1]{\makebox[\textwidth]{\hbox{\vbox{60mm}}}}
\newcommand{\ctxt}{{\mathcal L},{\mathcal D},{\mathcal P},{\mathcal W}}
\newcommand{\noWctxt}{{\mathcal L},{\mathcal D},{\mathcal P}}
\newcommand{\myvdash}{\:\vdash\:}
\newcommand{\mysemi}{\::\:}
\newcommand{\Spc}          {{\mathcal{S}}}
\newcommand{\corner}[1]    {\ulcorner #1\urcorner}
\newcommand{\db}[1]        {\{#1\}}
\newcommand{\mtt}[1]       {{\mathtt{#1}}}
\newcommand{\mrm}[1]       {{\mathrm{#1}}}
\newcommand{\mem}[1]       {{\mathit{#1}}}

\newcommand{\mathfbyd}     {{\mathtt{fby.d}}}
\newcommand{\mathfirstd}   {{\mathtt{first.d}}}
\newcommand{\mathnextd}    {{\mathtt{next.d}}}
\newcommand{\mathprevd}    {{\mathtt{prev.d}}}
\newcommand{\mathwvrd}     {{\mathtt{wvr.d}}}
\newcommand{\mathasad}     {{\mathtt{asa.d}}}
\newcommand{\mathupond}    {{\mathtt{upon.d}}}
\newcommand{\mathfby}      {{\mathtt{fby}}}
\newcommand{\mathbefore}   {{\mathtt{before}}}
\newcommand{\mathfirst}    {{\mathtt{first}}}
\newcommand{\mathnext}     {{\mathtt{next}}}
\newcommand{\mathprev}     {{\mathtt{prev}}}
\newcommand{\mathwvr}      {{\mathtt{wvr}}}
\newcommand{\mathasa}      {{\mathtt{asa}}}
\newcommand{\mathupon}     {{\mathtt{upon}}}
\newcommand{\mathif}       {{\mathtt{if}}}
\newcommand{\maththen}     {{\mathtt{then}}}
\newcommand{\mathelse}     {{\mathtt{else}}}
\newcommand{\mathfi}       {{\mathtt{fi}}}
\newcommand{\mathatd}      {{\mathtt{@.d}}}
\newcommand{\mathat}       {{\mathtt{@.}}}
\newcommand{\mathtagd}     {{\mathtt{\#.d}}}
\newcommand{\mathtag}      {{\mathtt{\#.}}}
\newcommand{\mathwhere}    {{\mathtt{where}}}
\newcommand{\mathdimension}{{\mathtt{dimension}}}
\newcommand{\mathhome}	   {{\mathtt{home}}}
\newcommand{\mathheavy}	   {{\mathtt{heavy}}}
\newcommand{\mathlight}	   {{\mathtt{light}}}
\newcommand{\mathiseod}    {{\mathtt{iseod}}}
\newcommand{\mathiserror}  {{\mathtt{iserror}}}
\newcommand{\mathend}      {{\mathtt{end}}}
\newcommand{\matheod}      {{\mathtt{eod}}}
\newcommand{\matherror}    {{\mathtt{error}}}
\newcommand{\mathtrue}     {{\mathtt{true}}}
\newcommand{\mathfalse}    {{\mathtt{false}}}
\newcommand{\Ek}           {${\mathbf{E_{k}}}$}
\newcommand{\Eop}           {${\mathbf{E_{op}}}$}
\newcommand{\Eid}           {${\mathbf{E_{id}}}$}
\newcommand{\Efid}          {${\mathbf{E_{fid}}}$}
\newcommand{\Econdt}        {${\mathbf{E_{c_{T}}}}$}
\newcommand{\Econdf}        {${\mathbf{E_{c_{F}}}}$}
\newcommand{\Ewhere}        {${\mathbf{E_{w}}}$}
\newcommand{\Eat}           {${\mathbf{E_{at}}}$}
\newcommand{\Etag}          {${\mathbf{E_{tag}}}$}
\newcommand{\Qid}           {${\mathbf{Q_{id}}}$}
\newcommand{\Qfid}          {${\mathbf{Q_{fid}}}$}
\newcommand{\QQ}            {${\mathbf{QQ}}$}
\newcommand{\const}        {{\mathit{k}}}
\newcommand{\varid}        {{\mathit{id}}}
\newcommand{\dimid}        {{\mathit{did}}}
\newcommand{\letter}       {{\mathit{letter}}}
\newcommand{\digit}        {{\mathit{digit}}}
\newcommand{\character}    {{\mathit{char}}}
\newcommand{\mystring}     {{\mathit{string}}}
\newcommand{\boolean}      {{\mathit{boolean}}}
\newcommand{\real}         {{\mathit{real}}}
\newcommand{\ASCIIchar}    {{\mathit{ASCIIchar}}}
\newcommand{\alphanum}     {{\mathit{alphanum}}}
\newcommand{\integer}      {{\mathit{integer}}}
\newcommand{\E}            {{\mathit{E}}}
%\renewcommand{\E}            {{\mathit{E}}}
\newcommand{\userfct}      {{\mathit{userfct}}}
\newcommand{\llop}         {{\textit{intensional-op}}}
\newcommand{\luop}         {{\textit{i-unary-op}}}
\newcommand{\lbop}         {{\textit{i-binary-op}}}
\newcommand{\op}           {{\textit{data-op}}}
\newcommand{\uop}          {{\textit{unary-op}}}
\newcommand{\bop}          {{\textit{binary-op}}}
\newcommand{\ifexpr}       {{\mathit{ifexpr}}}
\newcommand{\deflist}      {{\mathit{deflist}}}
%\newcommand{\dimdef}       {{\mathit{dimdef}}}
\newcommand{\fctid  }      {{\mathit{fid}}}
\newcommand{\tensorid}[2]  {{\mathit{tid_{#1}#2}}}
\newcommand{\usc}          {\mathit{\raisebox{0mm}{\_}}}
\newcommand{\dimlist}      {{\mathit{dimlist}}}
\newcommand{\Elist}        {{\mathit{Elist}}}
\newcommand{\simpleuop}    {{\mathit{mathuop}}}
\newcommand{\complexuop}   {{\mathit{intuop}}}
\newcommand{\defmy}        {{\mathit{Q}}}
\newcommand{\paramlist}    {{\mathit{parlist}}}
\newcommand{\id}           {{\mathit{identifier}}}
\newcommand{\Luciduop}     {{\mathit{Luciduop}}}
\newcommand{\Lucidbop}     {{\mathit{Lucidbop}}}
\newcommand{\simplebop}    {{\mathit{mathbop}}}
\newcommand{\complexbop}   {{\mathit{intbop}}}
\newcommand{\arithbop}     {{\textit{arith-op}}}
\newcommand{\relbop}       {{\textit{rel-op}}}
\newcommand{\logbop}       {{\textit{log-op}}}
\newcommand{\bitbop}       {{\textit{bit-op}}}
\newcommand{\seqbop}       {{\textit{seq-op}}}
\newcommand{\B}{\!\!\!\!\!\!\!\!\!\!\!\!\!\!\!\!}
\newcommand{\Bs}{\!\!\!}
\newcommand{\Bt}{\!}
\newcommand{\Dim}{{\mathcal{D}}}
\newcommand{\Point}{{\mathcal{P}}}
\newcommand{\PointP}{{\mathcal{P}}\!\dagger\!}
\newcommand{\Tag}{{\mathcal{T}}}
\newcommand{\Lang}{{\mathcal{L}}}
\newcommand{\Def}{{\mathcal{D}}}
\newcommand{\Ware}{{\mathcal{W}}}
\newcommand{\WareD}[2]{{\mathcal{W}}?\!\left\{[#1]#2\right\}}
\newcommand{\WareP}[3]{{\mathcal{W}}\!\dagger\!\left\{[#1]#2:#3\right\}}
\newcommand{\Id}{{\mathcal{I}}}
\newcommand{\Val}{{\mathcal{V}}}
\newcommand{\Stream}{{\mathcal{I}}}
\newcommand{\Expr}{{\mathcal{E}}}
\newcommand{\allExpr}{\Expr^\infty}
\newcommand{\allDim}{\Delta^\infty}
\newcommand{\allPoint}{\Pi^\infty}
\newcommand{\allStream}{\Stream^\infty}
\newcommand{\allVal}{\Val^\infty}
\newcommand{\allTag}{\Tag^\infty}
\newcommand{\extdef}{\stackrel{ext}{\equiv}}
\newcommand{\Sb}{\mathbf{Sb}}
\newcommand{\Sw}{\mathbf{Sw}}

%\newenvironment{program}
%		{\begin{quote}}
%		{\end{quote}}
\newtheorem{mydef}
		{{\bf Definition:}}
		{}
\newcommand{\paracite}[2]
		{\vspace{0.5cm}
		{\it{#1

		}}
		{\begin{flushright}---#2\end{flushright}}
}
\newcommand{\cutecite}[2]
		{\vspace{0.5cm}
		{\begin{flushright}
		{\it{#1}}\\
		---#2
		\end{flushright}}
}
\newcommand{\sembox}[3]
		{\TR{   \begin{small}
			\begin{tabular}{|p{4mm}|c|}\hline
			$\!\!$#1 & {\tt{#2}}\\\cline{2-2}
		   	   & [#3]\\\hline
			\end{tabular}
			\end{small}

		}}
%\floatstyle{boxed}
%\restylefloat{table}
%\restylefloat{figure}
%\floatname{boxtable}{Table}
%\newfloat{boxtable}{h}{lot}[chapter]


%\newcounter{definition}
%\setcounter{definition}{0}
%\newenvironment{definition}
%{
%\parindent0mm
%\parskip3mm
%\addtocounter{definition}{1}
%{\bf Definition \arabic{definition}}:
%}

%\newcounter{theorem}
%\setcounter{theorem}{0}
%\newenvironment{theorem}
%{
%\parindent0mm
%\parskip3mm
%\addtocounter{theorem}{1}
%{\bf Theorem \arabic{theorem}}:
%}

%\newtheorem{proposition}{Proposition}

\def\mymid{\vrule depth 4pt height 10pt width 0.2mm}
\def\myspace{\hspace*{3mm}}
\def\mymidspace{\mymid\myspace}
\def\myvert{\raise 2.27pt \hbox{\vrule depth 0pt height 8pt width 0.2mm}}
\def\myarrow{\hspace*{0.43mm}%
             \raise 2.29pt\hbox{\vrule depth 0pt height 8pt width 0.16mm}%
             \hspace*{-0.32mm}%
             $\longrightarrow$
             \ %
             }
\def\mmyarrow{$\rightarrow$\ }

%\psset{unit=.75cm}

\newcommand{\johndef}{\mathcal{D}}
\newcommand{\johnjvmdef}{\mathcal{D}_{jvm}}
\newcommand{\johntdef}{\mathcal{T}}
\newcommand{\myid}{\textit{id}}
\newcommand{\mytid}{\textit{tid}}
\newcommand{\mydagger}{\!\dagger\!}
\newcommand{\context}[2]{\mathcal{D},\mathcal{P} \vdash #1 : #2}
\newcommand{\jvmcontext}[2]{\mathcal{D}_{jvm} \vdash #1 : #2}
\newcommand{\pcontext}[2]{\mathcal{D},\mathcal{P},\mathcal{N} \vdash #1 : #2}
\newcommand{\contextW}[2]{\mathcal{D},\mathcal{P},\mathcal{W} \vdash #1 : #2}
\newcommand{\contextWp}[2]{\mathcal{D},\mathcal{P},\mathcal{W}' \vdash #1 : #2}
\newcommand{\qcontext}[2]{\mathcal{D},\mathcal{P} \vdash #1 \::\: #2}
\newcommand{\qjvmcontext}[2]{\mathcal{D}_{jvm} \vdash #1 \::\: #2}
\newcommand{\pqcontext}[2]{\mathcal{D},\mathcal{P},\mathcal{N} \vdash #1 \::\: #2}
\newcommand{\qcontextW}[2]{\mathcal{D},\mathcal{P,\mathcal{W}} \vdash #1 \::\: #2}
\newcommand{\myifthenelse}{\mathtt{if}\;E\;\mathtt{then}\;E'\;\mathtt{else}\;E''}

\def\Lfirst{\index{first@{\texttt{first}}}\texttt{first}\;}
\def\Lnext{\index{next@{\texttt{next}}}\texttt{next}\;}
\def\Lfby{\index{fby@{\texttt{fby}}}\;\texttt{fby}\;}
\def\Lat{\index{a@{\texttt{\char64}}}\;\texttt{\char64}\;}
\def\LSat{\index{a@{\texttt{\char64}}}\texttt{\char64}}
\def\Lhash{\index{a@{\texttt{\char35}}}\texttt{\char35}}
\def\Lwvr{\index{wvr@{\texttt{wvr}}}\;\texttt{wvr}\;}
\def\Lupon{\index{upon@{\texttt{upon}}}\;\texttt{upon}\;}
\def\LSupon{\index{upon@{\texttt{upon}}}\;\texttt{upon}}
\def\Lasa{\index{asa@{\texttt{asa}}}\;\texttt{asa}\;}
\def\Leod{\index{eod@{\texttt{eod}}}\texttt{eod}}
\def\Liseod{\index{iseod@{\texttt{iseod}}}\texttt{iseod}}
\def\Lif{\index{ifthenelse@{\texttt{if then else}}}\texttt{if}\;}
\def\Lthen{\;\texttt{then}\;}
\def\Lelse{\;\texttt{else}\;}
\def\Lsif{\index{ifthenelse@{\texttt{if then else}}}\texttt{\scriptsize if}\;}
\def\Lsthen{\;\texttt{\scriptsize then}\;}
\def\Lselse{\;\texttt{\scriptsize else}\;}

\def\mufirst{\index{first@{\texttt{first}}}\mathrm{\underline{\mathtt{first}}}\;}
\def\munext{\index{next@{\texttt{next}}}\mathrm{\underline{\mathtt{next}}}\;}
\def\mufby{\index{fby@{\texttt{fby}}}\;\mathrm{\underline{\mathtt{fby}}}\;}
\def\muwvr{\index{wvr@{\texttt{wvr}}}\;\mathrm{\underline{\mathtt{wvr}}}\;}
\def\muupon{\index{upon@{\texttt{upon}}}\;\mathrm{\underline{\mathtt{upon}}}\;}
\def\muasa{\index{asa@{\texttt{asa}}}\;\mathrm{\underline{\mathtt{asa}}}\;}

\def\mfirst{\index{first@{\texttt{first}}}\mathrm{{\mathtt{first}}}\;}
\def\mprev{\index{prev@{\texttt{prev}}}\mathrm{{\mathtt{prev}}}\;}
\def\mnext{\index{next@{\texttt{next}}}\mathrm{{\mathtt{next}}}\;}
\def\mfby{\index{fby@{\texttt{fby}}}\;\mathrm{{\mathtt{fby}}}\;}
\def\mwvr{\index{wvr@{\texttt{wvr}}}\;\mathrm{{\mathtt{wvr}}}\;}
\def\mupon{\index{upon@{\texttt{upon}}}\;\mathrm{{\mathtt{upon}}}\;}
\def\masa{\index{asa@{\texttt{asa}}}\;\mathrm{{\mathtt{asa}}}\;}

\def\Tfirst{\index{first@{\texttt{first}}}\texttt{first}}
\def\Tnext{\index{next@{\texttt{next}}}\texttt{next}}
\def\Tfby{\index{fby@{\texttt{fby}}}\texttt{fby}}
\def\Twvr{\index{wvr@{\texttt{wvr}}}\texttt{wvr}}
\def\Tupon{\index{upon@{\texttt{upon}}}\texttt{upon}}
\def\Tasa{\index{asa@{\texttt{asa}}}\texttt{asa}}

\newcommand{\eqdef}{\stackrel{{\mathrm{def}}}{=}}
\newcommand{\mylinebefore}{\noindent\rule{.1mm}{3mm}\rule[3mm]{.995\textwidth}{.1mm}\rule{.1mm}{3mm}\vspace*{-5mm}}
\newcommand{\mylineafter}{\vspace*{-5mm}\noindent\rule{.1mm}{3mm}\rule{.995\textwidth}{.1mm}\rule{.1mm}{3mm}}
\newcommand{\myprop}[1]{
\mylinebefore
\begin{proposition}
#1
\end{proposition}
\mylineafter
}


%% Document
%%
\begin{document}

% ------------------------------------------------------------------------------
%% Front Matter
%%
% Regular title as in the article class.
%
\title{Speed: The GCS ENCS Cluster}

% \titlerunning{} has to be set to either the main title or its shorter
% version for the running heads. Use {\sf} for highlighting your system
% name, application, or a tool.
%
\titlerunning{Speed: The GCS ENCS Cluster}

% Previously VI
%\date{Version 6.5}
%\date{\textbf{Version 6.6-dev-07}}
%\date{\textbf{Version 6.6} (final GE version)}
\date{\textbf{Version 7.0-dev-01} (final GE version)}

% Authors are joined by \and and their affiliations are on the
% subsequent lines separated by \\ just like the article class
% allows.
%
\author{
    Serguei A. Mokhov
\and
    Gillian A. Roper
\and
    Network, Security and HPC Group\footnote{The group acknowledges the initial manual version VI produced by Dr.~Scott Bunnell while with us
		as well as Dr.~Tariq Daradkeh for his instructional support of the users and contribution of examples.}\\
    \affiliation{Gina Cody School of Engineering and Computer Science}\\
    \affiliation{Concordia University}\\
    \affiliation{Montreal, Quebec, Canada}\\
    \affiliation{\url{rt-ex-hpc~AT~encs.concordia.ca}}\\
}

% \authorrunning{} has to be set for the shorter version of the authors' names;
% otherwise a warning will be rendered in the running heads.
%
\authorrunning{Mokhov, Roper, NAG/HPC, GCS ENCS}
\indexedauthor{Mokhov, Serguei}
\indexedauthor{Roper, Gillian}
\indexedauthor{NAG/HPC}

%%%%%%%%%%%%%%%%%%%%%%%%%%%%%%%%%%%%%%%%%%%%%%%%%%%
\maketitle
%%%%%%%%%%%%%%%%%%%%%%%%%%%%%%%%%%%%%%%%%%%%%%%%%%%

% ------------------------------------------------------------------------------
\begin{abstract}
This document presents a quick start guide to the usage of the Gina Cody School 
of Engineering and Computer Science compute server farm called ``Speed'' -- the 
GCS Speed cluster, managed by the HPC/ NAG group of the Academic Information 
Technology Services (AITS) at GCS, Concordia University, Montreal, Canada.
\end{abstract}

% ------------------------------------------------------------------------------
\tableofcontents
\clearpage

% ------------------------------------------------------------------------------
\section{Introduction}

This document contains basic information required to use ``Speed'' as well as 
tips and tricks, examples, and references to projects and papers that have used Speed. 
User contributions of sample jobs and/ or references are welcome. 
Details are sent to the \texttt{hpc-ml} mailing list.

% ------------------------------------------------------------------------------
\subsection{Resources}

\begin{itemize}
\item
Our public GitHub page where the manual and sample job scripts
are maintained (pull-requests (PRs), subject to review, are welcome):\\
\url{https://github.com/NAG-DevOps/speed-hpc}\\
\url{https://github.com/NAG-DevOps/speed-hpc/pulls}

\item
PDF version of this manual:\\
\url{https://github.com/NAG-DevOps/speed-hpc/blob/master/doc/speed-manual.pdf}\\
HTML version of this manual:\\
\url{https://nag-devops.github.io/speed-hpc/}

\item
Our official Concordia page for the ``Speed'' cluster:\\
\url{https://www.concordia.ca/ginacody/aits/speed.html}\\
which includes access request instructions.

\item
All Speed users are subscribed to the \texttt{hpc-ml} mailing
list.

\item
\href
	{https://docs.google.com/presentation/d/1zu4OQBU7mbj0e34Wr3ILXLPWomkhBgqGZ8j8xYrLf44}
	{Speed Server Farm Presentation 2022}~\cite{speed-intro-preso}.

\end{itemize}

% ------------------------------------------------------------------------------
\subsection{Team}

\begin{itemize}
	\item 
Serguei Mokhov, PhD, Manager, Networks, Security and HPC, AITS
	\item 
Gillian Roper, Senior Systems Administrator, HPC, AITS
	\item 
Carlos Alarcón Meza, Systems Administrator, HPC and Networking, AITS
	%\item 
%Tariq Daradkeh, PhD, IT Instructional Specialist, Information Technology
\end{itemize}

We receive support from the rest of AITS teams, such as NAG, SAG, FIS, and DOG.

% ------------------------------------------------------------------------------
\subsection{What Speed Consists of}

\begin{itemize}
\item
Twenty four (24) 32-core compute nodes, each with 512~GB of memory and 
approximately 1~TB of local volatile-scratch disk space. 

\item
Twelve (12) NVIDIA Tesla P6 GPUs, with 16~GB of memory (compatible with the 
CUDA, OpenGL, OpenCL, and Vulkan APIs). 

\item
4 VIDPRO nodes, with 6 P6 cards, and 6 V100 cards (32GB), and 
256GB of RAM.

\item
7 new SPEED2 servers with 4x A100 80GB GPUs each, partitioned
into 4x 20GB each.

\item
One AMD FirePro S7150 GPUs, with 8~GB of memory (compatible with the
Direct~X, OpenGL, OpenCL, and Vulkan APIs). 

\end{itemize}

% ------------------------------------------------------------------------------
\subsection{What Speed Is Ideal For}
\label{sect:speed-is-for}

\begin{itemize}
\item
To design and develop, test and run parallel, batch, and other algorithms, scripts with partial data sets. ``Speed'' has been optimised for compute jobs that are multi-core aware, require a large memory space, or are iteration intensive.
\item
Prepare them for big clusters:
	\begin{itemize}
	\item 
	Digital Research Alliance of Canada (Calcul Quebec and Compute Canada)
	\item 
	Cloud platforms
	\end{itemize}
\item
Jobs that are too demanding for a desktop. 
\item
Single-core batch jobs; multithreaded jobs up to 32 cores (i.e., a single machine).
\item
Anything that can fit into a 500-GB memory space and a scratch space of approximately ~10TB. 
\item
CPU-based jobs. 
\item
CUDA GPU jobs (\texttt{speed-05}, \texttt{speed-17}, \texttt{speed-37}--\texttt{speed-43}).
\item
Non-CUDA GPU jobs using OpenCL (\texttt{speed-19} and \texttt{speed-05|17}).
\end{itemize}

% ------------------------------------------------------------------------------
\subsection{What Speed Is Not}
\label{sect:speed-is-not}

\begin{itemize}
\item Speed is not a web host and does not host websites.
\item Speed is not meant for Continuous Integration (CI) automation deployments for Ansible or similar tools. 
\item Does not run Kubernetes or other container orchestration software.
\item Does not run Docker. (Note: Speed does run Singularity and many Docker containers can be converted to Singularity containers with a single command.)
\item Speed is not for jobs executed outside of the scheduler. (Jobs running outside of the scheduler will be killed and all data lost.)
\end{itemize}

% ------------------------------------------------------------------------------
\subsection{Available Software}

We have a great number of open-source software available and installed
on ``Speed''~--~various Python, CUDA versions, {\cpp}/{\java} compilers, OpenGL,
OpenFOAM, OpenCV, TensorFlow, OpenMPI, OpenISS, {\marf}~\cite{marf}, etc.
There are also a number of commercial packages, subject to licensing
contributions, available, such as MATLAB~\cite{matlab,scholarpedia-matlab}, Abaqus~\cite{abaqus}, 
Ansys, Fluent~\cite{fluent}, etc. 

To see the packages available, run \texttt{ls -al /encs/pkg/} on \texttt{speed.encs}.

In particular, there are over 2200 programs available in
\texttt{/encs/bin} and \texttt{/encs/pkg} under Scientific Linux 7 (EL7).
We are building an equivalent array of programs for the EL9 SPEED2 nodes.

\begin{itemize}
	\item 
Popular concrete examples:
\begin{itemize}
	\item 
MATLAB (R2016b, R2018a, R2018b)
	\item 
Fluent (19.2)
	\item 
Singularity containers can run other operating systems and linux distributions, like Ubuntu's, as well as converted Docker containers.
\end{itemize}
	\item 
We do our best to accommodate custom software requests.
Python environments can use user-custom installs 
from within the scratch directory.
	\item 
A number of specific environments are available and 
can be loaded using the \tool{module} command:
\begin{itemize}
	\item 
Python (2.3.0 - 3.5.1)
	\item 
Gurobi (7.0.1, 7.5.0, 8.0.0, 8.1.0)
	\item 
Ansys (16, 17, 18, 19)
	\item 
OpenFOAM (2.3.1, 3.0.1, 5.0, 6.0)
	\item 
Cplex 12.6.x to 12.8.x
	\item 
OpenMPI 1.6.x, 1.8.x, 3.1.3
\end{itemize}
\end{itemize}

% ------------------------------------------------------------------------------
\subsection{Requesting Access}

After reviewing the ``What Speed is'' (\xs{sect:speed-is-for}) and
``What Speed is Not'' (\xs{sect:speed-is-not}), request access to the ``Speed'' 
cluster by emailing: \texttt{rt-ex-hpc AT encs.concordia.ca}.
%
Faculty and staff may request access directly.
Students must include the following in their message:

\begin{itemize} 
	\item GCS ENCS username
	\item Name and email (CC) of the supervisor or instructor
	\item Written request from the supervisor or instructor for the ENCS username to be granted access to ``Speed''
\end{itemize}

% ------------------------------------------------------------------------------
\section{Job Management}
\label{sect:job-management}

In these instructions, anything bracketed like so, \verb+<>+, indicates a
label/value to be replaced (the entire bracketed term needs replacement).

% ------------------------------------------------------------------------------
\subsection{Getting Started}

Before getting started, please review the ``What Speed is'' (\xs{sect:speed-is-for})
and ``What Speed is Not'' (\xs{sect:speed-is-not}).
Once your GCS ENCS account has been granted access to ``Speed'',
use your GCS ENCS account credentials to create an SSH connection to 
\texttt{speed} (an alias for \texttt{speed-submit.encs.concordia.ca}). 

All users are expected to have a basic understanding of
 Linux and its commonly used commands. 
 % ------------------------------------------------------------------------------
\subsubsection{SSH Connections}
\label{sect:ssh}

Requirements to create connections to Speed:
\begin{enumerate}
	\item
An active \textbf{ENCS user account} which has permission to connect to Speed.
	\item
If you are off campus, an active connection to Concordia's VPN.
Accessing Concordia's VPN requires a Concordia \textbf{netname}. 
	\item
Windows systems require a terminal emulator such as PuTTY, Cygwin (or MobaXterm). 
\end{enumerate}

Open up a terminal window and type in the following SSH command being sure to replace
\verb!<ENCSusername>! with your ENCS account's username.

\begin{verbatim}
ssh <ENCSusername>@speed.encs.concordia.ca
\end{verbatim}

% ------------------------------------------------------------------------------
% TMP scheduler-specific section
% ------------------------------------------------------------------------------
\subsubsection{Environment Set Up}
\label{sect:envsetup}

After creating an SSH connection to Speed, you will need to
make sure the \tool{srun}, \tool{sbatch}, and \tool{salloc}
commands are available to you. 
Type the command name at the command prompt and press enter.
If the command is not available, e.g., (``command not found'') is returned,
you need to make sure your \api{\$PATH} has \texttt{/local/bin} in it.
To view your \api{\$PATH} type \texttt{echo \$PATH} at the prompt.
%
%source 
%the ``Altair Grid Engine (AGE)'' scheduler's settings file. 
%Sourcing the settings file will set the environment variables required to 
%execute scheduler commands.
%
%Based on the UNIX shell type, choose one of the following commands to source
%the settings file. 
%
%csh/\tool{tcsh}:
%\begin{verbatim}
%source /local/pkg/uge-8.6.3/root/default/common/settings.csh 
%\end{verbatim}
%
%Bourne shell/\tool{bash}:
%\begin{verbatim}
%. /local/pkg/uge-8.6.3/root/default/common/settings.sh 
%\end{verbatim}
%
%In order to set up the default ENCS bash shell, executing the following command 
%is also required:
%\begin{verbatim}
%printenv ORGANIZATION | grep -qw ENCS || . /encs/Share/bash/profile 
%\end{verbatim}
%
%To verify that you have access to the scheduler commands execute 
%\texttt{qstat -f -u "*"}. If an error is returned, attempt sourcing 
%the settings file again.

The next step is to copy a job template to your home directory and to set up your
cluster-specific storage. Execute the following command from within your
home directory. (To move to your home directory, type \texttt{cd} at the Linux
prompt and press \texttt{Enter}.) 

\begin{verbatim}
cp /home/n/nul-uge/template.sh . && mkdir /speed-scratch/$USER
\end{verbatim}

%\textbf{Tip:} Add the source command to your shell-startup script. 

\textbf{Tip:} the default shell for GCS ENCS users is \tool{tcsh}.
If you would like to use \tool{bash}, please contact 
\texttt{rt-ex-hpc AT encs.concordia.ca}.

%For \textbf{new GCS ENCS Users}, and/or those who don't have a shell-startup script, 
%based on your shell type use one of the following commands to copy a start up script 
%from \texttt{nul-uge}'s home directory to your home directory. (To move to your home
%directory, type \tool{cd} at the Linux prompt and press \texttt{Enter}.)

%csh/\tool{tcsh}:
%\begin{verbatim}
%cp /home/n/nul-uge/.tcshrc . 
%\end{verbatim}

%Bourne shell/\tool{bash}:
%\begin{verbatim}
%cp /home/n/nul-uge/.bashrc . 
%\end{verbatim}

%Users who already have a shell-startup script, can use a text editor, such as
%\tool{vim} or \tool{emacs}, to add the source request to your existing
%shell-startup environment (i.e., to the \file{.tcshrc} file in your home directory). 

%csh/\tool{tcsh}:
%Sample \file{.tcshrc} file:
%\begin{verbatim}
%# Speed environment set up 
%if ($HOSTNAME == speed-submit.encs.concordia.ca) then
   %source /local/pkg/uge-8.6.3/root/default/common/settings.csh
%endif
%\end{verbatim}
%
%Bourne shell/\tool{bash}:
%Sample \file{.bashrc} file:
%\begin{verbatim}
%# Speed environment set up 
%if [ $HOSTNAME = "speed-submit.encs.concordia.ca" ]; then
    %. /local/pkg/uge-8.6.3/root/default/common/settings.sh
    %printenv ORGANIZATION | grep -qw ENCS || . /encs/Share/bash/profile
%fi
%\end{verbatim}

Note, if you are getting ``command not found'' error(s) when logging in, you
probably have old Grid Engine environment commands. Remove them
as per \xa{appdx:uge-to-slurm}.


% ------------------------------------------------------------------------------
\subsection{Job Submission Basics}

Preparing your job for submission is fairly straightforward. Start by basing your job script on one of the examples available in the ``src''
directory of our GitHub's (\url{https://github.com/NAG-DevOps/speed-hpc}).

Job scripts are broken into four main sections: 
\begin{itemize}
	\item Directives
	\item Module Loads
	\item User Scripting
\end{itemize}

% ------------------------------------------------------------------------------
% TMP scheduler-specific section
% ------------------------------------------------------------------------------
\subsubsection{Directives}
\label{sect:directives}

Directives are comments included at the beginning of a job script that set the shell 
and the options for the job scheduler. 

The shebang directive is always the first line of a script. In your job script, 
this directive sets which shell your script's commands will run in. On ``Speed'', 
we recommend that your script use a shell from the \texttt{/encs/bin} directory. 

To use the \texttt{tcsh} shell, start your script with: \verb|#!/encs/bin/tcsh|

For \texttt{bash}, start with: \verb|#!/encs/bin/bash|

Directives that start with \verb|#SBATCH|, set the options for the cluster's 
SLURM scheduler. The script template, \file{template.sh}, 
provides the essentials:

%\begin{verbatim}
%#$ -N <jobname>
%#$ -cwd
%#$ -m bea
%#$ -pe smp <corecount>
%#$ -l h_vmem=<memory>G
%\end{verbatim}
\begin{verbatim}
#SBATCH --job-name=tmpdir           ## Give the job a name
#SBATCH --mail-type=ALL             ## Receive all email type notifications
#SBATCH --mail-user=$USER@encs.concordia.ca
#SBATCH --chdir=./                  ## Use current directory as working directory
#SBATCH --nodes=1
#SBATCH --ntasks=1
#SBATCH --cpus-per-task=<corecount> ## Request, e.g. 8 cores
#SBATCH --mem=<memory>              ## Assign, e.g., 32G memory per node 
\end{verbatim}

and its short option equivalents:

\begin{verbatim}
#SBATCH -J tmpdir                   ## Give the job a name
#SBATCH --mail-type=ALL             ## Receive all email type notifications
#SBATCH --mail-user=$USER@encs.concordia.ca
#SBATCH --chdir=./                  ## Use current directory as working directory
#SBATCH -N 1
#SBATCH --ntasks=1
#SBATCH -n 8                        ## Request 8 cores
#SBATCH --mem=32G                   ## Assign 32G memory per node 
\end{verbatim}

Replace, \verb+<jobname>+, with the name that you want your cluster job to have;
\option{--chdir}, makes the current working directory the ``job working directory'',
and your standard output file will appear here; \option{--mail-type}, provides e-mail
notifications (success, error, etc. or all); replace, \verb+<corecount>+, with the degree of
(multithreaded) parallelism (i.e., cores) you attach to your job (up to 32 by default).
%be sure to delete or comment out the \verb| #$ -pe smp | parameter if it 
%is not relevant;
Replace, \verb+<memory>+, with the value (in GB), that you want 
your job's memory space to be (up to 500 depending on the node), and all jobs MUST have a memory-space 
assignment.

If you are unsure about memory footprints, err on assigning a generous
memory space to your job, so that it does not get prematurely terminated.
%(the value given to \api{h\_vmem} is a hard memory ceiling).
You can refine
%\api{h\_vmem}
\option{--mem}
values for future jobs by monitoring the size of a job's active
memory space on \texttt{speed-submit} with:

%\begin{verbatim}
%qstat -j <jobID> | grep maxvmem
%\end{verbatim}

\begin{verbatim}
sstat -j <jobID>
\end{verbatim}

\noindent
This can be customized to show specific columns:

\begin{verbatim}
sstat -o jobid,maxvmsize,ntasks%7,tresusageouttot%25 -j <jobID>
\end{verbatim}

Memory-footprint values are also provided for completed jobs in the final
e-mail notification (as, ``maxvmsize'').

\emph{Jobs that request a low-memory footprint are more likely to load on a busy
cluster.}


% ------------------------------------------------------------------------------
\subsubsection{Module Loads}

As your job will run on a compute or GPU ``Speed'' node, and not the submit node,
any software that is needed must be loaded by the job script. Software is loaded
within the script just as it would be from the command line.

To see a list of which modules are available, execute the following from the 
command line on \texttt{speed-submit}.

\begin{verbatim}
module avail
\end{verbatim}

To list for a particular program (\tool{matlab}, for example):

\begin{verbatim}
module -t avail matlab
\end{verbatim}

Which, of course, can be shortened to match all that start with a
particular letter:

\begin{verbatim}
module -t avail m
\end{verbatim}

Insert the following in your script to load the \tool{matlab/R2020a}) module:

\begin{verbatim}
module load matlab/R2020a/default
\end{verbatim}

Use, \option{unload}, in place of, \option{load}, to remove a module from active use.

To list loaded modules:

\begin{verbatim}
module list
\end{verbatim}

To purge all software in your working environment:

\begin{verbatim}
module purge
\end{verbatim}

Typically, only the \texttt{module load} command will be used in your script.

% ------------------------------------------------------------------------------
% TMP scheduler-specific section
%  2.2.3 User Scripting
% -------------------
% TMP scheduler-specific section

The final part of the job script involves the commands that will be executed by the job.
This section should include all necessary commands to set up and run the tasks 
your script is designed to perform. You can use any Linux command in this section, 
ranging from a simple executable call to a complex loop iterating through multiple commands.\\

\noindent \textbf{Best Practice}: prefix any compute-heavy step with \tool{srun}.
This ensures you gain proper insights on the execution of your job.\\

\noindent Each software program may have its own execution framework, as it's the script's author (e.g., you) 
responsibility to review the software's documentation to understand its requirements.
Your script should be written to clearly specify the location of input and output files and the degree of parallelism needed.\\

\noindent Jobs that involve multiple interactions with data input and output files, should make use of \api{TMPDIR}, 
a scheduler-provided workspace nearly 1~TB in size.
\api{TMPDIR} is created on the local disk of the compute node at the start of a job, offering faster I/O operations 
compared to shared storage (provided over NFS).

An sample job script using \api{TMPDIR} is available at \texttt{/home/n/nul-uge/templateTMPDIR.sh}: 
the job is instructed to change to \api{\$TMPDIR}, to make the new directory \texttt{input}, to copy data from
\texttt{\$SLURM\_SUBMIT\_DIR/references/} to \texttt{input/} (\api{\$SLURM\_SUBMIT\_DIR} represents the
current working directory), to make the new directory \texttt{results}, to
execute the program (which takes input from \texttt{\$TMPDIR/input/} and writes
output to \texttt{\$TMPDIR/results/}), and finally to copy the total end results
to an existing directory, \texttt{processed}, that is located in the current
working directory.
% TODO: verify:
\api{TMPDIR} only exists for the duration of the job, though,
so it is very important to copy relevant results from it at job's end.

% -------------- 2.3 Sample Job Script ----------------------
% -----------------------------------------------------------
\subsection{Sample Job Script}
\label{sect:sample-job-script}

Here's a basic job script, \file{tcsh.sh} shown in \xf{fig:tcsh.sh}.
You can copy it from our \href{https://github.com/NAG-DevOps/speed-hpc}{GitHub repository}.

\begin{figure}[htpb]
	\lstinputlisting[language=csh,frame=single,basicstyle=\ttfamily]{tcsh.sh}
	\caption{Source code for \file{tcsh.sh}}
	\label{fig:tcsh.sh}
\end{figure}

\noindent
The first line is the shell declaration (also know as a shebang) and sets the shell to \emph{tcsh}.
The lines that begin with \texttt{\#SBATCH} are directives for the scheduler.
\begin{itemize}
	\item \option{-J} (or \option{--job-name}) sets \emph{tcsh-test} as the job name.
	%\item \texttt{--chdir} tells the scheduler to execute the job from the current working directory
	\item \option{--mem=1GB} requests and assigns 1GB of memory to the job. 
	Jobs require the \option{--mem} option to be set either in the script
	or on the command line; \textbf{if it's missing, job submission will be rejected.}
\end{itemize}

\noindent The script then:
\begin{enumerate}
	\item Sleeps on a node for 30 seconds.
	\item Uses the \tool{module} command to load the \texttt{gurobi/8.1.0} environment.
	\item Prints the list of loaded modules into a file.
\end{enumerate}

\noindent
The scheduler command, \tool{sbatch}, is used to submit (non-interactive) jobs. 
From an ssh session on ``speed-submit'', submit this job with
\begin{verbatim}
    sbatch ./tcsh.sh
\end{verbatim}

\noindent
You will see, \texttt{Submitted batch job 2653} where $2653$ is a job ID assigned.
The commands \tool{squeue} and \tool{sinfo} can be used 
to look at the status of the cluster:
%\texttt{squeue -l} and \texttt{sinfo -la}.

\small
\begin{verbatim}
[serguei@speed-submit src] % squeue -l
Thu Oct 19 11:38:54 2023
JOBID PARTITION     NAME     USER    STATE       TIME TIME_LIMI  NODES NODELIST(REASON)
 2641        ps interact   b_user  RUNNING   19:16:09 1-00:00:00      1 speed-07
 2652        ps interact   a_user  RUNNING      41:40 1-00:00:00      1 speed-07
 2654        ps tcsh-tes  serguei  RUNNING       0:01 7-00:00:00      1 speed-07
[serguei@speed-submit src] % sinfo
PARTITION AVAIL  TIMELIMIT  NODES  STATE NODELIST
ps*          up 7-00:00:00     14  drain speed-[08-10,12,15-16,20-22,30-32,35-36]
ps*          up 7-00:00:00      1    mix speed-07
ps*          up 7-00:00:00      7   idle speed-[11,19,23-24,29,33-34]
pg           up 1-00:00:00      1  drain speed-17
pg           up 1-00:00:00      3   idle speed-[05,25,27]
pt           up 7-00:00:00      7   idle speed-[37-43]
pa           up 7-00:00:00      4   idle speed-[01,03,25,27]
\end{verbatim}
\normalsize

\noindent
\textbf{Remember} that you only have 30 seconds before the job is essentially over, so 
if you do not see a similar output, either adjust the sleep time in the 
script, or execute the \tool{squeue} statement more quickly. The \tool{squeue} 
output listed above shows that your job 2654 is running on node \texttt{speed-07}, 
and its time limit is 7 days, etc.\\
% TODO
%, that it 
%was started at 16:39:30 on 12/03/2018, and that it is a single-core job (the 
%default). 

Once the job finishes, there will be a new file in the directory that the job 
was started from, with the syntax of, \texttt{slurm-<job id>.out}, so 
in this example the file is, \file{slurm-2654.out}. This file represents the 
standard output (and error, if there is any) of the job in question. If you 
look at the contents of your newly created file, you will see that it 
contains the output of the, \texttt{module list} command. 
Important information is often written to this file.
%
%Congratulations on your first job! 

% -------------- 2.4 Common Job Management Commands Summary ---
% -------------------------------------------------------------
\subsection{Common Job Management Commands Summary}
\label{sect:job-management-commands}

Here is a summary of useful job management commands for handling various aspects of 
job submission and monitoring on the Speed cluster:

\begin{itemize}
	\item Submitting a job:
	\small
	\begin{verbatim}
		sbatch -A <ACCOUNT> -t <MINUTES> --mem=<MEMORY> -p <PARTITION> ./<myscript>.sh
	\end{verbatim}
	\normalsize

	\item Checking your job(s) status:
	\small
	\begin{verbatim}
		squeue -u <ENCSusername>
	\end{verbatim}
	\normalsize

	\item Displaying cluster status:
	\small
	\begin{verbatim}
		squeue
	\end{verbatim}
	\normalsize
		\begin{itemize}
			\item Use \option{-A} for per account (e.g., \texttt{-A vidpro}, \texttt{-A aits}), 
			\item Use \option{-p} for per partition (e.g., \texttt{-p ps}, \texttt{-p pg}, \texttt{-p pt}), etc.
		\end{itemize}

	\item Displaying job information:
	\small
	\begin{verbatim}
		squeue --job <job-ID>
	\end{verbatim}
	\normalsize

	\item Displaying individual job steps: (to see which step failed if you used \tool{srun})
	\small
	\begin{verbatim}
		squeue -las
	\end{verbatim}
	\normalsize

	\item Monitoring job and cluster status: (view \tool{sinfo} and watch the queue for your job(s))
	\small
	\begin{verbatim}
		watch -n 1 "sinfo -Nel -pps,pt,pg,pa && squeue -la"
	\end{verbatim}
	\normalsize

	\item Canceling a job:
	\small
	\begin{verbatim}
		scancel <job-ID>
	\end{verbatim}
	\normalsize

	\item Holding a job:
	\small
	\begin{verbatim}
		scontrol hold <job-ID>
	\end{verbatim}
	\normalsize

	\item Releasing a job:
	\small
	\begin{verbatim}
		scontrol release <job-ID>
	\end{verbatim}
	\normalsize

	\item Getting job statistics: (including useful metrics like ``maxvmem'')
	\small
	\begin{verbatim}
		sacct -j <job-ID>
	\end{verbatim}
	\normalsize
	
	\api{maxvmem} is one of the more useful stats that you can elect to display
	as a format option.
	\small
	\begin{verbatim}
	% sacct -j 2654
	JobID           JobName  Partition    Account  AllocCPUS      State ExitCode
	------------ ---------- ---------- ---------- ---------- ---------- --------
	2654          tcsh-test         ps     speed1          1  COMPLETED      0:0
	2654.batch        batch                speed1          1  COMPLETED      0:0
	2654.extern      extern                speed1          1  COMPLETED      0:0
	% sacct -j 2654 -o jobid,user,account,MaxVMSize,Reason%10,TRESUsageOutMax%30
	JobID             User    Account  MaxVMSize     Reason        TRESUsageOutMax
	------------ --------- ---------- ---------- ---------- ----------------------
	2654           serguei     speed1                  None
	2654.batch                 speed1    296840K             energy=0,fs/disk=1975
	2654.extern                speed1    296312K              energy=0,fs/disk=343
	\end{verbatim}
	\normalsize

	See \texttt{man sacct} or \texttt{sacct -e} for details of the available formatting options. 
	You can define your preferred default format in the \api{SACCT\_FORMAT} environment variable
	in your \texttt{.cshrc} or \texttt{.bashrc} files.

	\item Displaying job efficiency: (including CPU and memory utilization)
	\small
	\begin{verbatim}
	seff <job-ID>
	\end{verbatim}
	\normalsize
	
	Don't execute it on \texttt{RUNNING} jobs (only on completed/finished jobs), else
	efficiency statistics may be misleading. If you define the following 
	directive in your batch script, your GCS ENCS email address will receive an email 
	with \tool{seff}'s output when your job is finished.

	\small
	\begin{verbatim}
	#SBATCH --mail-type=ALL        
	\end{verbatim}
	\normalsize

	Output example:
	\small
	\begin{verbatim}
	Job ID: XXXXX
	Cluster: speed
	User/Group: user1/user1
	State: COMPLETED (exit code 0)
	Nodes: 1
	Cores per node: 4
	CPU Utilized: 00:04:29
	CPU Efficiency: 0.35% of 21:32:20 core-walltime
	Job Wall-clock time: 05:23:05
	Memory Utilized: 2.90 GB
	Memory Efficiency: 2.90% of 100.00 GB
	\end{verbatim}
	\normalsize
\end{itemize}


% -------------- 2.5 Advanced sbatch Options ------------------
% -------------------------------------------------------------
\subsection{Advanced \tool{sbatch} Options}
\label{sect:submit-options}
\label{sect:qsub-options}

In addition to the basic sbatch options presented earlier, 
there are several advanced options that are generally useful:

\begin{itemize}
	\item E-mail notifications:
	\begin{verbatim}
		--mail-type=<TYPE>
	\end{verbatim}
	Requests the scheduler to send an email when the job changes state.
	\texttt{<TYPE>} can be \texttt{ALL}, \texttt{BEGIN}, \texttt{END}, or \texttt{FAIL}.
	Mail is sent to the default address of,
	%
	\begin{verbatim}
	<ENCSusername>@encs.concordia.ca 
	\end{verbatim}
	%
	which you can consult via \url{webmail.encs.concordia.ca} (use VPN from off-campus)
	unless a different address is supplied 
	(see, \option{--mail-user}).
	The report sent when a job ends includes job 
	runtime, as well as the maximum memory value hit (\api{maxvmem}). 
	\begin{verbatim}
		--mail-user email@domain.com
	\end{verbatim}
	Specifies a different email address for notifications rather than the default.

	\item Export environment variables used by the script.:
	\begin{verbatim}
		--export=ALL
		--export=NONE
		--export=VARIABLES
	\end{verbatim}

	\item Job runtime:
	\begin{verbatim}
		-t <MINUTES> or DAYS-HH:MM:SS
	\end{verbatim} 
	sets a job runtime of min or HH:MM:SS. Note that if you give a single number,
	that represents \emph{minutes}, not hours. The set runtime should not exceed
	the default maximums of 24h for interactive jobs and 7 days for batch jobs.

	\item Job Dependencies:
	\begin{verbatim}
		--depend=<state:job-ID>
	\end{verbatim} 
	Runs the job only when the specified job \verb|<job-ID>| finishes. This is useful for creating job chains where 
	subsequent jobs depend on the completion of previous ones.
\end{itemize}

\noindent \textbf{Note:} \tool{sbatch} options can be specified during the job-submission 
command, and these \emph{override} existing script options (if present). The 
syntax is
\begin{verbatim}
	sbatch [options] PATHTOSCRIPT
\end{verbatim}
but unlike in the script, the options are specified without the leading \verb+#SBATCH+
e.g.: 
\begin{verbatim}
	sbatch -J sub-test --chdir=./ --mem=1G ./tcsh.sh
\end{verbatim}

% -------------- 2.6 Array Jobs -------------------------------
% -------------------------------------------------------------
\subsection{Array Jobs}
\label{sect:array-jobs}

Array jobs are those that start a batch job or a parallel job multiple times.
Each iteration of the job array is called a task and receives a unique job ID.
Array jobs are particularly useful for running a large number of similar tasks with slight variations.\\

\noindent
To submit an array job (Only supported for batch jobs), use the \option{--array} option of the \tool{sbatch} 
command as follows:

\begin{verbatim}
	sbatch --array=n-m[:s]] <batch_script>
\end{verbatim}

\noindent \textbf{where}
\begin{itemize}
	\item
	\texttt{n}: indicates the start-id.
	\item
	\texttt{m}: indicates the max-id.
	\item
	\texttt{s}: indicates the step size.
\end{itemize}

\noindent \textbf{Examples:}
\begin{itemize}
	\item Submit a job with 1 task where the task-id is 10. 
	\begin{verbatim}
		sbatch --array=10 array.sh
	\end{verbatim}

	\item Submit a job with 10 tasks numbered consecutively from 1 to 10.
	\begin{verbatim}
		sbatch --array=1-10 array.sh
	\end{verbatim}

	\item Submit a job with 5 tasks numbered consecutively with a step size of 3 (task-ids 3,6,9,12,15)
	\begin{verbatim}
		sbatch --array=3-15:3 array.sh
	\end{verbatim}

	\item Submit a job with 50000 elements, where \%a maps to the task-id between 1 and 50K. 
	\begin{verbatim}
		sbatch --array=1-50000 -N1 -i my_in_%a -o my_out_%a array.sh
	\end{verbatim}
\end{itemize}

\noindent \textbf{Output files for Array Jobs:}\\
The default output and error-files are \texttt{slurm-job\_id\_task\_id.out}.
%
This means that Speed creates an output and an error-file for each task 
generated by the array-job, as well as one for the super-ordinate array-job. 
To alter this behavior use the \option{-o} and \option{-e} options of \tool{sbatch}.\\

For more details about Array Job options, please review the manual pages for 
\tool{sbatch} by executing the following at the command line on \tool{speed-submit}
\texttt{man sbatch}.
 
% -------------- 2.7 Requesting Multiple Cores ----------------
% -------------------------------------------------------------
\subsection{Requesting Multiple Cores (i.e., Multithreading Jobs)}
\label{sect:multicore-jobs}

For jobs that can take advantage of multiple machine cores, you can 
request up to 32 cores (per job) in your script using the following options:

\begin{verbatim}
	#SBATCH -n <#cores for processes>
	#SBATCH -n 1
	#SBATCH -c <#cores for threads of a single process>
\end{verbatim}

\noindent Both \tool{sbatch} and \tool{salloc} support \option{-n} on the command line,
and it should always be used either in the script or on the command line as the
default $n=1$.\\

\noindent \textbf{Important Considerations}:
\begin{itemize}
	\item Do not request more cores than you think will be useful, 
	as larger-core jobs are more difficult to schedule.

	\item If you are running a program that scales out to the maximum single-machine 
	core count available, please request 32 cores to avoid node 
	oversubscription (i.e., overloading the CPUs).
\end{itemize}

\noindent \textbf{Note:} \option{--ntasks} or \option{--ntasks-per-node}
(\option{-n}) refers to processes (usually the ones run with \tool{srun}).
\option{--cpus-per-task} (\option{-c}) corresponds to threads per process.\\

\noindent Some programs consider them equivalent, while others do not. For example, 
Fluent uses \option{--ntasks-per-node=8} and \option{--cpus-per-task=1},
whereas others may set \option{--cpus-per-task=8} and \option{--ntasks-per-node=1}.
If one of these is not 1, some applications need to be configured to use \texttt{n * c} total cores.\\

\noindent Core count associated with a job appears under,
``AllocCPUS'', in the, \texttt{sacct -j <job-id>}, output.

\small
\begin{verbatim}
	[serguei@speed-submit src] % squeue -l
	Thu Oct 19 20:32:32 2023
	JOBID PARTITION     NAME     USER    STATE       TIME TIME_LIMI  NODES NODELIST(REASON)
	2652        ps interact   a_user  RUNNING   9:35:18 1-00:00:00      1 speed-07
	[serguei@speed-submit src] % sacct -j 2652
	JobID           JobName  Partition    Account  AllocCPUS      State ExitCode
	------------ ---------- ---------- ---------- ---------- ---------- --------
	2652         interacti+         ps     speed1         20    RUNNING      0:0
	2652.intera+ interacti+                speed1         20    RUNNING      0:0
	2652.extern      extern                speed1         20    RUNNING      0:0
	2652.0       gydra_pmi+                speed1         20  COMPLETED      0:0
	2652.1       gydra_pmi+                speed1         20  COMPLETED      0:0
	2652.2       gydra_pmi+                speed1         20     FAILED      7:0
	2652.3       gydra_pmi+                speed1         20     FAILED      7:0
	2652.4       gydra_pmi+                speed1         20  COMPLETED      0:0
	2652.5       gydra_pmi+                speed1         20  COMPLETED      0:0
	2652.6       gydra_pmi+                speed1         20  COMPLETED      0:0
	2652.7       gydra_pmi+                speed1         20  COMPLETED      0:0
\end{verbatim}
\normalsize

% -------------- 2.8 Interactive Jobs -------------------------
% -------------------------------------------------------------
\subsection{Interactive Jobs}
\label{sect:interactive-jobs}

Interactive job sessions allow you to interact with the system in real-time. 
These sessions are particularly useful for tasks such as testing, debugging, optimizing code, 
setting up environments, and other preparatory work before submitting batch jobs.

%  2.8.1 Command Line
% -------------------
\subsubsection{Command Line}
\label{sect:command-line}

To request an interactive job session, use the \texttt{salloc} command with appropriate options.
This is similar to submitting a batch job but allows you to run shell commands interactively 
within the allocated resources. For example:
\begin{verbatim}
	salloc -J interactive-test --mem=1G -p ps -n 8
\end{verbatim}

Within the allocated \tool{salloc} session, you can run shell commands as usual. 
It is recommended to use \tool{srun} for compute-intensive steps within \tool{salloc}. 
If you need a quick, short job just to compile something on a GPU node, 
you can use an interactive srun directly. For example, a 1-hour allocation:\\

\noindent \textbf{For tcsh}:
\begin{verbatim}
	srun --pty -n 8 -p pg --gpus=1 --mem=1G -t 60 /encs/bin/tcsh
\end{verbatim}

\noindent \textbf{For bash}:
\begin{verbatim}
	srun --pty -n 8 -p pg --gpus=1 --mem=1G -t 60 /encs/bin/bash
\end{verbatim}

%  2.8.2 Graphical Applications
% -------------------
\subsubsection{Graphical Applications}
\label{sect:graphical-applications}

To run graphical UI applications (e.g., MALTLAB, Abaqus CME, IDEs like PyCharm, VSCode, Eclipse, etc.) on Speed, 
you need to enable X11 forwarding from your client machine Speed then to the compute node.
To do so, follow these steps:

\begin{enumerate}
	\item Run an X server on your client machine:
	\begin{itemize}
		\item \textbf{Windows:} Use MobaXterm with X turned on, or Xming + PuTTY with X11 forwarding, or XOrg under Cygwin
		\item \textbf{macOS:} Use XQuarz -- use its \tool{xterm} and \texttt{ssh -X}
		\item \textbf{Linux:} Use \texttt{ssh -X speed.encs.concordia.ca}
	\end{itemize}
	For more details, see \href{https://www.concordia.ca/ginacody/aits/support/faq/xserver.html}{How do I remotely launch X(Graphical) applications?}

	\item Verify that X11 forwarding is enabled by printing the \api{DISPLAY} variable:
	\begin{verbatim}
		echo $DISPLAY
	\end{verbatim}

	\item Start an interactive session with X11 forwarding enabled (Use the \option{--x11} with \tool{salloc} or \tool{srun}), for example:
	\begin{verbatim}
		salloc -p ps --x11=first --mem=4G -t 0-06:00
	\end{verbatim}

	\item Once landed on a compute node, verify \api{DISPLAY} again.
	
	\item Set the \api{XDG\_RUNTIME\_DIR} variable to a directory in your \tool{speed-scratch} space:
	\begin{verbatim}
		mkdir -p /speed-scratch/$USER/run-dir
		setenv XDG_RUNTIME_DIR /speed-scratch/$USER/run-dir
	\end{verbatim}
	
	\item Launch your graphical application:
	\begin{verbatim}
		module load matlab/R2023a/default 
		matlab
	\end{verbatim}
\end{enumerate}

\noindent
\textbf{Note:} with X11 forwarding the graphical rendering is happening on
your client machine! That is you are not using GPUs on Speed to render
graphics, instead all graphical information is forwarded from Speed to
your desktop or laptop over X11, which in turn renders it using its
own graphics card. Thus, for GPU rendering jobs either keep them
non-interactive or use VirtualGL.\\

\noindent
Here's an example of starting PyCharm (see \xf{fig:pycharm}). 
\textbf{Note:} If using VSCode, it's currently only supported with the \tool{--no-sandbox} option.\\

\noindent \textbf{TCSH version:}
\small
\begin{verbatim}
ssh -X speed (XQuartz xterm, PuTTY or MobaXterm have X11 forwarding too)
[speed-submit] [/home/c/carlos] > echo $DISPLAY
localhost:14.0
[speed-submit] [/home/c/carlos] > cd /speed-scratch/$USER
[speed-submit] [/speed-scratch/carlos] > echo $DISPLAY
localhost:13.0
[speed-submit] [/speed-scratch/carlos] > salloc -pps --x11=first --mem=4Gb -t 0-06:00
[speed-07] [/speed-scratch/carlos] > echo $DISPLAY
localhost:42.0
[speed-07] [/speed-scratch/carlos] > hostname
speed-07.encs.concordia.ca
[speed-07] [/speed-scratch/carlos] > setenv XDG_RUNTIME_DIR /speed-scratch/$USER/run-dir
[speed-07] [/speed-scratch/carlos] > /speed-scratch/nag-public/bin/pycharm.sh
\end{verbatim}
\normalsize
\noindent \textbf{BASH version:}
\small
\begin{verbatim}
bash-3.2$ ssh -X speed (XQuartz xterm, PuTTY or MobaXterm have X11 forwarding too)
serguei@speed's password: 
[serguei@speed-submit ~] % echo $DISPLAY
localhost:14.0
[serguei@speed-submit ~] % salloc -p ps --x11=first --mem=4Gb -t 0-06:00 
bash-4.4$ echo $DISPLAY
localhost:77.0
bash-4.4$ hostname
speed-01.encs.concordia.ca
bash-4.4$ export XDG_RUNTIME_DIR=/speed-scratch/$USER/run-dir
bash-4.4$ /speed-scratch/nag-public/bin/pycharm.sh
\end{verbatim}
\normalsize

\begin{figure}[htpb]
	\includegraphics[width=\columnwidth]{images/pycharm}
	\caption{Launching PyCharm on a Speed Node}
	\label{fig:pycharm}
\end{figure}

% -----------------------------------------------------------------------------
\subsubsection{Jupyter Notebooks}
\label{sect:jupyter}

%  2.8.3 Jupyter Notebooks in Singularity
% -------------------
%\subsubsection{Jupyter Notebook in Singularity}
\paragraph{Jupyter Notebook in Singularity}
\label{sect:jupyter-singularity}

To run Jupyter Notebooks using Singularity (more on Singularity see \xs{sect:singularity-containers}), follow these steps:

\begin{enumerate}
  % X11 is not really needed for Jupyter since we tunnel and use a browser
	%\item Connect to Speed with X11 forwarding enabled:
	\item Connect to Speed, e.g. interactively, using \tool{salloc}
	%\item Use the \option{--x11} with \tool{salloc} or \tool{srun} as described in the above example
	\item Load Singularity module
		\verb+module load singularity/3.10.4/default+

	\item
	Execute this Singularity command on a single line or save it in a shell script
	\href{https://github.com/NAG-DevOps/speed-hpc/blob/master/src/jupyter.sh}{from our GitHub}
	where you could easily invoke it.
	
	\scriptsize
	\begin{verbatim}
srun singularity exec -B $PWD\:/speed-pwd,/speed-scratch/$USER\:/my-speed-scratch,/nettemp \
--env SHELL=/bin/bash --nv /speed-scratch/nag-public/openiss-cuda-conda-jupyter.sif \
/bin/bash -c '/opt/conda/bin/jupyter notebook --no-browser --notebook-dir=/speed-pwd \
--ip="*" --port=8888 --allow-root'
	\end{verbatim}
	\normalsize

	\item
	In a new terminal window, create an \tool{ssh} tunnel between your computer and the node (\texttt{speed-XX}) where Jupyter is
	running (using \texttt{speed-submit} as a ``jump server'', see, e.g., in PuTTY, in \xf{fig:putty1} and \xf{fig:putty2})
	\small
	\begin{verbatim}
		ssh -L 8888:speed-XX:8888 <ENCS-username>@speed-submit.encs.concordia.ca
	\end{verbatim}
	\normalsize
	Don't close the tunnel after establishing.

	\item
	Open a browser, and copy your Jupyter's token (it's printed to you in the terminal)
	and paste it in the browser's URL field.
	In our case, the URL is:
	\small
	\begin{verbatim}
		http://localhost:8888/?token=5a52e6c0c7dfc111008a803e5303371ed0462d3d547ac3fb
	\end{verbatim}
	\normalsize

	\item Access the Jupyter Notebook interface in your browser.
\end{enumerate}

\begin{figure}[htbp]
	\centering
	\fbox{\includegraphics{images/putty1}}
	\caption{SSH tunnel configuration 1}
	\label{fig:putty1}
\end{figure}

\begin{figure}[htbp]
	\centering
	\fbox{\includegraphics{images/putty2}}
	\caption{SSH tunnel configuration 2}
	\label{fig:putty2}
\end{figure}

\begin{figure}[htbp]
	\centering
	\fbox{\includegraphics[width=1.00\textwidth]{images/jupyter.png}}
	\caption{Jupyter running on a Speed node}
	\label{fig:jupyter}
\end{figure}

\noindent
Another sample is the OpenISS-derived containers with Conda and Jupyter,
see \xs{sect:openiss-examples} for details.

%  2.8.4 JupyterLab in Conda and Pytorch
% -------------------
%\subsubsection{JupyterLab in Conda and Pytorch}
\paragraph{JupyterLab in Conda and Pytorch}
\label{sect:jupyterlabs}

For setting up Jupyter Labs with Conda and Pytorch, follow these steps:

\begin{itemize}
	\item Environment preparation: (only once, takes some time to run to install all required dependencies)
	\begin{enumerate}
		\item Navigate to your speed-scratch directory:
		\begin{verbatim}
			cd /speed-scratch/\$USER
		\end{verbatim}

		\item Create a Jupyter (name of your choice) directory and \tool{cd} into it:
		\begin{verbatim}
			mkdir -p Jupyter
			cd Jupyter
		\end{verbatim}

		\item Start an interactive session:
		\begin{verbatim}
			salloc --mem=50G --gpus=1 -ppg (or -ppt)
		\end{verbatim}
		
		\item
		Set \tool{conda} environment variables, and install \tool{jupyterlab} and \tool{pytorch},
		as shown in \xf{fig:firsttime.sh} from our GitHub.
		
		\begin{figure}[htpb]
			\tiny
			\lstinputlisting[language=csh,frame=single,basicstyle=\ttfamily]{../src/jupyterlabs/firsttime.sh}
			\normalsize
			\caption{Source code for \texttt{firsttime.sh}}
			\label{fig:firsttime.sh}
		\end{figure}
	\end{enumerate}

	\item
	Execution of Jupyter Labs from \textbf{speed-submit} (repeat this every time you want to run Jupyter Labs):
	\begin{enumerate}
		\item Start an interactive session:
		\begin{verbatim}
			salloc --mem=50G --gpus=1 -p pg (or -p pt)
		\end{verbatim}

		\item
		Activate your \tool{conda} environment and run Jupyter Labs, as shown in
		\xf{fig:run.sh} (also available on our GitHub).
		
		\begin{figure}[htpb]
			\scriptsize
			\lstinputlisting[language=csh,frame=single,basicstyle=\ttfamily]{../src/jupyterlabs/run.sh}
			\normalsize
			\caption{Source code for \texttt{run.sh}}
			\label{fig:run.sh}
		\end{figure}

		\item
		Verify which port the system has assigned to your Jupyter Lab instance by examining the URL
		\texttt{http://localhost:XXXX/lab?token=} in your terminal as described
		previously.

		\item
		In a new terminal window, create an \tool{ssh} tunnel similar to Jupyter 
    in Singularity, see \xs{sect:jupyter-singularity}.

		\item
		Open a browser and copy your Jupyter's token and paste it in the browser's URL field
	\end{enumerate}
\end{itemize}

%  2.8.5 JupyterLab + Pytorch in Python venv
% -------------------
%\subsubsection{JupyterLab + Pytorch in Python venv}
\paragraph{JupyterLab + Pytorch in Python venv}
\label{sect:jupyterlabs-venv}

This is an example of Jupyter Labs running in a Python Virtual environment (\texttt{venv}), with Pytorch on Speed.\\

\noindent
\textbf{Note:} Use of Python virtual environments is preferred over Conda at Alliance Canada clusters.
If you prefer to make jobs that are more compatible between Speed and Alliance clusters, use Python
\texttt{venv}s. See \url{https://docs.alliancecan.ca/wiki/Anaconda/en}
and \url{https://docs.alliancecan.ca/wiki/JupyterNotebook}.

\begin{itemize}
\item
Environment preparation: for the FIRST time only:
\begin{enumerate}
\item
Go to your speed-scratch directory: \texttt{cd /speed-scratch/\$USER}
\item
Open an interactive session: \texttt{salloc --mem=50G --gpus=1 --constraint=el9}
\item
Create a Python \texttt{venv} and install \tool{jupyterlab}+\tool{pytorch}
\scriptsize
\begin{verbatim}
module load python/3.11.5/default
setenv TMPDIR /speed-scratch/$USER/tmp
setenv TMP /speed-scratch/$USER/tmp
setenv PIP_CACHE_DIR /speed-scratch/$USER/tmp/cache
python -m venv /speed-scratch/$USER/tmp/jupyter-venv
source /speed-scratch/$USER/tmp/jupyter-venv/bin/activate.csh
pip install jupyterlab
pip3 install torch torchvision torchaudio --index-url https://download.pytorch.org/whl/cu118
exit
\end{verbatim}
\normalsize
\end{enumerate}
\item
Running Jupyter Labs, from \textbf{speed-submit}:
\begin{enumerate}
\item
Open an interactive session: \texttt{salloc --mem=50G --gpus=1 --constraint=el9} 
\scriptsize
\begin{verbatim}
cd /speed-scratch/$USER
module load python/3.11.5/default
setenv PIP_CACHE_DIR /speed-scratch/$USER/tmp/cache
source /speed-scratch/$USER/tmp/jupyter-venv/bin/activate.csh
jupyter lab --no-browser --notebook-dir=$PWD --ip="0.0.0.0" --port=8888 --port-retries=50
\end{verbatim}
\normalsize
\item
Verify which port the system has assigned to Jupyter:\\
\texttt{http://localhost:XXXX/lab?token=}
\item
SSH Tunnel creation: similar to Jupyter in Singularity, see \xs{sect:jupyter-singularity}
\item
Open a browser and type: \texttt {localhost:XXXX} (using the port assigned)
\end{enumerate}
\end{itemize}


%  2.8.6 Visual Studio Code
% -------------------
\subsubsection{Visual Studio Code}
\label{sect:vscode}

This is an example of running VScode, it's similar to Jupyter notebooks, but 
it doesn't use containers. \textbf{Note:} this a Web-based version; there exists the local
(workstation)~--~remote (speed-node) client-server version too, but it is for advanced users
and is out of scope here (so no support, use it at your own risk).
 
\begin{itemize}

\item
Environment preparation: for the FIRST time:
\begin{enumerate}
\item
Go to your speed-scratch directory: \texttt{cd /speed-scratch/\$USER}
\item
Create a vscode directory: \texttt{mkdir vscode}
\item
Go to vscode: \texttt{cd vscode}
\item
Create home and projects: \texttt{mkdir \{home,projects\}}
\item
Create this directory: \texttt{mkdir -p /speed-scratch/\$USER/run-user}
\end{enumerate}

\item
Running VScode
\begin{enumerate}
\item 
Go to your vscode directory: \texttt{cd /speed-scratch/\$USER/vscode}
\item
Open interactive session: \texttt{salloc --mem=10Gb --constraint=el9}
\item
Set environment variable:\\\texttt{setenv XDG\_RUNTIME\_DIR /speed-scratch/\$USER/run-user}
\item 
Run VScode, change the port if needed.
\scriptsize
\begin{verbatim}
/speed-scratch/nag-public/code-server-4.22.1/bin/code-server --user-data-dir=$PWD\/projects \
--config=$PWD\/home/.config/code-server/config.yaml --bind-addr="0.0.0.0:8080" $PWD\/projects
\end{verbatim}
\normalsize
\item
SSH Tunnel creation: similar to Jupyter, see \xs{sect:jupyter-singularity}
\item
Open a browser and type: \texttt{localhost:8080}
\item
If the browser asks for a password, consult:
\begin{verbatim}
cat /speed-scratch/$USER/vscode/home/.config/code-server/config.yaml
\end{verbatim}

\end{enumerate}
\end{itemize}

\begin{figure}[htbp]
	\centering
	\fbox{\includegraphics[width=1.00\textwidth]{images/vscode.png}}
	\caption{VScode running on a Speed node}
	\label{fig:vscode}
\end{figure}

% ------------------------------------------------------------------------------
\subsection{SSH Keys For MPI}

Some programs effect their parallel processing via MPI (which is a 
communication protocol). An example of such software is Fluent. MPI needs to 
have `passwordless login' set up, which means SSH keys. In your NFS-mounted 
home directory:

\begin{itemize}
\item
\texttt{cd .ssh}
\item
\texttt{ssh-keygen -t ed25519} (default location; blank passphrase) 
\item
\texttt{cat id\_ed25519.pub >> authorized\_keys} (if the \texttt{\href{https://www.ssh.com/academy/ssh/authorized-keys-file}{authorized\_keys}}
file already exists) \emph{OR} \texttt{cat id\_ed25519.pub > authorized\_keys} (if does not) 
\item
Set file permissions of \texttt{authorized\_keys} to 600; of your NFS-mounted home
to 700 (note that you likely will not have to do anything here, as most people
will have those permissions by default). 
\end{itemize}

% ------------------------------------------------------------------------------
\subsection{Creating Virtual Environments}
\label{sect:environments}

The following documentation is specific to the \textbf{Speed} HPC Facility at the
Gina Cody School of Engineering and Computer Science.

% ------------------------------------------------------------------------------
\subsubsection{Anaconda}

To create an anaconda environment in your speed-scratch directory, use the \texttt{\-\-prefix} 
option when executing \texttt{conda create}. For example, to create an anaconda environment for 
\texttt{a\_user}, execute the following at the command line:

\begin{verbatim}
conda create --prefix /speed-scratch/a_user/myconda
\end{verbatim}

\vspace{10pt}
\noindent
\textbf{Note:} Without the \texttt{\-\-prefix} option, the \texttt{conda create} command creates the 
environment in \texttt{a\_user}'s home directory by default.
\vspace{10pt}

% ------------------------------------------------------------------------------
\paragraph{List Environments.}

To view your conda environments, type: \texttt{conda info --envs}

\begin{verbatim}
# conda environments:
#
base                  *  /encs/pkg/anaconda3-2019.07/root
                         /speed-scratch/a_user/myconda
\end{verbatim}      

% ------------------------------------------------------------------------------
\paragraph{Activate an Environment.}

Activate the environment \texttt{\/speed\-scratch\/a\_user\/myconda} as follows
\begin{verbatim}
conda activate /speed-scratch/a_user/myconda
\end{verbatim}
After activating your environment, add \tool{pip} to your environment by using 
\begin{verbatim}
conda install pip
\end{verbatim}
This will install \tool{pip} and \tool{pip}'s dependencies, including python, 
into the environment.

\vspace{10pt}
\noindent
\textbf{Important Note:} \tool{pip} (and \tool{pip3}) are used to install modules
 from the python distribution while \texttt{conda install} installs modules from 
 anaconda's repository.
\vspace{10pt}

% ------------------------------------------------------------------------------
% TMP scheduler-specific section
% ------------------------------------------------------------------------------
\subsection{Example Job Script: Fluent}

\begin{figure}[htpb]
    \lstinputlisting[language=csh,frame=single,basicstyle=\footnotesize\ttfamily]{fluent.sh}
    \caption{Source code for \file{fluent.sh}}
	\label{fig:fluent.sh}
\end{figure}

The job script in \xf{fig:fluent.sh} runs Fluent in parallel over 32 cores. 
Of note, we have requested e-mail notifications (\texttt{-m}), are defining the 
parallel environment for, \tool{fluent}, with, \texttt{-sgepe smp} (\textbf{very 
important}), and are setting \api{\$TMPDIR} as the in-job location for the
``moment'' \file{rfile.out} file (in-job, because the last line of the script 
copies everything from \api{\$TMPDIR} to a directory in the user's NFS-mounted home). 
Job progress can be monitored by examining the standard-out file (e.g.,
\file{flu10000.o249}), and/or by examining the ``moment'' file in 
\texttt{/disk/nobackup/<yourjob>} (hint: it starts with your job-ID) on the node running
the job. \textbf{Caveat:} take care with journal-file file paths.

% ------------------------------------------------------------------------------
\subsection{Example Job: efficientdet}

The following steps describing how to create an efficientdet environment on
\emph{Speed}, were submitted by a member of Dr. Amer's research group.

\begin{itemize}
    \item 
    Enter your ENCS user account's speed-scratch directory 
    \verb!cd /speed-scratch/<encs_username>!
    \item
    load python \verb!module load python/3.8.3!
    create virtual environment \verb!python3 -m venv <env_name>!
    activate virtual environment \verb!source <env_name>/bin/activate.csh!
    install DL packages for Efficientdet
\end{itemize}
\begin{verbatim}
pip install tensorflow==2.7.0
pip install lxml>=4.6.1
pip install absl-py>=0.10.0
pip install matplotlib>=3.0.3
pip install numpy>=1.19.4
pip install Pillow>=6.0.0
pip install PyYAML>=5.1
pip install six>=1.15.0
pip install tensorflow-addons>=0.12
pip install tensorflow-hub>=0.11
pip install neural-structured-learning>=1.3.1
pip install tensorflow-model-optimization>=0.5
pip install Cython>=0.29.13
pip install git+https://github.com/cocodataset/cocoapi.git#subdirectory=PythonAPI
\end{verbatim}

% ------------------------------------------------------------------------------
\subsection{Java Jobs}

Jobs that call \tool{java} have a memory overhead, which needs to be taken 
into account when assigning a value to \api{h\_vmem}. Even the most basic 
\tool{java} call, \texttt{java -Xmx1G -version}, will need to have,
\texttt{-l h\_vmem=5G}, with the 4-GB difference representing the memory overhead. 
Note that this memory overhead grows proportionally with the value of
\texttt{-Xmx}. To give you an idea, when \texttt{-Xmx} has a value of 100G,
\api{h\_vmem} has to be at least 106G; for 200G, at least 211G; for 300G, at least 314G.

% TODO: add a MARF Java job

% ------------------------------------------------------------------------------
\subsection{Scheduling On The GPU Nodes}

The primary cluster has two GPU nodes, each with six Tesla (CUDA-compatible) P6
cards: each card has 2048 cores and 16GB of RAM. Though note that the P6
is mainly a single-precision card, so unless you need the GPU double
precision, double-precision calculations will be faster on a CPU node.

Job scripts for the GPU queue differ in that they do not need these
statements:

\begin{verbatim}
#$ -pe smp <threadcount>
#$ -l h_vmem=<memory>G
\end{verbatim}

But do need this statement, which attaches either a single GPU, or, two
GPUs, to the job:

\begin{verbatim}
#$ -l gpu=[1|2]
\end{verbatim}

Single-GPU jobs are granted 5~CPU cores and 80GB of system memory, and
dual-GPU jobs are granted 10~CPU cores and 160GB of system memory. A
total of \emph{four} GPUs can be actively attached to any one user at any given
time.

Once that your job script is ready, you can submit it to the GPU queue
with:

\begin{verbatim}
qsub -q g.q ./<myscript>.sh
\end{verbatim}

And you can query \tool{nvidia-smi} on the node that is running your job with:

\begin{verbatim}
ssh <username>@speed[-05|-17] nvidia-smi
\end{verbatim}

Status of the GPU queue can be queried with:

\begin{verbatim}
qstat -f -u "*" -q g.q
\end{verbatim}

\textbf{Very important note} regarding TensorFlow and PyTorch: 
if you are planning to run TensorFlow and/or PyTorch multi-GPU jobs, 
do not use the \api{tf.distribute} and/or\\
\api{torch.nn.DataParallel} 
functions, as they will crash the compute node (100\% certainty). 
This appears to be the current hardware's architecture's defect.
%
The workaround is to either
% TODO: Need to link to that example
manually effect GPU parallelisation (TensorFlow has an example on how to
do this), or to run on a single GPU.

\vspace{10pt}
\noindent
\textbf{Important}
\vspace{10pt}

Users without permission to use the GPU nodes can submit jobs to the \texttt{g.q}
queue but those jobs will hang and never run.

There are two GPUs in both \texttt{speed-05} and \texttt{speed-17}, and one 
in \texttt{speed-19}. Their availability is seen with, \texttt{qstat -F g}
(note the capital): 

\small
\begin{verbatim}
queuename                      qtype resv/used/tot. load_avg arch          states
---------------------------------------------------------------------------------
...
---------------------------------------------------------------------------------
g.q@speed-05.encs.concordia.ca BIP   0/0/32         0.04     lx-amd64
        hc:gpu=6
---------------------------------------------------------------------------------
g.q@speed-17.encs.concordia.ca BIP   0/0/32         0.01     lx-amd64
        hc:gpu=6
---------------------------------------------------------------------------------
...
---------------------------------------------------------------------------------
s.q@speed-19.encs.concordia.ca BIP   0/32/32        32.37    lx-amd64
        hc:gpu=1
---------------------------------------------------------------------------------
etc. 
\end{verbatim}
\normalsize

This status demonstrates that all five are available (i.e., have not been 
requested as resources). To specifically request a GPU node, add,
\texttt{-l g=[\#GPUs]}, to your \tool{qsub} (statement/script) or
\tool{qlogin} (statement) request. For example,
\texttt{qsub -l h\_vmem=1G -l g=1 ./count.sh}. You 
will see that this job has been assigned to one of the GPU nodes:

\small
\begin{verbatim}
queuename                      qtype resv/used/tot. load_avg arch          states
--------------------------------------------------------------------------------- 
g.q@speed-05.encs.concordia.ca BIP 0/0/32 0.01 lx-amd64  hc:gpu=6 
--------------------------------------------------------------------------------- 
g.q@speed-17.encs.concordia.ca BIP 0/0/32 0.01 lx-amd64  hc:gpu=6 
--------------------------------------------------------------------------------- 
s.q@speed-19.encs.concordia.ca BIP 0/1/32 0.04 lx-amd64  hc:gpu=0 (haff=1.000000) 
       538 100.00000 count.sh   sbunnell     r     03/07/2019 02:39:39     1
---------------------------------------------------------------------------------
etc. 
\end{verbatim}
\normalsize

And that there are no more GPUs available on that node (\texttt{hc:gpu=0}). Note
that no more than two GPUs can be requested for any one job. 

% ------------------------------------------------------------------------------
\subsubsection{CUDA}

When calling \tool{CUDA} within job scripts, it is important to create a link to
the desired \tool{CUDA} libraries and set the runtime link path to the same libraries. 
For example, to use the \texttt{cuda-11.5} libraries, specify the following in 
your Makefile.

\begin{verbatim}
-L/encs/pkg/cuda-11.5/root/lib64 -Wl,-rpath,/encs/pkg/cuda-11.5/root/lib64
\end{verbatim}

In your job script, specify the version of \texttt{gcc} to use prior to calling 
cuda. For example: 
   \texttt{module load gcc/8.4}
or
   \texttt{module load gcc/9.3}

% ------------------------------------------------------------------------------
\subsubsection{Special Notes for sending CUDA jobs to the GPU Queue}

It is not possible to create a \texttt{qlogin} session on to a node in the 
\textbf{GPU Queue} (\texttt{g.q}). As direct logins to these nodes is not 
available, jobs must be submitted to the \textbf{GPU Queue} in order to compile 
and link.

We have several versions of CUDA installed in:
\begin{verbatim}
/encs/pkg/cuda-11.5/root/
/encs/pkg/cuda-10.2/root/
/encs/pkg/cuda-9.2/root
\end{verbatim}

For CUDA to compile properly for the GPU queue, edit your Makefile 
replacing \option{\/usr\/local\/cuda} with one of the above.


% ------------------------------------------------------------------------------
\section{Conclusion}
\label{sect:conclusion}

The cluster is, ``first come, first served'', until it fills, and then job
position in the queue is based upon past usage. The scheduler does attempt
to fill gaps, though, so sometimes a single-core job of lower priority
will schedule before a multi-core job of higher priority, for example.

% ------------------------------------------------------------------------------
\subsection{Important Limitations}
\label{sect:limitations}

\begin{itemize}
\item
New users are restricted to a total of 32 cores: write to \url{rt-ex-hpc@encs.concordia.ca}
if you need more temporarily (256 is the maximum possible, or, 8 jobs of 32 cores each).

\item
Job sessions are a maximum of one week in length (only 24 hours, though,
for interactive jobs).

\item
Scripts can live in your NFS-provided home, but any substantial data need
to be in your cluster-specific directory
(located at \verb+/speed-scratch/<ENCSusername>/+).

NFS is great for acute activity, but is not ideal for chronic activity.
Any data that a job will 
read more than once should be copied at the start to the scratch disk of a 
compute node using \api{\$TMPDIR} (and, perhaps, \api{\$SLURM\_SUBMIT\_DIR}), 
any intermediary job data should be produced in \api{\$TMPDIR}, and once a 
job is near to finishing, those data should be copied to your NFS-mounted 
home (or other NFS-mounted space) from \api{\$TMPDIR} (to, perhaps,
\api{\$SLURM\_SUBMIT\_DIR}). In other words, IO-intensive operations should be effected 
locally whenever possible, saving network activity for the start and end of 
jobs. 

\item
Your current resource allocation is based upon past usage, which is an 
amalgamation of approximately one week's worth of past wallclock (i.e., time 
spent on the node(s)) and compute activity (on the node(s)).

\item
Jobs should NEVER be run outside of the province of the scheduler.
Repeat offenders risk loss of cluster access. 

\end{itemize}

% ------------------------------------------------------------------------------
% TMP scheduler-specific section
% ------------------------------------------------------------------------------
\subsection{Tips/Tricks}
\label{sect:tips}

\begin{itemize}
\item
Files/scripts must have Linux line breaks in them (not Windows ones).
\item
Use \tool{rsync}, not \tool{scp}, when moving data around. 
\item
If you are going to move many many files between NFS-mounted storage and the 
cluster, \tool{tar} everything up first. 
\item
If you intend to use a different shell (e.g., \tool{bash}~\cite{aosa-book-vol1-bash}),
you will need to source a different scheduler file, and will need to 
change the shell declaration in your script(s).
\item
The load displayed in \tool{qstat} by default is \api{np\_load}, which is
load/\#cores. That means that a load of, ``1'', which represents a fully active 
core, is displayed as $0.03$ on the node in question, as there are 32 cores 
on a node. To display load ``as is'' (such that a node with a fully active 
core displays a load of approximately $1.00$), add the following to your
\file{.tcshrc} file: \texttt{setenv SGE\_LOAD\_AVG load\_avg}

\item
Try to request resources that closely match what your job will use: 
requesting many more cores or much more memory than will be needed makes a 
job more difficult to schedule when resources are scarce.

\item
E-mail, \texttt{rt-ex-hpc AT encs.concordia.ca}, with any concerns/questions.
\end{itemize}


% ------------------------------------------------------------------------------
\subsection{Use Cases}
\label{sect:cases}

\begin{itemize}
	\item 
HPC Committee's initial batch about 6 students (end of 2019):
\begin{itemize}
	\item 
10000 iterations job in Fluent finished in $<26$ hours vs. 46 hours in Calcul Quebec
\end{itemize}
	\item 
NAG's MAC spoofer analyzer~\cite{mac-spoofer-analyzer-intro-c3s2e2014,mac-spoofer-analyzer-detail-fps2014},
such as \url{https://github.com/smokhov/atsm/tree/master/examples/flucid}
\begin{itemize}
	\item 
compilation of forensic computing reasoning cases about false or true positives of hardware address spoofing in the labs
\end{itemize}
	\item 
S4 LAB/GIPSY R\&D Group's:
\begin{itemize}
	\item 
MARFCAT and MARFPCAT (OSS signal processing and machine learning tools for 
vulnerable and weak code analysis and network packet capture
analysis)~\cite{marfcat-nlp-ai2014,marfcat-sate2010-nist,fingerprinting-mal-traffic}
	\item 
Web service data conversion and analysis
	\item 
{\flucid} encoders (translation of large log data into {\flucid}~\cite{mokhov-phd-thesis-2013} for forensic analysis)
	\item 
Genomic alignment exercises
\end{itemize}
\item
\bibentry{oi-containers-poster-siggraph2023}
\item
\bibentry{Gopal2024Sep}
\item
\bibentry{Gopal2023Mob}
\item
\bibentry{niksirat2020}

\item
The work ``\bibentry{lai-haotao-mcthesis19}'' using TensorFlow and Keras on OpenISS
adjusted to run on Speed based on the repositories:
\begin{itemize}
	\item 
\bibentry{openiss-reid-tfk} and
	\item
\bibentry{openiss-yolov3}
\end{itemize}
and theirs forks by the team.

\end{itemize}

% ------------------------------------------------------------------------------
\appendix

% ------------------------------------------------------------------------------
\section{History}

% ------------------------------------------------------------------------------
\subsection{Acknowledgments}
\label{sect:acks}

\begin{itemize}
	\item 
The first 6 (to 6.5) versions of this manual and early UGE job script samples,
Singularity testing and user support were produced/done by Dr.~Scott Bunnell
during his time at Concordia as a part of the NAG/HPC group. We thank
him for his contributions.
	\item 
The HTML version with devcontainer support was contributed by Anh H Nguyen.
	\item 
Tariq Daradkeh, PhD, was our IT Instructional Specialist August 2022 to September 2023;
working on the scheduler, scheduling research, end user support, and integration of
examples, such as YOLOv3 in \xs{sect:openiss-yolov3} other tasks. We have a continued
collaboration on HPC/scheduling research.
\end{itemize}

% ------------------------------------------------------------------------------
\subsection{Phase 4}

Phase 4 had 7 SuperMicro servers with 4x A100 80GB GPUs each added,
dubbed as ``SPEED2''.

% ------------------------------------------------------------------------------
\subsection{Phase 3}

Phase 3 had 4 vidpro nodes added from Dr.~Amer totalling 6x P6 and 6x V100
GPUs added.

% ------------------------------------------------------------------------------
\subsection{Phase 2}

Phase 2 saw 6x NVIDIA Tesla P6 added and 8x more compute nodes.
The P6s replaced 4x of FirePro S7150.

% ------------------------------------------------------------------------------
\subsection{Phase 1}

Phase 1 of Speed was of the following configuration:

\begin{itemize}
\item
Sixteen, 32-core nodes, each with 512~GB of memory and approximately 1~TB of volatile-scratch disk space. 
\item
Five AMD FirePro S7150 GPUs, with 8~GB of memory (compatible with the Direct X, OpenGL, OpenCL, and Vulkan APIs). 
\end{itemize}

% ------------------------------------------------------------------------------
% TMP scheduler-specific section
% ------------------------------------------------------------------------------
\section{Frequently Asked Questions}
\label{sect:faqs}

% ------------------------------------------------------------------------------
\subsection{Where do I learn about Linux?}

All Speed users are expected to have a basic understanding of Linux and its commonly used commands.

% ------------------------------------------------------------------------------
\subsubsection*{Software Carpentry}

Software Carpentry provides free resources to learn software, including a workshop on the Unix shell.
\url{https://software-carpentry.org/lessons/} 

% ------------------------------------------------------------------------------
\subsubsection*{Udemy}

There are a number of Udemy courses, including free ones, that will assist 
you in learning Linux. Active Concordia faculty, staff and students have 
access to Udemy courses such as \textbf{Linux Mastery: Master the Linux 
Command Line in 11.5 Hours} is a good starting point for beginners. Visit
\url{https://www.concordia.ca/it/services/udemy.html} to learn how Concordians 
may access Udemy.

% ------------------------------------------------------------------------------
\subsection{How to use the ``bash shell'' on Speed?}

This section describes how to use the ``bash shell'' on Speed. Review
\xs{sect:envsetup} to ensure that your bash environment is set up.

% ------------------------------------------------------------------------------
\subsubsection{How do I set bash as my login shell?}

In order to set your login shell to bash on Speed, your login shell on all GCS servers must be changed to bash.
To make this change, create a ticket with the Service Desk (or email help at concordia.ca) to request that bash become your default login shell for your ENCS user account on all GCS servers.

% ------------------------------------------------------------------------------
\subsubsection{How do I move into a bash shell on Speed?}

To move to the bash shell, type \textbf{bash} at the command prompt.
For example:
\begin{verbatim}
	[speed-submit] [/home/a/a_user] > bash
	bash-4.4$ echo $0
	bash
\end{verbatim}	

Note how the command prompt changed from \verb![speed-submit] [/home/a/a_user] >! to \verb!bash-4.4$! after entering the bash shell.

% ------------------------------------------------------------------------------
\subsubsection{How do I run scripts written in bash on Speed?}

To execute bash scripts on Speed:
\begin{enumerate}
	\item 
Ensure that the shebang of your bash job script is \verb!#!/encs/bin/bash!
	\item 
Use the qsub command to submit your job script to the scheduler.
\end{enumerate}

The Speed GitHub contains a sample \href{https://github.com/NAG-DevOps/speed-hpc/blob/master/src/bash.sh}{bash job script}.  

% ------------------------------------------------------------------------------
\subsection{How to resolve ``Disk quota exceeded'' errors?}

% ------------------------------------------------------------------------------
\subsubsection{Probable Cause}

The \texttt{``Disk quota exceeded''} Error occurs when your application has run out of disk space to write to. On Speed this error can be returned when:
\begin{enumerate}
	\item
The \texttt{/tmp} directory on the speed node your application is running on is full and cannot be written to.
	\item
Your NFS-provided home is full and cannot be written to.
\end{enumerate}

% ------------------------------------------------------------------------------
\subsubsection{Possible Solutions}

\begin{enumerate}
	\item
Use the \textbf{-cwd} job script option to set the directory that the job 
script is submitted from the \texttt{job working directory}. The
\texttt{job working directory} is the directory that the job will write output files in.
 	\item
The use local disk space is generally recommended for IO intensive operations. However, as the size of \texttt{/tmp} on speed nodes 
is \texttt{1GB} it can be necessary for scripts to store temporary data 
elsewhere. 
Review the documentation for each module called within your script to 
determine how to set working directories for that application. 
The basic steps for this solution are:
\begin{itemize}
	\item
	Review the documentation on how to set working directories for 
	each module called by the job script.
	\item
	Create a working directory in speed-scratch for output files. 
	For example, this command will create a subdirectory called \textbf{output}
	 in your \verb!speed-scratch! directory:
	 \begin{verbatim}
		mkdir -m 750 /speed-scratch/$USER/output
	 \end{verbatim}
	\item
	To create a subdirectory for recovery files:
	\begin{verbatim}
		mkdir -m 750 /speed-scratch/$USER/recovery
	\end{verbatim}
	\item
	Update the job script to write output to the subdirectories you created in your \verb!speed-scratch! directory, e.g., \verb!/speed-scratch/$USER/output!.
	\end{itemize}
\end{enumerate}
In the above example, \verb!$USER! is an environment variable containing your ENCS username.

% ------------------------------------------------------------------------------
\subsubsection{Example of setting working directories for \tool{COMSOL}}

\begin{itemize}
	\item 
	Create directories for recovery, temporary, and configuration files. 
	For example, to create these directories for your GCS ENCS user account:
	\begin{verbatim}
	mkdir -m 750 -p /speed-scratch/$USER/comsol/{recovery,tmp,config}
	\end{verbatim}
	\item
	Add the following command switches to the COMSOL command to use the 
	directories created above:
	\begin{verbatim} 
	-recoverydir /speed-scratch/$USER/comsol/recovery 
	-tmpdir /speed-scratch/$USER/comsol/tmp
	-configuration/speed-scratch/$USER/comsol/config
	\end{verbatim}
\end{itemize} 
In the above example, \verb!$USER! is an environment variable containing your ENCS username.

% ------------------------------------------------------------------------------
\subsubsection{Example of setting working directories for \tool{Python Modules}}

By default when adding a python module the \texttt{/tmp} directory is set as the temporary repository for files downloads. 
The size of the \texttt{/tmp} directory on \verb!speed-submit! is too small for pytorch.
To add a python module
\begin{itemize}
    \item 	
	Create your own tmp directory in your \verb!speed-scratch! directory
	\begin{verbatim} 
  mkdir /speed-scratch/$USER/tmp
	\end{verbatim}
	\item
  Use the tmp directory you created
	\begin{verbatim} 
  setenv TMPDIR /speed-scratch/$USER/tmp
	\end{verbatim}
    \item
	Attempt the installation of pytorch
\end{itemize}

In the above example, \verb!$USER! is an environment variable containing your ENCS username.

% ------------------------------------------------------------------------------
\subsection{How do I check my job's status?}

When a job with a job id of 1234 is running, the status of that job can be tracked using \verb!`qstat -j 1234`!.
Likewise, if the job is pending, the \verb!`qstat -j 1234`! command will report as to why the job is not scheduled or running.
Once the job has finished, or has been killed, the \textbf{qacct} command must be used to query the job's status, e.g., \verb!`qaact -j [jobid]`!. 

% ------------------------------------------------------------------------------
\subsection{Why is my job pending when nodes are empty?}

% ------------------------------------------------------------------------------
\subsubsection{Disabled nodes}

It is possible that a (or a number of) the Speed nodes are disabled. Nodes are disabled if they require maintenance. 
To verify if Speed nodes are disabled, request the current list of disabled nodes from qstat.

\begin{verbatim}
qstat -f -qs d
queuename                      qtype resv/used/tot. load_avg arch          states
---------------------------------------------------------------------------------
g.q@speed-05.encs.concordia.ca BIP   0/0/32         0.27     lx-amd64      d
---------------------------------------------------------------------------------
s.q@speed-07.encs.concordia.ca BIP   0/0/32         0.01     lx-amd64      d
---------------------------------------------------------------------------------
s.q@speed-10.encs.concordia.ca BIP   0/0/32         0.01     lx-amd64      d
---------------------------------------------------------------------------------
s.q@speed-16.encs.concordia.ca BIP   0/0/32         0.02     lx-amd64      d
---------------------------------------------------------------------------------
s.q@speed-19.encs.concordia.ca BIP   0/0/32         0.03     lx-amd64      d
---------------------------------------------------------------------------------
s.q@speed-24.encs.concordia.ca BIP   0/0/32         0.01     lx-amd64      d
---------------------------------------------------------------------------------
s.q@speed-36.encs.concordia.ca BIP   0/0/32         0.03     lx-amd64      d
\end{verbatim}

Note how the all of the Speed nodes in the above list have a state of \textbf{d}, or disabled.

Your job will run once the maintenance has been completed and the disabled nodes have been enabled.

% ------------------------------------------------------------------------------
\subsubsection{Error in job submit request.}

It is possible that your job is pending, because the job requested resources that are not available within Speed.
To verify why pending job with job id 1234 is not running, execute \verb!`qstat -j 1234`! 
and review the messages in the \textbf{scheduling info:} section.


% ------------------------------------------------------------------------------
\section{Sister Facilities}

Below is a list of resources and facilities similar to Speed at various capacities.
Depending on your research group and needs, they might be available to you. They
are not managed by HPC/NAG of AITS, so contact their respective representatives.

\begin{itemize}
\item
\texttt{computation.encs} CPU only 3-machine cluster running longer jobs
without a scheduler at the moment
\item
\texttt{apini.encs} cluster for teaching and MPI programming (see the corresponding
course in CSSE)
\item
Computer Science and Software Engineering (CSSE) Virya GPU Cluster. For CSSE 
members only. The cluster has 4 nodes with total of 32 NVIDIA GPUs (a mix of
V100s and A100s). To request access send email to \texttt{virya.help@concordia.ca}.
\item
Dr. Maria Amer's VidPro group's nodes in Speed (-01, -03, -25, -27) with additional V100 and P6 GPUs.
\item
There are various Lambda Labs other GPU servers and like computers
acquired by individual researchers; if you are member of their
research group, contact them directly. These resources are not
managed by us.
\begin{itemize}
\item
Dr. Amin Hammad's \texttt{construction.encs} Lambda Labs station
\item
Dr. Hassan Rivaz's \texttt{impactlab.encs} Lambda Labs station
\item
Dr. Nizar Bouguila's \texttt{xailab.encs} Lambda Labs station
\item
Dr. Roch Glitho's \texttt{femto.encs} server
\item
Dr. Maria Amer's \texttt{venom.encs} Lambda Labs station
\item
Dr. Leon Wang's \texttt{guerrera.encs} DGX station
\end{itemize}
\item
Dr. Ivan Contreras' servers (managed by AITS)
\item
If you are a member of School of Health (formerly PERFORM Center),
you may have access to their local 
\href
{https://perform-wiki.concordia.ca/mediawiki/index.php/HPC_Cluster}
{PERFORM's High Performance Computing (HPC) Cluster}.
Contact Thomas Beaudry for details and how to obtain access.
\item
Digital Research Alliance Canada (Compute Canada / Calcul Quebec),\\
\url{https://alliancecan.ca/}

\end{itemize}

% ------------------------------------------------------------------------------
% Refs:
%
\nocite{aosa-book-vol1}
\label{sect:bib}
%\bibliographystyle{IEEEtran}
\bibliographystyle{plain}
%\bibliographystyle{alpha}
%\bibliographystyle{unsrt}
%\bibliographystyle{abbrv}
% Create a section for references otherwise it appears to be part of the "Sister Facilities" Appendix
\clearpage
\addcontentsline{toc}{section}{Annotated Bibliography} 
\bibliography{speed-manual}

%------------------------------------------------------------------------------
\end{document}
