\documentclass{easychair}
%\documentclass[draft]{easychair}

% https://en.wikibooks.org/wiki/LaTeX/Source_Code_Listings
\usepackage{listings}

% For inline citations
\usepackage{bibentry}
\nobibliography*

% Folders with images
\makeatletter
\providecommand*{\input@path}{}
\g@addto@macro\input@path{{../src/}{src/}}% append
\g@addto@macro\input@path{{../doc/images/}{images/}}% append
\makeatother

% My own commands, commands adapted from Joey and Peter

% Cross-reference commands.
% Per Dr. Grogono and my own self.
\newcommand{\xf}[1]{Figure~\ref{#1}}
\newcommand{\xp}[1]{page~\pageref{#1}}
\newcommand{\xs}[1]{Section~\ref{#1}}
\newcommand{\xa}[1]{Appendix~\ref{#1}}
\newcommand{\xc}[1]{Chapter~\ref{#1}}
\newcommand{\xt}[1]{Table~\ref{#1}}
\newcommand{\xl}[1]{Listing~\ref{#1}}

%
% Abbrs
%

\newcommand{\rpc}{{RPC\index{RPC}}}
\newcommand{\rmi}{{RMI\index{RMI}}}
\newcommand{\clp}{{CLP\index{CLP}}}
\newcommand{\tlp}{{TLP\index{TLP}}}
\newcommand{\slp}{{SLP\index{SLP}}}
\newcommand{\complus}{{DCOM+\index{DCOM+}}}
\newcommand{\corba}{{CORBA\index{CORBA}}}
\newcommand{\jini}{{Jini\index{Jini}}}
\newcommand{\jms}{{JMS\index{JMS}}}
\newcommand{\dotnet}{{.NET Remoting\index{.NET Remoting}}}
\newcommand{\gnu}{{GNU\index{GNU}}}
\newcommand{\tcpip}{{TCP/IP\index{TCP/IP}}}
\newcommand{\AST}{{AST\index{AST}}}


%
% The GIPSY
%

\newcommand{\gipc}{{GIPC\index{GIPC}\index{Frameworks!GIPC}}}
\newcommand{\gicf}{{GICF\index{GICF}\index{Frameworks!GICF}}}
\newcommand{\iplcf}{{IPLCF\index{IPLCF}\index{Frameworks!IPLCF}}}
\newcommand{\gee}{{GEE\index{GEE}\index{Frameworks!GEE}}}
\newcommand{\geer}{{GEER\index{GEER}}}
\newcommand{\gipsy}{{GIPSY\index{GIPSY}}}
\newcommand{\agipsy}{{AGIPSY\index{AGIPSY}}}
\newcommand{\ripe}{{RIPE\index{RIPE}\index{Frameworks!RIPE}}}
\newcommand{\dpr}{{DPR\index{DPR}}}
\newcommand{\dms}{{DMS\index{DMS}}}
\newcommand{\dmf}{{DMF\index{DMF}\index{Frameworks!DMF}}}
\newcommand{\dfg}{{DFG\index{DFG}}}


%
% The Lucids Family
%

\newcommand{\glu}{{GLU\index{GLU}}}
\newcommand{\glusharp}{{GLU\#\index{GLU\#}}}
\newcommand{\gipl}{{GIPL\index{GIPL}}}
\newcommand{\sipl}{{SIPL\index{SIPL}}}
\newcommand{\ipl}{{IPL\index{IPL}}}
\newcommand{\lucid}{{Lucid\index{Lucid}}}
\newcommand{\ilucid}{{Indexical Lucid\index{Indexical Lucid}}}
\newcommand{\jlucid}{{JLucid\index{JLucid}}}
\newcommand{\olucid}{{Objective Lucid\index{Tensor Lucid}}}
\newcommand{\tlucid}{{Tensor Lucid\index{Tensor Lucid}}}
\newcommand{\plucid}{{Partial Lucid\index{Partial Lucid}}}
\newcommand{\flucid}{{Forensic Lucid\index{Forensic Lucid}}}
\newcommand{\onyx}{{Onyx\index{Onyx}}}
\newcommand{\lucx}{{Lucx\index{Lucx}}}
\newcommand{\ooip}{{OOIP\index{OOIP}}}
\newcommand{\ioop}{{IOOP\index{IOOP}}}
\newcommand{\jooip}{{JOOIP\index{JOOIP}}}

%
% The Other Intensionals
%

\newcommand{\marfl}{{MARFL\index{MARFL}}}


%
% The Imperatives
%

\newcommand{\C}{{C\index{C}}}
\newcommand{\cpp}{{C++\index{C++}}}
\newcommand{\perl}{{Perl\index{Perl}}}
\newcommand{\java}{{Java\index{Java}}}
\newcommand{\python}{{Python\index{Python}}}
\newcommand{\fortran}{{Fortran\index{Fortran}}}
\newcommand{\aspectj}{{AspectJ\index{AspectJ}}}
\newcommand{\php}{{PHP\index{PHP}}}


%
% The Functionals
%

\newcommand{\lisp}{{LISP\index{LISP}}}
\newcommand{\scheme}{{Scheme\index{Scheme}}}
\newcommand{\haskell}{{Haskell\index{Haskell}}}
\newcommand{\mllessequal}{{ML$_{\le}$\index{ML$_{\le}$}}}
\newcommand{\fcpp}{{FC++\index{FC++}}}


%
% Lucid Operators: The Original and The New
%

\newcommand{\olucidop}[1]{{\bf \texttt{\textmd{\textsc{#1}}}}}
\newcommand{\lucidop}[1]{{\bf \texttt{#1}}}


%
% Forensic terms
%

% Forward transition
\newcommand{\trans}{$\psi$}
\newcommand{\transeq}[2]{$\psi(#1) = #2$}
% Inverse transition function
\newcommand{\invtrans}{$\Psi^{-1}$}
\newcommand{\invtranseq}[2]{$\Psi^{-1}(#1) = #2$}


%
% Util
%

\newcommand{\tab}[1]{\hspace{#1pt}}

\newcommand{\shrule}[0]{\vspace{3pt}\hrule\vspace{6pt}}
\newcommand{\ehrule}[0]{\vspace{6pt}\hrule\vspace{3pt}}

\newcommand{\nonterminal}[1]{$\mathtt{<\!\!#1\!\!>}$}

\newcommand{\source}[1]
{
	{\shrule}
	\scriptsize
	#1
	\normalsize
	\hrule
}

\newcommand{\sourcefloat}[3]
{
	\begin{figure}[!hp]
	\begin{centering}
	\begin{minipage}{0.5\textwidth}
	\source{#1}
	\end{minipage}
	\caption{\small{#3}}
	\label{#2}
	\end{centering}
	\end{figure}
}

\newcommand{\todo}[0]
{
	\begin{center}{\Large [TODO]}\index{TODO}\end{center}
}

\newcommand{\file}[1]{\url{#1}\index{Files!#1}}
\newcommand{\tool}[1]{\texttt{#1}\index{Tools!#1}}
\newcommand{\option}[1]{\texttt{#1}\index{Options!#1}}
\newcommand{\api}[1]{\texttt{#1}\index{API!#1}}
\newcommand{\apipackage}[1]{\url{#1}\index{API!Packages!#1}\index{Packages!#1}}
\newcommand{\datatype}[1]{\texttt{#1}\index{Type!#1}}
\newcommand{\codesegment}[1]{\texttt{\##1}\index{Segments!\##1}}


%
% Tools
%

\newcommand{\javacc}[0]{JavaCC\index{Tools!JavaCC}}
\newcommand{\junit}[0]{JUnit\index{Tools!JUnit}}


%
% Frameworks, APIs, Libraries
%

\newcommand{\marf}[0]{MARF\index{MARF}\index{Frameworks!MARF}\index{Libraries!MARF}}
\newcommand{\dmarf}[0]{DMARF\index{MARF!Distributed}\index{Frameworks!Distributed MARF}\index{Libraries!Distributed MARF}}
\newcommand{\admarf}
	[0]
	{ADMARF%
	\index{ADMARF}%
	\index{MARF!Autonomic}%
	\index{DMARF!Autonomic}%
	\index{Frameworks!Autonomic Distributed MARF}%
	\index{Libraries!Autonomic Distributed MARF}%
	}
\newcommand{\jdsf}[0]{JDSF\index{Frameworks!JDSF}\index{Libraries!JDSF}}
\newcommand{\sqlrand}[0]{SQLrand\index{SQLrand}}
\newcommand{\hsqldb}[0]{HSQLDB\index{HSQLDB}\index{Tools!HSQLDB}\index{Databases!HSQLDB}}
\newcommand{\cryptolysis}[0]{Cryptolysis\index{Frameworks!Cryptolysis}}
\newcommand{\assl}{ASSL\index{ASSL}\index{Autonomic Systems Specification Language}}


%
% Def
%

\newcommand{\statement}[2]
{
	\vspace{7pt}
	\shrule
	{\bf #1}

	#2
	\ehrule
	\vspace{7pt}
}

% \newcommand{\proposition}[2]
\newcommand{\sproposition}[2]
{
	\statement{Proposition #1}{#2}
}

% \newcommand{\definition}[2]
\newcommand{\sdefinition}[2]
{
	\statement{Definition #1}{#2}
}

% \newcommand{\axiom}[2]
\newcommand{\saxiom}[2]
{
	\statement{Axiom #1}{#2}
}

% \newcommand{\theorem}[2]
\newcommand{\stheorem}[2]
{
	\statement{Theorem #1}{#2}
}

%
% OS
%

\newcommand{\unix}{\index{Unix@{\sc{Unix}}}{\sc{Unix}}}
\newcommand{\macos}[1]{\index{Mac OS #1@{\sc{Mac OS #1}}}{\sc{Mac OS #1}}}
\newcommand{\linux}{\index{Linux@{\sc{Linux}}}{\sc{Linux}}}
\newcommand{\rhl}[1]{\index{Red Hat Linux #1@{\sc{Red Hat Linux #1}}}{\sc{Red Hat Linux #1}}}
\newcommand{\fcore}[1]{\index{Fedora Core #1@{\sc{Fedora Core #1}}}{\sc{Fedora Core #1}}}
\newcommand{\ubuntu}[1]{\index{Ubuntu #1@{\sc{Ubuntu #1}}}{\sc{Ubuntu #1}}}
\newcommand{\debian}[1]{\index{Debian #1@{\sc{Debian #1}}}{\sc{Debian #1}}}
\newcommand{\solaris}[1]{\index{Solaris #1@{\sc{Solaris #1}}}{\sc{Solaris #1}}}
\newcommand{\win}[1]{\index{Windows #1@{\sc{Windows #1}}}{\sc{Windows #1}}}


% Joey:

\newtheorem{defn}{Definition}
\newtheorem{axioms}{Axiom}
% \newtheorem{lemma}{Lemma}
\newtheorem{lemmas}{Lemma}
% \newcommand{\web}{{WWW}}
\newcommand{\wwweb}{{WWW}}
\newcommand{\bic}{{\index{BIC}BIC}}
\newcommand{\mni}{{\index{MNI}MNI}}
\newcommand{\nfs}{{\index{NFS}NFS}}
\newcommand{\crim}{{\index{CRIM}CRIM}}
\newcommand{\animal}{\index{Animal@{\sc{Animal}}}{\sc{Animal}}}
\newcommand{\paranimal}{\index{Paranimal@{\sc{ParAnimal}}}{\sc{ParAnimal}}}
\newcommand{\minc}{{\sc{MINC}}}
\newcommand{\netcdf}{{\sc{NetCDF}}}
\newcommand{\sgi}{{\index{SGI}}SGI}
\newcommand{\vv}{{\tt{*var}}}
\newcommand{\vd}{{\tt{?var}}}
\newcommand{\tv}{{\tt{*term}}}
\newcommand{\td}{{\tt{?term}}}
\newcommand{\fv}{{\tt{*fn}}}
\newcommand{\fd}{{\tt{?fn}}}
\newcommand{\home}{{\tt{home}}}
\newcommand{\light}{{\tt{light}}}
\newcommand{\heavy}{{\tt{heavy}}}
\newcommand{\lucidA}[1]{${\mathit{Lucid}}(#1)$}
\newcommand{\lucidL}[1]{{$\mathit{Lucid}$}($L$) }
\newcommand{\tristan}{\index{Tristan}Tristan}
\newcommand{\commercial}[1]{#1}
\newcommand{\al}{\mbox{$\alpha$}}
\newcommand{\be}{\mbox{$\beta$}}
\newcommand{\ga}{\mbox{$\gamma$}}
\newcommand{\vx}[1]{\mbox{$\overrightarrow{#1}$}}
\newcommand{\lvx}[1]{\mbox{$\mid\!\!\overrightarrow{#1}\!\!\mid$}}
\newcommand{\svx}[1]{{\small \mbox{$\overrightarrow{#1}$}}}
\newcommand{\curl}[1]{\nabla\times\;\mathbf{#1}}
\newcommand{\components}[3]{{_{#3}}{#2}_{#1}}
\newcommand{\componentsp}[3]{{_{#3}}{#2}'_{#1}}
\newcommand{\mypageheader}[1]{\vspace*{22mm}{\Huge \bf #1}\vspace*{5mm}}
\newcommand{\myfig}[1]{\center{\makebox[\textwidth]{\hbox{\vbox{\epsfbox{#1}}}}}}
%\newcommand{\myfig}[1]{\makebox[\textwidth]{\hbox{\vbox{60mm}}}}
\newcommand{\ctxt}{{\mathcal L},{\mathcal D},{\mathcal P},{\mathcal W}}
\newcommand{\noWctxt}{{\mathcal L},{\mathcal D},{\mathcal P}}
\newcommand{\myvdash}{\:\vdash\:}
\newcommand{\mysemi}{\::\:}
\newcommand{\Spc}          {{\mathcal{S}}}
\newcommand{\corner}[1]    {\ulcorner #1\urcorner}
\newcommand{\db}[1]        {\{#1\}}
\newcommand{\mtt}[1]       {{\mathtt{#1}}}
\newcommand{\mrm}[1]       {{\mathrm{#1}}}
\newcommand{\mem}[1]       {{\mathit{#1}}}

\newcommand{\mathfbyd}     {{\mathtt{fby.d}}}
\newcommand{\mathfirstd}   {{\mathtt{first.d}}}
\newcommand{\mathnextd}    {{\mathtt{next.d}}}
\newcommand{\mathprevd}    {{\mathtt{prev.d}}}
\newcommand{\mathwvrd}     {{\mathtt{wvr.d}}}
\newcommand{\mathasad}     {{\mathtt{asa.d}}}
\newcommand{\mathupond}    {{\mathtt{upon.d}}}
\newcommand{\mathfby}      {{\mathtt{fby}}}
\newcommand{\mathbefore}   {{\mathtt{before}}}
\newcommand{\mathfirst}    {{\mathtt{first}}}
\newcommand{\mathnext}     {{\mathtt{next}}}
\newcommand{\mathprev}     {{\mathtt{prev}}}
\newcommand{\mathwvr}      {{\mathtt{wvr}}}
\newcommand{\mathasa}      {{\mathtt{asa}}}
\newcommand{\mathupon}     {{\mathtt{upon}}}
\newcommand{\mathif}       {{\mathtt{if}}}
\newcommand{\maththen}     {{\mathtt{then}}}
\newcommand{\mathelse}     {{\mathtt{else}}}
\newcommand{\mathfi}       {{\mathtt{fi}}}
\newcommand{\mathatd}      {{\mathtt{@.d}}}
\newcommand{\mathat}       {{\mathtt{@.}}}
\newcommand{\mathtagd}     {{\mathtt{\#.d}}}
\newcommand{\mathtag}      {{\mathtt{\#.}}}
\newcommand{\mathwhere}    {{\mathtt{where}}}
\newcommand{\mathdimension}{{\mathtt{dimension}}}
\newcommand{\mathhome}	   {{\mathtt{home}}}
\newcommand{\mathheavy}	   {{\mathtt{heavy}}}
\newcommand{\mathlight}	   {{\mathtt{light}}}
\newcommand{\mathiseod}    {{\mathtt{iseod}}}
\newcommand{\mathiserror}  {{\mathtt{iserror}}}
\newcommand{\mathend}      {{\mathtt{end}}}
\newcommand{\matheod}      {{\mathtt{eod}}}
\newcommand{\matherror}    {{\mathtt{error}}}
\newcommand{\mathtrue}     {{\mathtt{true}}}
\newcommand{\mathfalse}    {{\mathtt{false}}}
\newcommand{\Ek}           {${\mathbf{E_{k}}}$}
\newcommand{\Eop}           {${\mathbf{E_{op}}}$}
\newcommand{\Eid}           {${\mathbf{E_{id}}}$}
\newcommand{\Efid}          {${\mathbf{E_{fid}}}$}
\newcommand{\Econdt}        {${\mathbf{E_{c_{T}}}}$}
\newcommand{\Econdf}        {${\mathbf{E_{c_{F}}}}$}
\newcommand{\Ewhere}        {${\mathbf{E_{w}}}$}
\newcommand{\Eat}           {${\mathbf{E_{at}}}$}
\newcommand{\Etag}          {${\mathbf{E_{tag}}}$}
\newcommand{\Qid}           {${\mathbf{Q_{id}}}$}
\newcommand{\Qfid}          {${\mathbf{Q_{fid}}}$}
\newcommand{\QQ}            {${\mathbf{QQ}}$}
\newcommand{\const}        {{\mathit{k}}}
\newcommand{\varid}        {{\mathit{id}}}
\newcommand{\dimid}        {{\mathit{did}}}
\newcommand{\letter}       {{\mathit{letter}}}
\newcommand{\digit}        {{\mathit{digit}}}
\newcommand{\character}    {{\mathit{char}}}
\newcommand{\mystring}     {{\mathit{string}}}
\newcommand{\boolean}      {{\mathit{boolean}}}
\newcommand{\real}         {{\mathit{real}}}
\newcommand{\ASCIIchar}    {{\mathit{ASCIIchar}}}
\newcommand{\alphanum}     {{\mathit{alphanum}}}
\newcommand{\integer}      {{\mathit{integer}}}
\newcommand{\E}            {{\mathit{E}}}
%\renewcommand{\E}            {{\mathit{E}}}
\newcommand{\userfct}      {{\mathit{userfct}}}
\newcommand{\llop}         {{\textit{intensional-op}}}
\newcommand{\luop}         {{\textit{i-unary-op}}}
\newcommand{\lbop}         {{\textit{i-binary-op}}}
\newcommand{\op}           {{\textit{data-op}}}
\newcommand{\uop}          {{\textit{unary-op}}}
\newcommand{\bop}          {{\textit{binary-op}}}
\newcommand{\ifexpr}       {{\mathit{ifexpr}}}
\newcommand{\deflist}      {{\mathit{deflist}}}
%\newcommand{\dimdef}       {{\mathit{dimdef}}}
\newcommand{\fctid  }      {{\mathit{fid}}}
\newcommand{\tensorid}[2]  {{\mathit{tid_{#1}#2}}}
\newcommand{\usc}          {\mathit{\raisebox{0mm}{\_}}}
\newcommand{\dimlist}      {{\mathit{dimlist}}}
\newcommand{\Elist}        {{\mathit{Elist}}}
\newcommand{\simpleuop}    {{\mathit{mathuop}}}
\newcommand{\complexuop}   {{\mathit{intuop}}}
\newcommand{\defmy}        {{\mathit{Q}}}
\newcommand{\paramlist}    {{\mathit{parlist}}}
\newcommand{\id}           {{\mathit{identifier}}}
\newcommand{\Luciduop}     {{\mathit{Luciduop}}}
\newcommand{\Lucidbop}     {{\mathit{Lucidbop}}}
\newcommand{\simplebop}    {{\mathit{mathbop}}}
\newcommand{\complexbop}   {{\mathit{intbop}}}
\newcommand{\arithbop}     {{\textit{arith-op}}}
\newcommand{\relbop}       {{\textit{rel-op}}}
\newcommand{\logbop}       {{\textit{log-op}}}
\newcommand{\bitbop}       {{\textit{bit-op}}}
\newcommand{\seqbop}       {{\textit{seq-op}}}
\newcommand{\B}{\!\!\!\!\!\!\!\!\!\!\!\!\!\!\!\!}
\newcommand{\Bs}{\!\!\!}
\newcommand{\Bt}{\!}
\newcommand{\Dim}{{\mathcal{D}}}
\newcommand{\Point}{{\mathcal{P}}}
\newcommand{\PointP}{{\mathcal{P}}\!\dagger\!}
\newcommand{\Tag}{{\mathcal{T}}}
\newcommand{\Lang}{{\mathcal{L}}}
\newcommand{\Def}{{\mathcal{D}}}
\newcommand{\Ware}{{\mathcal{W}}}
\newcommand{\WareD}[2]{{\mathcal{W}}?\!\left\{[#1]#2\right\}}
\newcommand{\WareP}[3]{{\mathcal{W}}\!\dagger\!\left\{[#1]#2:#3\right\}}
\newcommand{\Id}{{\mathcal{I}}}
\newcommand{\Val}{{\mathcal{V}}}
\newcommand{\Stream}{{\mathcal{I}}}
\newcommand{\Expr}{{\mathcal{E}}}
\newcommand{\allExpr}{\Expr^\infty}
\newcommand{\allDim}{\Delta^\infty}
\newcommand{\allPoint}{\Pi^\infty}
\newcommand{\allStream}{\Stream^\infty}
\newcommand{\allVal}{\Val^\infty}
\newcommand{\allTag}{\Tag^\infty}
\newcommand{\extdef}{\stackrel{ext}{\equiv}}
\newcommand{\Sb}{\mathbf{Sb}}
\newcommand{\Sw}{\mathbf{Sw}}

%\newenvironment{program}
%		{\begin{quote}}
%		{\end{quote}}
\newtheorem{mydef}
		{{\bf Definition:}}
		{}
\newcommand{\paracite}[2]
		{\vspace{0.5cm}
		{\it{#1

		}}
		{\begin{flushright}---#2\end{flushright}}
}
\newcommand{\cutecite}[2]
		{\vspace{0.5cm}
		{\begin{flushright}
		{\it{#1}}\\
		---#2
		\end{flushright}}
}
\newcommand{\sembox}[3]
		{\TR{   \begin{small}
			\begin{tabular}{|p{4mm}|c|}\hline
			$\!\!$#1 & {\tt{#2}}\\\cline{2-2}
		   	   & [#3]\\\hline
			\end{tabular}
			\end{small}

		}}
%\floatstyle{boxed}
%\restylefloat{table}
%\restylefloat{figure}
%\floatname{boxtable}{Table}
%\newfloat{boxtable}{h}{lot}[chapter]


%\newcounter{definition}
%\setcounter{definition}{0}
%\newenvironment{definition}
%{
%\parindent0mm
%\parskip3mm
%\addtocounter{definition}{1}
%{\bf Definition \arabic{definition}}:
%}

%\newcounter{theorem}
%\setcounter{theorem}{0}
%\newenvironment{theorem}
%{
%\parindent0mm
%\parskip3mm
%\addtocounter{theorem}{1}
%{\bf Theorem \arabic{theorem}}:
%}

%\newtheorem{proposition}{Proposition}

\def\mymid{\vrule depth 4pt height 10pt width 0.2mm}
\def\myspace{\hspace*{3mm}}
\def\mymidspace{\mymid\myspace}
\def\myvert{\raise 2.27pt \hbox{\vrule depth 0pt height 8pt width 0.2mm}}
\def\myarrow{\hspace*{0.43mm}%
             \raise 2.29pt\hbox{\vrule depth 0pt height 8pt width 0.16mm}%
             \hspace*{-0.32mm}%
             $\longrightarrow$
             \ %
             }
\def\mmyarrow{$\rightarrow$\ }

%\psset{unit=.75cm}

\newcommand{\johndef}{\mathcal{D}}
\newcommand{\johnjvmdef}{\mathcal{D}_{jvm}}
\newcommand{\johntdef}{\mathcal{T}}
\newcommand{\myid}{\textit{id}}
\newcommand{\mytid}{\textit{tid}}
\newcommand{\mydagger}{\!\dagger\!}
\newcommand{\context}[2]{\mathcal{D},\mathcal{P} \vdash #1 : #2}
\newcommand{\jvmcontext}[2]{\mathcal{D}_{jvm} \vdash #1 : #2}
\newcommand{\pcontext}[2]{\mathcal{D},\mathcal{P},\mathcal{N} \vdash #1 : #2}
\newcommand{\contextW}[2]{\mathcal{D},\mathcal{P},\mathcal{W} \vdash #1 : #2}
\newcommand{\contextWp}[2]{\mathcal{D},\mathcal{P},\mathcal{W}' \vdash #1 : #2}
\newcommand{\qcontext}[2]{\mathcal{D},\mathcal{P} \vdash #1 \::\: #2}
\newcommand{\qjvmcontext}[2]{\mathcal{D}_{jvm} \vdash #1 \::\: #2}
\newcommand{\pqcontext}[2]{\mathcal{D},\mathcal{P},\mathcal{N} \vdash #1 \::\: #2}
\newcommand{\qcontextW}[2]{\mathcal{D},\mathcal{P,\mathcal{W}} \vdash #1 \::\: #2}
\newcommand{\myifthenelse}{\mathtt{if}\;E\;\mathtt{then}\;E'\;\mathtt{else}\;E''}

\def\Lfirst{\index{first@{\texttt{first}}}\texttt{first}\;}
\def\Lnext{\index{next@{\texttt{next}}}\texttt{next}\;}
\def\Lfby{\index{fby@{\texttt{fby}}}\;\texttt{fby}\;}
\def\Lat{\index{a@{\texttt{\char64}}}\;\texttt{\char64}\;}
\def\LSat{\index{a@{\texttt{\char64}}}\texttt{\char64}}
\def\Lhash{\index{a@{\texttt{\char35}}}\texttt{\char35}}
\def\Lwvr{\index{wvr@{\texttt{wvr}}}\;\texttt{wvr}\;}
\def\Lupon{\index{upon@{\texttt{upon}}}\;\texttt{upon}\;}
\def\LSupon{\index{upon@{\texttt{upon}}}\;\texttt{upon}}
\def\Lasa{\index{asa@{\texttt{asa}}}\;\texttt{asa}\;}
\def\Leod{\index{eod@{\texttt{eod}}}\texttt{eod}}
\def\Liseod{\index{iseod@{\texttt{iseod}}}\texttt{iseod}}
\def\Lif{\index{ifthenelse@{\texttt{if then else}}}\texttt{if}\;}
\def\Lthen{\;\texttt{then}\;}
\def\Lelse{\;\texttt{else}\;}
\def\Lsif{\index{ifthenelse@{\texttt{if then else}}}\texttt{\scriptsize if}\;}
\def\Lsthen{\;\texttt{\scriptsize then}\;}
\def\Lselse{\;\texttt{\scriptsize else}\;}

\def\mufirst{\index{first@{\texttt{first}}}\mathrm{\underline{\mathtt{first}}}\;}
\def\munext{\index{next@{\texttt{next}}}\mathrm{\underline{\mathtt{next}}}\;}
\def\mufby{\index{fby@{\texttt{fby}}}\;\mathrm{\underline{\mathtt{fby}}}\;}
\def\muwvr{\index{wvr@{\texttt{wvr}}}\;\mathrm{\underline{\mathtt{wvr}}}\;}
\def\muupon{\index{upon@{\texttt{upon}}}\;\mathrm{\underline{\mathtt{upon}}}\;}
\def\muasa{\index{asa@{\texttt{asa}}}\;\mathrm{\underline{\mathtt{asa}}}\;}

\def\mfirst{\index{first@{\texttt{first}}}\mathrm{{\mathtt{first}}}\;}
\def\mprev{\index{prev@{\texttt{prev}}}\mathrm{{\mathtt{prev}}}\;}
\def\mnext{\index{next@{\texttt{next}}}\mathrm{{\mathtt{next}}}\;}
\def\mfby{\index{fby@{\texttt{fby}}}\;\mathrm{{\mathtt{fby}}}\;}
\def\mwvr{\index{wvr@{\texttt{wvr}}}\;\mathrm{{\mathtt{wvr}}}\;}
\def\mupon{\index{upon@{\texttt{upon}}}\;\mathrm{{\mathtt{upon}}}\;}
\def\masa{\index{asa@{\texttt{asa}}}\;\mathrm{{\mathtt{asa}}}\;}

\def\Tfirst{\index{first@{\texttt{first}}}\texttt{first}}
\def\Tnext{\index{next@{\texttt{next}}}\texttt{next}}
\def\Tfby{\index{fby@{\texttt{fby}}}\texttt{fby}}
\def\Twvr{\index{wvr@{\texttt{wvr}}}\texttt{wvr}}
\def\Tupon{\index{upon@{\texttt{upon}}}\texttt{upon}}
\def\Tasa{\index{asa@{\texttt{asa}}}\texttt{asa}}

\newcommand{\eqdef}{\stackrel{{\mathrm{def}}}{=}}
\newcommand{\mylinebefore}{\noindent\rule{.1mm}{3mm}\rule[3mm]{.995\textwidth}{.1mm}\rule{.1mm}{3mm}\vspace*{-5mm}}
\newcommand{\mylineafter}{\vspace*{-5mm}\noindent\rule{.1mm}{3mm}\rule{.995\textwidth}{.1mm}\rule{.1mm}{3mm}}
\newcommand{\myprop}[1]{
\mylinebefore
\begin{proposition}
#1
\end{proposition}
\mylineafter
}


%% Document
%%
\begin{document}

% ------------------------------------------------------------------------------
%% Front Matter
%%
% Regular title as in the article class.
%
\title{Speed: The GCS ENCS Cluster}

% \titlerunning{} has to be set to either the main title or its shorter
% version for the running heads. Use {\sf} for highlighting your system
% name, application, or a tool.
%
\titlerunning{Speed: The GCS ENCS Cluster}

% Previously VI
%\date{Version 6.5}
%\date{\textbf{Version 6.6-dev-07}}
%\date{\textbf{Version 6.6} (final GE version)}
%\date{\textbf{Version 7.0-dev-01}}
%\date{\textbf{Version 7.0}}
%\date{\textbf{Version 7.1}}
\date{\textbf{Version 7.2}}

% Authors are joined by \and and their affiliations are on the
% subsequent lines separated by \\ just like the article class
% allows.
%
\author{
    Serguei A. Mokhov
\and
    Gillian A. Roper
\and
    Carlos Alarcón Meza
\and
    Farah Salhany
\and
    Network, Security and HPC Group\footnote{The group acknowledges the initial manual version VI produced by Dr.~Scott Bunnell while with us
		as well as Dr.~Tariq Daradkeh for his instructional support of the users and contribution of examples.}\\
    \affiliation{Gina Cody School of Engineering and Computer Science}\\
    \affiliation{Concordia University}\\
    \affiliation{Montreal, Quebec, Canada}\\
    \affiliation{\url{rt-ex-hpc~AT~encs.concordia.ca}}\\
}

% \authorrunning{} has to be set for the shorter version of the authors' names;
% otherwise a warning will be rendered in the running heads.
%
\authorrunning{Mokhov, Roper, Alarcón Meza, Salhany, NAG/HPC, GCS ENCS}
\indexedauthor{Mokhov, Serguei}
\indexedauthor{Roper, Gillian}
\indexedauthor{Alarcón Meza, Carlos}
\indexedauthor{Salhany, Farah}
\indexedauthor{NAG/HPC}

%%%%%%%%%%%%%%%%%%%%%%%%%%%%%%%%%%%%%%%%%%%%%%%%%%%
\maketitle
%%%%%%%%%%%%%%%%%%%%%%%%%%%%%%%%%%%%%%%%%%%%%%%%%%%

% ------------------------------------------------------------------------------
\begin{abstract}
This document serves as a quick start guide to using the Gina Cody School of Engineering and Computer Science (GCS ENCS) 
compute server farm, known as ``Speed.'' Managed by the HPC/NAG group of the 
Academic Information Technology Services (AITS) at GCS, Concordia University, Montreal, Canada.
\end{abstract}

% ------------------------------------------------------------------------------
\tableofcontents
\clearpage

% ------------------------------------------------------------------------------
%						1 Introduction
% ------------------------------------------------------------------------------
\section{Introduction}
\label{sect:introduction}

This document contains basic information required to use ``Speed'', along with tips, 
tricks, examples, and references to projects and papers that have used Speed.
User contributions of sample jobs and/or references are welcome.\\

\noindent
\textbf{Note:} On October 20, 2023, we completed the migration to SLURM 
from Grid Engine (UGE/AGE) as our job scheduler. 
This manual has been updated to use SLURM's syntax and commands. 
If you are a long-time GE user, refer to \xa{appdx:uge-to-slurm} for key highlights needed to 
translate your GE jobs to SLURM as well as environment changes. 
These changes are also elaborated throughout this document and our examples.

% ------------------------------------------------------------------------------
\subsection{Citing Us}
\label{sect:citing-speed-hpc}

If you wish to cite this work in your acknowledgements, you can use our general DOI found on our GitHub page
\url{https://dx.doi.org/10.5281/zenodo.5683642} or a specific version of the manual and scripts from that link individually.
You can also use the ``cite this repository'' feature of GitHub.

% ----------------------------- 1.1 Resources ----------------------------------
% ------------------------------------------------------------------------------
\subsection{Resources}
\label{sect:resources}

\begin{itemize}
	\item
	Public GitHub page where the manual and sample job scripts are maintained at\\
	\url{https://github.com/NAG-DevOps/speed-hpc}
		\begin{itemize}
			\item Pull requests (PRs) are subject to review and are welcome:\\
			\url{https://github.com/NAG-DevOps/speed-hpc/pulls}
		\end{itemize}

	\item
	Speed Manual:
		\begin{itemize}
			\item PDF version of the manual:\\
			\url{https://github.com/NAG-DevOps/speed-hpc/blob/master/doc/speed-manual.pdf}
			
			\item HTML version of the manual:\\
			\url{https://nag-devops.github.io/speed-hpc/}
		\end{itemize}

	\item
	Concordia official page for ``Speed'' cluster which includes access request instructions.
	\url{https://www.concordia.ca/ginacody/aits/speed.html}

	\item
	All Speed users are subscribed to the \texttt{hpc-ml} mailing list.

\end{itemize}

% TODO: for now comment out for 7.0; if when we update that
%       preso, we will re-link it here. However, keep the citation.
\nocite{speed-intro-preso}
%\item
%\href
%	{https://docs.google.com/presentation/d/1zu4OQBU7mbj0e34Wr3ILXLPWomkhBgqGZ8j8xYrLf44}
%	{Speed Server Farm Presentation 2022}~\cite{speed-intro-preso}.

% ----------------------------- 1.2 Team ---------------------------------------
% ------------------------------------------------------------------------------
\subsection{Team}
\label{sect:speed-team}

Speed is supported by:
\begin{itemize}
	\item 
	Serguei Mokhov, PhD, Manager, Networks, Security and HPC, AITS
	\item 
	Gillian Roper, Senior Systems Administrator, HPC, AITS
	\item 
	Carlos Alarcón Meza, Systems Administrator, HPC and Networking, AITS
	\item 
	Farah Salhany, IT Instructional Specialist, AITS
\end{itemize}

\noindent We receive support from the rest of AITS teams, such as NAG, SAG, FIS, and DOG.\\
\url{https://www.concordia.ca/ginacody/aits.html}


% ----------------------------- 1.3 What Speed Consists of ---------------------
% ------------------------------------------------------------------------------
\subsection{What Speed Consists of}
\label{sect:speed-arch}

\begin{itemize}
	\item
	Twenty four (24) 32-core compute nodes, each with 512~GB of memory and 
	approximately 1~TB of local volatile-scratch disk space (pictured in \xf{fig:speed-pics}).

	\item
	Twelve (12) NVIDIA Tesla P6 GPUs, with 16~GB of GPU memory (compatible with the 
	CUDA, OpenGL, OpenCL, and Vulkan APIs). 

	\item
	4 VIDPRO nodes (ECE. Dr.~Amer), with 6 P6 cards, and 6 V100 cards (32GB), and 
	256GB of RAM.

	\item
	7 new SPEED2 servers with 256 CPU cores each 4x~A100 80~GB GPUs, partitioned
	into 4x~20GB MIGs  each; larger local storage for TMPDIR (see \xf{fig:speed-architecture-full}).

	\item
	One AMD FirePro S7150 GPU, with 8~GB of memory (compatible with the
	Direct~X, OpenGL, OpenCL, and Vulkan APIs).

 	\item
  	Salus compute node (CSSE CLAC, Drs.~Bergler and Kosseim), 56 cores and 728GB of RAM, see \xf{fig:speed-architecture-full}.

	\item
 	Magic subcluster partition (ECE, Dr.~Khendek, 11 nodes, see \xf{fig:speed-architecture-full}).

  	\item
   	Nebular subcluster partition (CIISE, Drs. Yan, Assi, Ghafouri, et al., Nebulae GPU node with 2x RTX 48GB cards,
    	Stellar compute node, and Matrix 177TB storage/compute node, see \xf{fig:speed-architecture-full}).
\end{itemize}

\begin{figure}[htpb]
	\centering
	\includegraphics[width=\columnwidth]{images/speed-pics}
	\caption{Speed}
	\label{fig:speed-pics}
\end{figure}

\begin{figure}[htpb]
	\centering
	\includegraphics[width=\columnwidth]{images/speed-architecture-full}
	\caption{Speed Cluster Hardware Architecture}
	\label{fig:speed-architecture-full}
\end{figure}

\begin{figure}[htpb]
	\centering
	\includegraphics[width=\columnwidth]{images/slurm-arch}
	\caption{Speed SLURM Architecture}
	\label{fig:slurm-arch}
\end{figure}

% ----------------------------- 1.4 What Speed Is Ideal For --------------------
% ------------------------------------------------------------------------------
\subsection{What Speed Is Ideal For}
\label{sect:speed-is-for}

\begin{itemize}
	\item
	Design, develop, test, and run parallel, batch, and other algorithms and scripts with partial data sets.
	``Speed'' has been optimized for compute jobs that are multi-core aware,
	require a large memory space, or are iteration intensive.

	\item
	Prepare jobs for large clusters such as:
		\begin{itemize}
			\item Digital Research Alliance of Canada (Calcul Quebec and Compute Canada)
			\item Cloud platforms
		\end{itemize}
	\item
	Jobs that are too demanding for a desktop. 
	\item
	Single-core batch jobs; multithreaded jobs typically up to 32 cores (i.e., a single machine).
	\item
	Multi-node multi-core jobs (MPI).
	\item
	Anything that can fit into a 500-GB memory space and a \textbf{speed scratch} space of approximately 10~TB. 
	\item
	CPU-based jobs. 
	\item
	CUDA GPU jobs.
	\item
	Non-CUDA GPU jobs using OpenCL.
\end{itemize}

% ----------------------------- 1.5 What Speed Is Not --------------------------
% ------------------------------------------------------------------------------
\subsection{What Speed Is Not}
\label{sect:speed-is-not}

\begin{itemize}
	\item Speed is not a web host and does not host websites.
	\item Speed is not meant for Continuous Integration (CI) automation deployments for Ansible or similar tools. 
	\item Does not run Kubernetes or other container orchestration software.
	\item Does not run Docker. (\textbf{Note:} Speed does run Singularity and many Docker containers can be converted to Singularity 
	containers with a single command. See \xs{sect:singularity-containers}.)
	\item Speed is not for jobs executed outside of the scheduler. (Jobs running outside of the scheduler will be killed and all data lost.)
\end{itemize}

% ----------------------------- 1.6 Available Software -------------------------
% ------------------------------------------------------------------------------
\subsection{Available Software}
\label{sect:available-software}

There are a wide range of open-source and commercial software available and installed on ``Speed.'' 
This includes Abaqus~\cite{abaqus}, AllenNLP, Anaconda (Anaconda, Anaconda2, Anaconda3), ANSYS, Bazel,
COMSOL, CPLEX, CUDA, Eclipse, Fluent~\cite{fluent}, Gurobi, MATLAB~\cite{matlab,scholarpedia-matlab}, 
OMNeT++, OpenCV, OpenFOAM, OpenMPI, OpenPMIx, ParaView, PyTorch, QEMU, R, Rust, and Singularity.
Programming environments include various versions of Python, C++/Java compilers, TensorFlow, OpenGL, OpenISS, and {\marf}~\cite{marf}.\\

In particular, there are over 2200 programs available in \texttt{/encs/bin} and \texttt{/encs/pkg} under Scientific Linux 7 (EL7).
We are building an equivalent array of programs for the EL9 SPEED2 nodes. To see the packages available, run \texttt{ls -al /encs/pkg/} on \texttt{speed.encs}.\\

\noindent
\textbf{Note:} We do our best to accommodate custom software requests. Python environments can use user-custom installs 
from within the scratch directory.

% ----------------------------- 1.7 Requesting Access --------------------------
% ------------------------------------------------------------------------------
\subsection{Requesting Access}
\label{sect:access-requests}

After reviewing the ``What Speed is'' (\xs{sect:speed-is-for}) and
``What Speed is Not'' (\xs{sect:speed-is-not}), request access to the ``Speed'' 
cluster by emailing: \texttt{rt-ex-hpc AT encs.concordia.ca}.

\begin{itemize} 
	\item GCS ENCS faculty and staff may request access directly.
	\item GCS students must include the following in their request message:
	\begin{itemize}
		\item GCS ENCS username
		\item Name and email (CC) of the approver -- either a supervisor, course instructor,
		or a department representative (e.g., in the case of undergraduate or M.Eng.\ students it
		can be the Chair, associate chair, a technical officer, or a department administrator) for approval.
		\item Written request from the
		%supervisor or instructor
		approver
		for the GCS ENCS username to be granted access to ``Speed.''
	\end{itemize}
	\item Non-GCS students taking a GCS course will have their GCS ENCS account created automatically, but still need the course instructor's approval to use the service.
	\item Non-GCS faculty and students need to get a ``sponsor'' within GCS, so that a guest GCS ENCS account is created first. A sponsor can be any GCS Faculty member
	you collaborate with. Failing that, request the approval from our Dean's Office;
	via our Associate Deans Drs.~Eddie Hoi Ng or Emad Shihab.
	\item External entities collaborating with GCS Concordia researchers should also go through the Dean's Office for approvals.
\end{itemize}

% The web page is currently less detailed than the above.
%For detailed instructions, refer to the Concordia 
%\href{https://www.concordia.ca/ginacody/aits/speed.html}{Computing (HPC) Facility: Speed} webpage.

% ------------------------------------------------------------------------------
%						2 Job Management
% ------------------------------------------------------------------------------
\section{Job Management}
\label{sect:job-management}

We use SLURM as the workload manager. It supports primarily two types of jobs: batch and interactive.
Batch jobs are used to run unattended tasks, whereas, 
interactive jobs are are ideal for setting up virtual environments, compilation, and debugging.\\

\noindent \textbf{Note:} In the following instructions, anything bracketed like, \verb+<>+, indicates a
label/value to be replaced (the entire bracketed term needs replacement).\\

\noindent Job instructions in a script start with \verb+#SBATCH+ prefix, for example:
\begin{verbatim}
    #SBATCH --mem=100M -t 600 -J <job-name> -A <slurm account>
    #SBATCH -p pg --gpus=2 --mail-type=ALL
\end{verbatim}
%
For complex compute steps within a script, use \tool{srun}. We recommend using \tool{salloc} for interactive jobs as it supports multiple steps.
However, srun can also be used to start interactive jobs (see \xs{sect:interactive-jobs}).
%
Common and required job parameters include:
\begin{verbatim}
memory (--mem), time (-t), job-name (-J), slurm project account (-A), partition (-p), 
mail type (--mail-type), ntasks (-n), CPUs per task (--cpus-per-task).
\end{verbatim}

% -------------- 2.1 Getting Started ------------------------
% -----------------------------------------------------------
\subsection{Getting Started}
\label{sect:getting-started}

Before getting started, please review the ``What Speed is'' (\xs{sect:speed-is-for})
and ``What Speed is Not'' (\xs{sect:speed-is-not}).
Once your GCS ENCS account has been granted access to ``Speed'',
use your GCS ENCS account credentials to create an SSH connection to 
\texttt{speed} (an alias for \texttt{speed-submit.encs.concordia.ca}).\\

All users are expected to have a basic understanding of
Linux and its commonly used commands (see \xa{sect:faqs} for resources).

%  2.1.1 SSH Connections 
% -----------------------
\subsubsection{SSH Connections}
\label{sect:ssh}

Requirements to create connections to ``Speed'':
\begin{enumerate}
	\item \textbf{Active GCS ENCS user account:} Ensure you have an active GCS ENCS user account with 
	permission to connect to Speed (see \xs{sect:access-requests}).
	\item \textbf{VPN Connection} (for off-campus access): If you are off-campus, you wil need to establish an active connection to Concordia's VPN, 
	which requires a Concordia netname.
	\item \textbf{Terminal Emulator for Windows:} Windows systems use a terminal emulator such as PuTTY, Cygwin, or MobaXterm.
	\item \textbf{Terminal for macOS:} macOS systems have a built-in Terminal app or \tool{xterm} that comes with XQuartz.
\end{enumerate}

\noindent To create an SSH connection to Speed, open a terminal window and type the following command, replacing \verb!<ENCSusername>! with your ENCS account's username:
\begin{verbatim}
    ssh <ENCSusername>@speed.encs.concordia.ca
\end{verbatim}

\noindent For detailed instructions on securely connecting to a GCS server, refer to the AITS FAQ: 
\href{https://www.concordia.ca/ginacody/aits/support/faq/ssh-to-gcs.html}{How do I securely connect to a GCS server?}

%  2.1.2 Environment Set Up
% --------------------------
% TMP scheduler-specific section
\subsubsection{Environment Set Up}
\label{sect:envsetup}
% ------------------------------------------------------------------------------
\subsubsection{Environment Set Up}
\label{sect:envsetup}

After creating an SSH connection to Speed, you will need to
make sure the \tool{srun}, \tool{sbatch}, and \tool{salloc}
commands are available to you. 
Type the command name at the command prompt and press enter.
If the command is not available, e.g., (``command not found'') is returned,
you need to make sure your \api{\$PATH} has \texttt{/local/bin} in it.
To view your \api{\$PATH} type \texttt{echo \$PATH} at the prompt.
%
%source 
%the ``Altair Grid Engine (AGE)'' scheduler's settings file. 
%Sourcing the settings file will set the environment variables required to 
%execute scheduler commands.
%
%Based on the UNIX shell type, choose one of the following commands to source
%the settings file. 
%
%csh/\tool{tcsh}:
%\begin{verbatim}
%source /local/pkg/uge-8.6.3/root/default/common/settings.csh 
%\end{verbatim}
%
%Bourne shell/\tool{bash}:
%\begin{verbatim}
%. /local/pkg/uge-8.6.3/root/default/common/settings.sh 
%\end{verbatim}
%
%In order to set up the default ENCS bash shell, executing the following command 
%is also required:
%\begin{verbatim}
%printenv ORGANIZATION | grep -qw ENCS || . /encs/Share/bash/profile 
%\end{verbatim}
%
%To verify that you have access to the scheduler commands execute 
%\texttt{qstat -f -u "*"}. If an error is returned, attempt sourcing 
%the settings file again.

The next step is to copy a job template to your home directory and to set up your
cluster-specific storage. Execute the following command from within your
home directory. (To move to your home directory, type \texttt{cd} at the Linux
prompt and press \texttt{Enter}.) 

\begin{verbatim}
cp /home/n/nul-uge/template.sh . && mkdir /speed-scratch/$USER
\end{verbatim}

%\textbf{Tip:} Add the source command to your shell-startup script. 

\textbf{Tip:} the default shell for GCS ENCS users is \tool{tcsh}.
If you would like to use \tool{bash}, please contact 
\texttt{rt-ex-hpc AT encs.concordia.ca}.

%For \textbf{new GCS ENCS Users}, and/or those who don't have a shell-startup script, 
%based on your shell type use one of the following commands to copy a start up script 
%from \texttt{nul-uge}'s home directory to your home directory. (To move to your home
%directory, type \tool{cd} at the Linux prompt and press \texttt{Enter}.)

%csh/\tool{tcsh}:
%\begin{verbatim}
%cp /home/n/nul-uge/.tcshrc . 
%\end{verbatim}

%Bourne shell/\tool{bash}:
%\begin{verbatim}
%cp /home/n/nul-uge/.bashrc . 
%\end{verbatim}

%Users who already have a shell-startup script, can use a text editor, such as
%\tool{vim} or \tool{emacs}, to add the source request to your existing
%shell-startup environment (i.e., to the \file{.tcshrc} file in your home directory). 

%csh/\tool{tcsh}:
%Sample \file{.tcshrc} file:
%\begin{verbatim}
%# Speed environment set up 
%if ($HOSTNAME == speed-submit.encs.concordia.ca) then
   %source /local/pkg/uge-8.6.3/root/default/common/settings.csh
%endif
%\end{verbatim}
%
%Bourne shell/\tool{bash}:
%Sample \file{.bashrc} file:
%\begin{verbatim}
%# Speed environment set up 
%if [ $HOSTNAME = "speed-submit.encs.concordia.ca" ]; then
    %. /local/pkg/uge-8.6.3/root/default/common/settings.sh
    %printenv ORGANIZATION | grep -qw ENCS || . /encs/Share/bash/profile
%fi
%\end{verbatim}

Note, if you are getting ``command not found'' error(s) when logging in, you
probably have old Grid Engine environment commands. Remove them
as per \xa{appdx:uge-to-slurm}.


% -------------- 2.2 Job Submission Basics ------------------
% -----------------------------------------------------------
\subsection{Job Submission Basics}
\label{sect:job-submission-basics}

Preparing your job for submission is fairly straightforward.
Start by basing your job script on one of the examples available in the \texttt{src/}
directory of our \href{https://github.com/NAG-DevOps/speed-hpc}{GitHub repository}.
You can clone the repository to get the examples to start with via the command line:

\begin{verbatim}
    git clone --depth=1 https://github.com/NAG-DevOps/speed-hpc.git
    cd speed-hpc/src
\end{verbatim}

\noindent The job script is a shell script that contains directives, module loads, and user scripting.
To quickly run some sample jobs, use the following commands:
\begin{verbatim}
    sbatch -p ps -t 10 env.sh
    sbatch -p ps -t 10 bash.sh
    sbatch -p ps -t 10 manual.sh
    sbatch -p pg -t 10 lambdal-singularity.sh
\end{verbatim}

%  2.2.1 Directives
% -------------------
% TMP scheduler-specific section
\subsubsection{Directives}
\label{sect:directives}
% ------------------------------------------------------------------------------
\subsubsection{Directives}
\label{sect:directives}

Directives are comments included at the beginning of a job script that set the shell 
and the options for the job scheduler. 

The shebang directive is always the first line of a script. In your job script, 
this directive sets which shell your script's commands will run in. On ``Speed'', 
we recommend that your script use a shell from the \texttt{/encs/bin} directory. 

To use the \texttt{tcsh} shell, start your script with: \verb|#!/encs/bin/tcsh|

For \texttt{bash}, start with: \verb|#!/encs/bin/bash|

Directives that start with \verb|#SBATCH|, set the options for the cluster's 
SLURM scheduler. The script template, \file{template.sh}, 
provides the essentials:

%\begin{verbatim}
%#$ -N <jobname>
%#$ -cwd
%#$ -m bea
%#$ -pe smp <corecount>
%#$ -l h_vmem=<memory>G
%\end{verbatim}
\begin{verbatim}
#SBATCH --job-name=tmpdir           ## Give the job a name
#SBATCH --mail-type=ALL             ## Receive all email type notifications
#SBATCH --mail-user=$USER@encs.concordia.ca
#SBATCH --chdir=./                  ## Use current directory as working directory
#SBATCH --nodes=1
#SBATCH --ntasks=1
#SBATCH --cpus-per-task=<corecount> ## Request, e.g. 8 cores
#SBATCH --mem=<memory>              ## Assign, e.g., 32G memory per node 
\end{verbatim}

and its short option equivalents:

\begin{verbatim}
#SBATCH -J tmpdir                   ## Give the job a name
#SBATCH --mail-type=ALL             ## Receive all email type notifications
#SBATCH --mail-user=$USER@encs.concordia.ca
#SBATCH --chdir=./                  ## Use current directory as working directory
#SBATCH -N 1
#SBATCH --ntasks=1
#SBATCH -n 8                        ## Request 8 cores
#SBATCH --mem=32G                   ## Assign 32G memory per node 
\end{verbatim}

Replace, \verb+<jobname>+, with the name that you want your cluster job to have;
\option{--chdir}, makes the current working directory the ``job working directory'',
and your standard output file will appear here; \option{--mail-type}, provides e-mail
notifications (success, error, etc. or all); replace, \verb+<corecount>+, with the degree of
(multithreaded) parallelism (i.e., cores) you attach to your job (up to 32 by default).
%be sure to delete or comment out the \verb| #$ -pe smp | parameter if it 
%is not relevant;
Replace, \verb+<memory>+, with the value (in GB), that you want 
your job's memory space to be (up to 500 depending on the node), and all jobs MUST have a memory-space 
assignment.

If you are unsure about memory footprints, err on assigning a generous
memory space to your job, so that it does not get prematurely terminated.
%(the value given to \api{h\_vmem} is a hard memory ceiling).
You can refine
%\api{h\_vmem}
\option{--mem}
values for future jobs by monitoring the size of a job's active
memory space on \texttt{speed-submit} with:

%\begin{verbatim}
%qstat -j <jobID> | grep maxvmem
%\end{verbatim}

\begin{verbatim}
sstat -j <jobID>
\end{verbatim}

\noindent
This can be customized to show specific columns:

\begin{verbatim}
sstat -o jobid,maxvmsize,ntasks%7,tresusageouttot%25 -j <jobID>
\end{verbatim}

Memory-footprint values are also provided for completed jobs in the final
e-mail notification (as, ``maxvmsize'').

\emph{Jobs that request a low-memory footprint are more likely to load on a busy
cluster.}


%  2.2.2 Module Loads
% -------------------
\subsubsection{Module Loads}
\label{sect:modules}

After setting the directives in your job script, the next section typically involves loading 
the necessary software modules. The \texttt{module} command is used to manage the user environment, 
make sure to load all the modules your job depends on. You can check available modules with the 
module avail command. Loading the correct modules ensures that your environment is properly 
set up for execution.

\begin{verbatim}
    module avail
    module -t avail matlab		## show the list for a particular program (e.g., matlab)
    module -t avail m		    ## show the list for all programs starting with m     
\end{verbatim}

\noindent To list for a particular program (\tool{matlab}, for example):

\noindent For example, insert the following in your script to load the \tool{matlab/R2023a} module:
\begin{verbatim}
    module load matlab/R2023a/default
\end{verbatim}

\noindent \textbf{Note:} you can remove a module from active use by replacing \option{load} by \option{unload}.\\
    
\noindent To list loaded modules:
\begin{verbatim}
    module list
\end{verbatim}
    
\noindent To purge all software in your working environment:
\begin{verbatim}
    module purge
\end{verbatim}

%  2.2.3 User Scripting
% -------------------
% TMP scheduler-specific section
\subsubsection{User Scripting}
\label{sect:scripting}
%  2.2.3 User Scripting
% -------------------
% TMP scheduler-specific section

The final part of the job script involves the commands that will be executed by the job.
This section should include all necessary commands to set up and run the tasks 
your script is designed to perform. You can use any Linux command in this section, 
ranging from a simple executable call to a complex loop iterating through multiple commands.\\

\noindent \textbf{Best Practice}: prefix any compute-heavy step with \tool{srun}.
This ensures you gain proper insights on the execution of your job.\\

\noindent Each software program may have its own execution framework, as it's the script's author (e.g., you) 
responsibility to review the software's documentation to understand its requirements.
Your script should be written to clearly specify the location of input and output files and the degree of parallelism needed.\\

\noindent Jobs that involve multiple interactions with data input and output files, should make use of \api{TMPDIR}, 
a scheduler-provided workspace nearly 1~TB in size.
\api{TMPDIR} is created on the local disk of the compute node at the start of a job, offering faster I/O operations 
compared to shared storage (provided over NFS).

An sample job script using \api{TMPDIR} is available at \texttt{/home/n/nul-uge/templateTMPDIR.sh}: 
the job is instructed to change to \api{\$TMPDIR}, to make the new directory \texttt{input}, to copy data from
\texttt{\$SLURM\_SUBMIT\_DIR/references/} to \texttt{input/} (\api{\$SLURM\_SUBMIT\_DIR} represents the
current working directory), to make the new directory \texttt{results}, to
execute the program (which takes input from \texttt{\$TMPDIR/input/} and writes
output to \texttt{\$TMPDIR/results/}), and finally to copy the total end results
to an existing directory, \texttt{processed}, that is located in the current
working directory.
% TODO: verify:
\api{TMPDIR} only exists for the duration of the job, though,
so it is very important to copy relevant results from it at job's end.

% -------------- 2.3 Sample Job Script ----------------------
% -----------------------------------------------------------
\subsection{Sample Job Script}
\label{sect:sample-job-script}

Here's a basic job script, \file{tcsh.sh} shown in \xf{fig:tcsh.sh}.
You can copy it from our \href{https://github.com/NAG-DevOps/speed-hpc}{GitHub repository}.

\begin{figure}[htpb]
	\lstinputlisting[language=csh,frame=single,basicstyle=\ttfamily]{tcsh.sh}
	\caption{Source code for \file{tcsh.sh}}
	\label{fig:tcsh.sh}
\end{figure}

\noindent
The first line is the shell declaration (also know as a shebang) and sets the shell to \emph{tcsh}.
The lines that begin with \texttt{\#SBATCH} are directives for the scheduler.
\begin{itemize}
	\item \option{-J} (or \option{--job-name}) sets \emph{tcsh-test} as the job name.
	%\item \texttt{--chdir} tells the scheduler to execute the job from the current working directory
	\item \option{--mem=1GB} requests and assigns 1GB of memory to the job. 
	Jobs require the \option{--mem} option to be set either in the script
	or on the command line; \textbf{if it's missing, job submission will be rejected.}
\end{itemize}

\noindent The script then:
\begin{enumerate}
	\item Sleeps on a node for 30 seconds.
	\item Uses the \tool{module} command to load the \texttt{gurobi/8.1.0} environment.
	\item Prints the list of loaded modules into a file.
\end{enumerate}

\noindent
The scheduler command, \tool{sbatch}, is used to submit (non-interactive) jobs. 
From an ssh session on ``speed-submit'', submit this job with
\begin{verbatim}
    sbatch ./tcsh.sh
\end{verbatim}

\noindent
You will see, \texttt{Submitted batch job 2653} where $2653$ is a job ID assigned.
The commands \tool{squeue} and \tool{sinfo} can be used 
to look at the status of the cluster:
%\texttt{squeue -l} and \texttt{sinfo -la}.

\small
\begin{verbatim}
[serguei@speed-submit src] % squeue -l
Thu Oct 19 11:38:54 2023
JOBID PARTITION     NAME     USER    STATE       TIME TIME_LIMI  NODES NODELIST(REASON)
 2641        ps interact   b_user  RUNNING   19:16:09 1-00:00:00      1 speed-07
 2652        ps interact   a_user  RUNNING      41:40 1-00:00:00      1 speed-07
 2654        ps tcsh-tes  serguei  RUNNING       0:01 7-00:00:00      1 speed-07
[serguei@speed-submit src] % sinfo
PARTITION AVAIL  TIMELIMIT  NODES  STATE NODELIST
ps*          up 7-00:00:00     14  drain speed-[08-10,12,15-16,20-22,30-32,35-36]
ps*          up 7-00:00:00      1    mix speed-07
ps*          up 7-00:00:00      7   idle speed-[11,19,23-24,29,33-34]
pg           up 1-00:00:00      1  drain speed-17
pg           up 1-00:00:00      3   idle speed-[05,25,27]
pt           up 7-00:00:00      7   idle speed-[37-43]
pa           up 7-00:00:00      4   idle speed-[01,03,25,27]
\end{verbatim}
\normalsize

\noindent
\textbf{Remember} that you only have 30 seconds before the job is essentially over, so 
if you do not see a similar output, either adjust the sleep time in the 
script, or execute the \tool{squeue} statement more quickly. The \tool{squeue} 
output listed above shows that your job 2654 is running on node \texttt{speed-07}, 
and its time limit is 7 days, etc.\\
% TODO
%, that it 
%was started at 16:39:30 on 12/03/2018, and that it is a single-core job (the 
%default). 

Once the job finishes, there will be a new file in the directory that the job 
was started from, with the syntax of, \texttt{slurm-<job id>.out}, so 
in this example the file is, \file{slurm-2654.out}. This file represents the 
standard output (and error, if there is any) of the job in question. If you 
look at the contents of your newly created file, you will see that it 
contains the output of the, \texttt{module list} command. 
Important information is often written to this file.
%
%Congratulations on your first job! 

% -------------- 2.4 Common Job Management Commands Summary ---
% -------------------------------------------------------------
\subsection{Common Job Management Commands Summary}
\label{sect:job-management-commands}

Here is a summary of useful job management commands for handling various aspects of 
job submission and monitoring on the Speed cluster:

\begin{itemize}
	\item Submitting a job:
	\small
	\begin{verbatim}
		sbatch -A <ACCOUNT> -t <MINUTES> --mem=<MEMORY> -p <PARTITION> ./<myscript>.sh
	\end{verbatim}
	\normalsize

	\item Checking your job(s) status:
	\small
	\begin{verbatim}
		squeue -u <ENCSusername>
	\end{verbatim}
	\normalsize

	\item Displaying cluster status:
	\small
	\begin{verbatim}
		squeue
	\end{verbatim}
	\normalsize
		\begin{itemize}
			\item Use \option{-A} for per account (e.g., \texttt{-A vidpro}, \texttt{-A aits}), 
			\item Use \option{-p} for per partition (e.g., \texttt{-p ps}, \texttt{-p pg}, \texttt{-p pt}), etc.
		\end{itemize}

	\item Displaying job information:
	\small
	\begin{verbatim}
		squeue --job <job-ID>
	\end{verbatim}
	\normalsize

	\item Displaying individual job steps: (to see which step failed if you used \tool{srun})
	\small
	\begin{verbatim}
		squeue -las
	\end{verbatim}
	\normalsize

	\item Monitoring job and cluster status: (view \tool{sinfo} and watch the queue for your job(s))
	\small
	\begin{verbatim}
		watch -n 1 "sinfo -Nel -pps,pt,pg,pa && squeue -la"
	\end{verbatim}
	\normalsize

	\item Canceling a job:
	\small
	\begin{verbatim}
		scancel <job-ID>
	\end{verbatim}
	\normalsize

	\item Holding a job:
	\small
	\begin{verbatim}
		scontrol hold <job-ID>
	\end{verbatim}
	\normalsize

	\item Releasing a job:
	\small
	\begin{verbatim}
		scontrol release <job-ID>
	\end{verbatim}
	\normalsize

	\item Getting job statistics: (including useful metrics like ``maxvmem'')
	\small
	\begin{verbatim}
		sacct -j <job-ID>
	\end{verbatim}
	\normalsize
	
	\api{maxvmem} is one of the more useful stats that you can elect to display
	as a format option.
	\small
	\begin{verbatim}
	% sacct -j 2654
	JobID           JobName  Partition    Account  AllocCPUS      State ExitCode
	------------ ---------- ---------- ---------- ---------- ---------- --------
	2654          tcsh-test         ps     speed1          1  COMPLETED      0:0
	2654.batch        batch                speed1          1  COMPLETED      0:0
	2654.extern      extern                speed1          1  COMPLETED      0:0
	% sacct -j 2654 -o jobid,user,account,MaxVMSize,Reason%10,TRESUsageOutMax%30
	JobID             User    Account  MaxVMSize     Reason        TRESUsageOutMax
	------------ --------- ---------- ---------- ---------- ----------------------
	2654           serguei     speed1                  None
	2654.batch                 speed1    296840K             energy=0,fs/disk=1975
	2654.extern                speed1    296312K              energy=0,fs/disk=343
	\end{verbatim}
	\normalsize

	See \texttt{man sacct} or \texttt{sacct -e} for details of the available formatting options. 
	You can define your preferred default format in the \api{SACCT\_FORMAT} environment variable
	in your \texttt{.cshrc} or \texttt{.bashrc} files.

	\item Displaying job efficiency: (including CPU and memory utilization)
	\small
	\begin{verbatim}
	seff <job-ID>
	\end{verbatim}
	\normalsize
	
	Don't execute it on \texttt{RUNNING} jobs (only on completed/finished jobs), else
	efficiency statistics may be misleading. If you define the following 
	directive in your batch script, your GCS ENCS email address will receive an email 
	with \tool{seff}'s output when your job is finished.

	\small
	\begin{verbatim}
	#SBATCH --mail-type=ALL        
	\end{verbatim}
	\normalsize

	Output example:
	\small
	\begin{verbatim}
	Job ID: XXXXX
	Cluster: speed
	User/Group: user1/user1
	State: COMPLETED (exit code 0)
	Nodes: 1
	Cores per node: 4
	CPU Utilized: 00:04:29
	CPU Efficiency: 0.35% of 21:32:20 core-walltime
	Job Wall-clock time: 05:23:05
	Memory Utilized: 2.90 GB
	Memory Efficiency: 2.90% of 100.00 GB
	\end{verbatim}
	\normalsize
\end{itemize}


% -------------- 2.5 Advanced sbatch Options ------------------
% -------------------------------------------------------------
\subsection{Advanced \tool{sbatch} Options}
\label{sect:submit-options}
\label{sect:qsub-options}

In addition to the basic sbatch options presented earlier, 
there are several advanced options that are generally useful:

\begin{itemize}
	\item E-mail notifications:
	\begin{verbatim}
		--mail-type=<TYPE>
	\end{verbatim}
	Requests the scheduler to send an email when the job changes state.
	\texttt{<TYPE>} can be \texttt{ALL}, \texttt{BEGIN}, \texttt{END}, or \texttt{FAIL}.
	Mail is sent to the default address of,
	%
	\begin{verbatim}
	<ENCSusername>@encs.concordia.ca 
	\end{verbatim}
	%
	which you can consult via \url{webmail.encs.concordia.ca} (use VPN from off-campus)
	unless a different address is supplied 
	(see, \option{--mail-user}).
	The report sent when a job ends includes job 
	runtime, as well as the maximum memory value hit (\api{maxvmem}). 
	\begin{verbatim}
		--mail-user email@domain.com
	\end{verbatim}
	Specifies a different email address for notifications rather than the default.

	\item Export environment variables used by the script.:
	\begin{verbatim}
		--export=ALL
		--export=NONE
		--export=VARIABLES
	\end{verbatim}

	\item Job runtime:
	\begin{verbatim}
		-t <MINUTES> or DAYS-HH:MM:SS
	\end{verbatim} 
	sets a job runtime of min or HH:MM:SS. Note that if you give a single number,
	that represents \emph{minutes}, not hours. The set runtime should not exceed
	the default maximums of 24h for interactive jobs and 7 days for batch jobs.

	\item Job Dependencies:
	\begin{verbatim}
		--depend=<state:job-ID>
	\end{verbatim} 
	Runs the job only when the specified job \verb|<job-ID>| finishes. This is useful for creating job chains where 
	subsequent jobs depend on the completion of previous ones.
\end{itemize}

\noindent \textbf{Note:} \tool{sbatch} options can be specified during the job-submission 
command, and these \emph{override} existing script options (if present). The 
syntax is
\begin{verbatim}
	sbatch [options] PATHTOSCRIPT
\end{verbatim}
but unlike in the script, the options are specified without the leading \verb+#SBATCH+
e.g.: 
\begin{verbatim}
	sbatch -J sub-test --chdir=./ --mem=1G ./tcsh.sh
\end{verbatim}

% -------------- 2.6 Array Jobs -------------------------------
% -------------------------------------------------------------
\subsection{Array Jobs}
\label{sect:array-jobs}

Array jobs are those that start a batch job or a parallel job multiple times.
Each iteration of the job array is called a task and receives a unique job ID.
Array jobs are particularly useful for running a large number of similar tasks with slight variations.\\

\noindent
To submit an array job (Only supported for batch jobs), use the \option{--array} option of the \tool{sbatch} 
command as follows:

\begin{verbatim}
	sbatch --array=n-m[:s]] <batch_script>
\end{verbatim}

\noindent \textbf{where}
\begin{itemize}
	\item
	\texttt{n}: indicates the start-id.
	\item
	\texttt{m}: indicates the max-id.
	\item
	\texttt{s}: indicates the step size.
\end{itemize}

\noindent \textbf{Examples:}
\begin{itemize}
	\item Submit a job with 1 task where the task-id is 10. 
	\begin{verbatim}
		sbatch --array=10 array.sh
	\end{verbatim}

	\item Submit a job with 10 tasks numbered consecutively from 1 to 10.
	\begin{verbatim}
		sbatch --array=1-10 array.sh
	\end{verbatim}

	\item Submit a job with 5 tasks numbered consecutively with a step size of 3 (task-ids 3,6,9,12,15)
	\begin{verbatim}
		sbatch --array=3-15:3 array.sh
	\end{verbatim}

	\item Submit a job with 50000 elements, where \%a maps to the task-id between 1 and 50K. 
	\begin{verbatim}
		sbatch --array=1-50000 -N1 -i my_in_%a -o my_out_%a array.sh
	\end{verbatim}
\end{itemize}

\noindent \textbf{Output files for Array Jobs:}\\
The default output and error-files are \texttt{slurm-job\_id\_task\_id.out}.
%
This means that Speed creates an output and an error-file for each task 
generated by the array-job, as well as one for the super-ordinate array-job. 
To alter this behavior use the \option{-o} and \option{-e} options of \tool{sbatch}.\\

For more details about Array Job options, please review the manual pages for 
\tool{sbatch} by executing the following at the command line on \tool{speed-submit}
\texttt{man sbatch}.
 
% -------------- 2.7 Requesting Multiple Cores ----------------
% -------------------------------------------------------------
\subsection{Requesting Multiple Cores (i.e., Multithreading Jobs)}
\label{sect:multicore-jobs}

For jobs that can take advantage of multiple machine cores, you can 
request up to 32 cores (per job) in your script using the following options:

\begin{verbatim}
	#SBATCH -n <#cores for processes>
	#SBATCH -n 1
	#SBATCH -c <#cores for threads of a single process>
\end{verbatim}

\noindent Both \tool{sbatch} and \tool{salloc} support \option{-n} on the command line,
and it should always be used either in the script or on the command line as the
default $n=1$.\\

\noindent \textbf{Important Considerations}:
\begin{itemize}
	\item Do not request more cores than you think will be useful, 
	as larger-core jobs are more difficult to schedule.

	\item If you are running a program that scales out to the maximum single-machine 
	core count available, please request 32 cores to avoid node 
	oversubscription (i.e., overloading the CPUs).
\end{itemize}

\noindent \textbf{Note:} \option{--ntasks} or \option{--ntasks-per-node}
(\option{-n}) refers to processes (usually the ones run with \tool{srun}).
\option{--cpus-per-task} (\option{-c}) corresponds to threads per process.\\

\noindent Some programs consider them equivalent, while others do not. For example, 
Fluent uses \option{--ntasks-per-node=8} and \option{--cpus-per-task=1},
whereas others may set \option{--cpus-per-task=8} and \option{--ntasks-per-node=1}.
If one of these is not 1, some applications need to be configured to use \texttt{n * c} total cores.\\

\noindent Core count associated with a job appears under,
``AllocCPUS'', in the, \texttt{sacct -j <job-id>}, output.

\small
\begin{verbatim}
	[serguei@speed-submit src] % squeue -l
	Thu Oct 19 20:32:32 2023
	JOBID PARTITION     NAME     USER    STATE       TIME TIME_LIMI  NODES NODELIST(REASON)
	2652        ps interact   a_user  RUNNING   9:35:18 1-00:00:00      1 speed-07
	[serguei@speed-submit src] % sacct -j 2652
	JobID           JobName  Partition    Account  AllocCPUS      State ExitCode
	------------ ---------- ---------- ---------- ---------- ---------- --------
	2652         interacti+         ps     speed1         20    RUNNING      0:0
	2652.intera+ interacti+                speed1         20    RUNNING      0:0
	2652.extern      extern                speed1         20    RUNNING      0:0
	2652.0       gydra_pmi+                speed1         20  COMPLETED      0:0
	2652.1       gydra_pmi+                speed1         20  COMPLETED      0:0
	2652.2       gydra_pmi+                speed1         20     FAILED      7:0
	2652.3       gydra_pmi+                speed1         20     FAILED      7:0
	2652.4       gydra_pmi+                speed1         20  COMPLETED      0:0
	2652.5       gydra_pmi+                speed1         20  COMPLETED      0:0
	2652.6       gydra_pmi+                speed1         20  COMPLETED      0:0
	2652.7       gydra_pmi+                speed1         20  COMPLETED      0:0
\end{verbatim}
\normalsize

% -------------- 2.8 Interactive Jobs -------------------------
% -------------------------------------------------------------
\subsection{Interactive Jobs}
\label{sect:interactive-jobs}

Interactive job sessions allow you to interact with the system in real-time. 
These sessions are particularly useful for tasks such as testing, debugging, optimizing code, 
setting up environments, and other preparatory work before submitting batch jobs.

%  2.8.1 Command Line
% -------------------
\subsubsection{Command Line}
\label{sect:command-line}

To request an interactive job session, use the \texttt{salloc} command with appropriate options.
This is similar to submitting a batch job but allows you to run shell commands interactively 
within the allocated resources. For example:
\begin{verbatim}
	salloc -J interactive-test --mem=1G -p ps -n 8
\end{verbatim}

Within the allocated \tool{salloc} session, you can run shell commands as usual. 
It is recommended to use \tool{srun} for compute-intensive steps within \tool{salloc}. 
If you need a quick, short job just to compile something on a GPU node, 
you can use an interactive srun directly. For example, a 1-hour allocation:\\

\noindent \textbf{For tcsh}:
\begin{verbatim}
	srun --pty -n 8 -p pg --gpus=1 --mem=1G -t 60 /encs/bin/tcsh
\end{verbatim}

\noindent \textbf{For bash}:
\begin{verbatim}
	srun --pty -n 8 -p pg --gpus=1 --mem=1G -t 60 /encs/bin/bash
\end{verbatim}

%  2.8.2 Graphical Applications
% -------------------
\subsubsection{Graphical Applications}
\label{sect:graphical-applications}

To run graphical UI applications (e.g., MALTLAB, Abaqus CME, IDEs like PyCharm, VSCode, Eclipse, etc.) on Speed, 
you need to enable X11 forwarding from your client machine Speed then to the compute node.
To do so, follow these steps:

\begin{enumerate}
	\item Run an X server on your client machine:
	\begin{itemize}
		\item \textbf{Windows:} Use MobaXterm with X turned on, or Xming + PuTTY with X11 forwarding, or XOrg under Cygwin
		\item \textbf{macOS:} Use XQuarz -- use its \tool{xterm} and \texttt{ssh -X}
		\item \textbf{Linux:} Use \texttt{ssh -X speed.encs.concordia.ca}
	\end{itemize}
	For more details, see \href{https://www.concordia.ca/ginacody/aits/support/faq/xserver.html}{How do I remotely launch X(Graphical) applications?}

	\item Verify that X11 forwarding is enabled by printing the \api{DISPLAY} variable:
	\begin{verbatim}
		echo $DISPLAY
	\end{verbatim}

	\item Start an interactive session with X11 forwarding enabled (Use the \option{--x11} with \tool{salloc} or \tool{srun}), for example:
	\begin{verbatim}
		salloc -p ps --x11=first --mem=4G -t 0-06:00
	\end{verbatim}

	\item Once landed on a compute node, verify \api{DISPLAY} again.
	
	\item Set the \api{XDG\_RUNTIME\_DIR} variable to a directory in your \tool{speed-scratch} space:
	\begin{verbatim}
		mkdir -p /speed-scratch/$USER/run-dir
		setenv XDG_RUNTIME_DIR /speed-scratch/$USER/run-dir
	\end{verbatim}
	
	\item Launch your graphical application:
	\begin{verbatim}
		module load matlab/R2023a/default 
		matlab
	\end{verbatim}
\end{enumerate}

\noindent
\textbf{Note:} with X11 forwarding the graphical rendering is happening on
your client machine! That is you are not using GPUs on Speed to render
graphics, instead all graphical information is forwarded from Speed to
your desktop or laptop over X11, which in turn renders it using its
own graphics card. Thus, for GPU rendering jobs either keep them
non-interactive or use VirtualGL.\\

\noindent
Here's an example of starting PyCharm (see \xf{fig:pycharm}). 
\textbf{Note:} If using VSCode, it's currently only supported with the \tool{--no-sandbox} option.\\

\noindent \textbf{TCSH version:}
\small
\begin{verbatim}
ssh -X speed (XQuartz xterm, PuTTY or MobaXterm have X11 forwarding too)
[speed-submit] [/home/c/carlos] > echo $DISPLAY
localhost:14.0
[speed-submit] [/home/c/carlos] > cd /speed-scratch/$USER
[speed-submit] [/speed-scratch/carlos] > echo $DISPLAY
localhost:13.0
[speed-submit] [/speed-scratch/carlos] > salloc -pps --x11=first --mem=4Gb -t 0-06:00
[speed-07] [/speed-scratch/carlos] > echo $DISPLAY
localhost:42.0
[speed-07] [/speed-scratch/carlos] > hostname
speed-07.encs.concordia.ca
[speed-07] [/speed-scratch/carlos] > setenv XDG_RUNTIME_DIR /speed-scratch/$USER/run-dir
[speed-07] [/speed-scratch/carlos] > /speed-scratch/nag-public/bin/pycharm.sh
\end{verbatim}
\normalsize
\noindent \textbf{BASH version:}
\small
\begin{verbatim}
bash-3.2$ ssh -X speed (XQuartz xterm, PuTTY or MobaXterm have X11 forwarding too)
serguei@speed's password: 
[serguei@speed-submit ~] % echo $DISPLAY
localhost:14.0
[serguei@speed-submit ~] % salloc -p ps --x11=first --mem=4Gb -t 0-06:00 
bash-4.4$ echo $DISPLAY
localhost:77.0
bash-4.4$ hostname
speed-01.encs.concordia.ca
bash-4.4$ export XDG_RUNTIME_DIR=/speed-scratch/$USER/run-dir
bash-4.4$ /speed-scratch/nag-public/bin/pycharm.sh
\end{verbatim}
\normalsize

\begin{figure}[htpb]
	\includegraphics[width=\columnwidth]{images/pycharm}
	\caption{Launching PyCharm on a Speed Node}
	\label{fig:pycharm}
\end{figure}

% -----------------------------------------------------------------------------
\subsubsection{Jupyter Notebooks}
\label{sect:jupyter}

%  2.8.3 Jupyter Notebooks in Singularity
% -------------------
%\subsubsection{Jupyter Notebook in Singularity}
\paragraph{Jupyter Notebook in Singularity}
\label{sect:jupyter-singularity}

To run Jupyter Notebooks using Singularity (more on Singularity see \xs{sect:singularity-containers}), follow these steps:

\begin{enumerate}
  % X11 is not really needed for Jupyter since we tunnel and use a browser
	%\item Connect to Speed with X11 forwarding enabled:
	\item Connect to Speed, e.g. interactively, using \tool{salloc}
	%\item Use the \option{--x11} with \tool{salloc} or \tool{srun} as described in the above example
	\item Load Singularity module
		\verb+module load singularity/3.10.4/default+

	\item
	Execute this Singularity command on a single line or save it in a shell script
	\href{https://github.com/NAG-DevOps/speed-hpc/blob/master/src/jupyter.sh}{from our GitHub}
	where you could easily invoke it.
	
	\scriptsize
	\begin{verbatim}
srun singularity exec -B $PWD\:/speed-pwd,/speed-scratch/$USER\:/my-speed-scratch,/nettemp \
--env SHELL=/bin/bash --nv /speed-scratch/nag-public/openiss-cuda-conda-jupyter.sif \
/bin/bash -c '/opt/conda/bin/jupyter notebook --no-browser --notebook-dir=/speed-pwd \
--ip="*" --port=8888 --allow-root'
	\end{verbatim}
	\normalsize

	\item
	In a new terminal window, create an \tool{ssh} tunnel between your computer and the node (\texttt{speed-XX}) where Jupyter is
	running (using \texttt{speed-submit} as a ``jump server'', see, e.g., in PuTTY, in \xf{fig:putty1} and \xf{fig:putty2})
	\small
	\begin{verbatim}
		ssh -L 8888:speed-XX:8888 <ENCS-username>@speed-submit.encs.concordia.ca
	\end{verbatim}
	\normalsize
	Don't close the tunnel after establishing.

	\item
	Open a browser, and copy your Jupyter's token (it's printed to you in the terminal)
	and paste it in the browser's URL field.
	In our case, the URL is:
	\small
	\begin{verbatim}
		http://localhost:8888/?token=5a52e6c0c7dfc111008a803e5303371ed0462d3d547ac3fb
	\end{verbatim}
	\normalsize

	\item Access the Jupyter Notebook interface in your browser.
\end{enumerate}

\begin{figure}[htbp]
	\centering
	\fbox{\includegraphics{images/putty1}}
	\caption{SSH tunnel configuration 1}
	\label{fig:putty1}
\end{figure}

\begin{figure}[htbp]
	\centering
	\fbox{\includegraphics{images/putty2}}
	\caption{SSH tunnel configuration 2}
	\label{fig:putty2}
\end{figure}

\begin{figure}[htbp]
	\centering
	\fbox{\includegraphics[width=1.00\textwidth]{images/jupyter.png}}
	\caption{Jupyter running on a Speed node}
	\label{fig:jupyter}
\end{figure}

\noindent
Another sample is the OpenISS-derived containers with Conda and Jupyter,
see \xs{sect:openiss-examples} for details.

%  2.8.4 JupyterLab in Conda and Pytorch
% -------------------
%\subsubsection{JupyterLab in Conda and Pytorch}
\paragraph{JupyterLab in Conda and Pytorch}
\label{sect:jupyterlabs}

For setting up Jupyter Labs with Conda and Pytorch, follow these steps:

\begin{itemize}
	\item Environment preparation: (only once, takes some time to run to install all required dependencies)
	\begin{enumerate}
		\item Navigate to your speed-scratch directory:
		\begin{verbatim}
			cd /speed-scratch/\$USER
		\end{verbatim}

		\item Create a Jupyter (name of your choice) directory and \tool{cd} into it:
		\begin{verbatim}
			mkdir -p Jupyter
			cd Jupyter
		\end{verbatim}

		\item Start an interactive session:
		\begin{verbatim}
			salloc --mem=50G --gpus=1 -ppg (or -ppt)
		\end{verbatim}
		
		\item
		Set \tool{conda} environment variables, and install \tool{jupyterlab} and \tool{pytorch},
		as shown in \xf{fig:firsttime.sh} from our GitHub.
		
		\begin{figure}[htpb]
			\tiny
			\lstinputlisting[language=csh,frame=single,basicstyle=\ttfamily]{../src/jupyterlabs/firsttime.sh}
			\normalsize
			\caption{Source code for \texttt{firsttime.sh}}
			\label{fig:firsttime.sh}
		\end{figure}
	\end{enumerate}

	\item
	Execution of Jupyter Labs from \textbf{speed-submit} (repeat this every time you want to run Jupyter Labs):
	\begin{enumerate}
		\item Start an interactive session:
		\begin{verbatim}
			salloc --mem=50G --gpus=1 -p pg (or -p pt)
		\end{verbatim}

		\item
		Activate your \tool{conda} environment and run Jupyter Labs, as shown in
		\xf{fig:run.sh} (also available on our GitHub).
		
		\begin{figure}[htpb]
			\scriptsize
			\lstinputlisting[language=csh,frame=single,basicstyle=\ttfamily]{../src/jupyterlabs/run.sh}
			\normalsize
			\caption{Source code for \texttt{run.sh}}
			\label{fig:run.sh}
		\end{figure}

		\item
		Verify which port the system has assigned to your Jupyter Lab instance by examining the URL
		\texttt{http://localhost:XXXX/lab?token=} in your terminal as described
		previously.

		\item
		In a new terminal window, create an \tool{ssh} tunnel similar to Jupyter 
    in Singularity, see \xs{sect:jupyter-singularity}.

		\item
		Open a browser and copy your Jupyter's token and paste it in the browser's URL field
	\end{enumerate}
\end{itemize}

%  2.8.5 JupyterLab + Pytorch in Python venv
% -------------------
%\subsubsection{JupyterLab + Pytorch in Python venv}
\paragraph{JupyterLab + Pytorch in Python venv}
\label{sect:jupyterlabs-venv}

This is an example of Jupyter Labs running in a Python Virtual environment (\texttt{venv}), with Pytorch on Speed.\\

\noindent
\textbf{Note:} Use of Python virtual environments is preferred over Conda at Alliance Canada clusters.
If you prefer to make jobs that are more compatible between Speed and Alliance clusters, use Python
\texttt{venv}s. See \url{https://docs.alliancecan.ca/wiki/Anaconda/en}
and \url{https://docs.alliancecan.ca/wiki/JupyterNotebook}.

\begin{itemize}
\item
Environment preparation: for the FIRST time only:
\begin{enumerate}
\item
Go to your speed-scratch directory: \texttt{cd /speed-scratch/\$USER}
\item
Open an interactive session: \texttt{salloc --mem=50G --gpus=1 --constraint=el9}
\item
Create a Python \texttt{venv} and install \tool{jupyterlab}+\tool{pytorch}
\scriptsize
\begin{verbatim}
module load python/3.11.5/default
setenv TMPDIR /speed-scratch/$USER/tmp
setenv TMP /speed-scratch/$USER/tmp
setenv PIP_CACHE_DIR /speed-scratch/$USER/tmp/cache
python -m venv /speed-scratch/$USER/tmp/jupyter-venv
source /speed-scratch/$USER/tmp/jupyter-venv/bin/activate.csh
pip install jupyterlab
pip3 install torch torchvision torchaudio --index-url https://download.pytorch.org/whl/cu118
exit
\end{verbatim}
\normalsize
\end{enumerate}
\item
Running Jupyter Labs, from \textbf{speed-submit}:
\begin{enumerate}
\item
Open an interactive session: \texttt{salloc --mem=50G --gpus=1 --constraint=el9} 
\scriptsize
\begin{verbatim}
cd /speed-scratch/$USER
module load python/3.11.5/default
setenv PIP_CACHE_DIR /speed-scratch/$USER/tmp/cache
source /speed-scratch/$USER/tmp/jupyter-venv/bin/activate.csh
jupyter lab --no-browser --notebook-dir=$PWD --ip="0.0.0.0" --port=8888 --port-retries=50
\end{verbatim}
\normalsize
\item
Verify which port the system has assigned to Jupyter:\\
\texttt{http://localhost:XXXX/lab?token=}
\item
SSH Tunnel creation: similar to Jupyter in Singularity, see \xs{sect:jupyter-singularity}
\item
Open a browser and type: \texttt {localhost:XXXX} (using the port assigned)
\end{enumerate}
\end{itemize}


%  2.8.6 Visual Studio Code
% -------------------
\subsubsection{Visual Studio Code}
\label{sect:vscode}

This is an example of running VScode, it's similar to Jupyter notebooks, but 
it doesn't use containers. \textbf{Note:} this a Web-based version; there exists the local
(workstation)~--~remote (speed-node) client-server version too, but it is for advanced users
and is out of scope here (so no support, use it at your own risk).
 
\begin{itemize}

\item
Environment preparation: for the FIRST time:
\begin{enumerate}
\item
Go to your speed-scratch directory: \texttt{cd /speed-scratch/\$USER}
\item
Create a vscode directory: \texttt{mkdir vscode}
\item
Go to vscode: \texttt{cd vscode}
\item
Create home and projects: \texttt{mkdir \{home,projects\}}
\item
Create this directory: \texttt{mkdir -p /speed-scratch/\$USER/run-user}
\end{enumerate}

\item
Running VScode
\begin{enumerate}
\item 
Go to your vscode directory: \texttt{cd /speed-scratch/\$USER/vscode}
\item
Open interactive session: \texttt{salloc --mem=10Gb --constraint=el9}
\item
Set environment variable:\\\texttt{setenv XDG\_RUNTIME\_DIR /speed-scratch/\$USER/run-user}
\item 
Run VScode, change the port if needed.
\scriptsize
\begin{verbatim}
/speed-scratch/nag-public/code-server-4.22.1/bin/code-server --user-data-dir=$PWD\/projects \
--config=$PWD\/home/.config/code-server/config.yaml --bind-addr="0.0.0.0:8080" $PWD\/projects
\end{verbatim}
\normalsize
\item
SSH Tunnel creation: similar to Jupyter, see \xs{sect:jupyter-singularity}
\item
Open a browser and type: \texttt{localhost:8080}
\item
If the browser asks for a password, consult:
\begin{verbatim}
cat /speed-scratch/$USER/vscode/home/.config/code-server/config.yaml
\end{verbatim}

\end{enumerate}
\end{itemize}

\begin{figure}[htbp]
	\centering
	\fbox{\includegraphics[width=1.00\textwidth]{images/vscode.png}}
	\caption{VScode running on a Speed node}
	\label{fig:vscode}
\end{figure}

% scheduler-scripting also includes: 
% 2.3 Sample Job Script
% 2.4 Common Job Management Commands Summary
% 2.5 Advanced sbatch Options
% 2.6 Array Jobs
% 2.7 Requesting Multiple Cores
% 2.8 Interactive Jobs
%  	2.8.1 Command Line
%  	2.8.2 Graphical Applications
%	2.8.3 Jupyter Notebooks in Singularity
%	2.8.4 JupyterLab in Conda and Pytorch
% 	2.8.5 JupyterLab + Pytorch in Python venv
%  	2.8.6 Visual Studio Code

% -------------- 2.9 Scheduler Environment Variables ----------
% -------------------------------------------------------------
\subsection{Scheduler Environment Variables}
\label{sect:env-vars}

The scheduler provides several environment variables that can be useful in your job scripts. 
These variables can be accessed within the job using commands like \tool{env} or \tool{printenv}. 
Many of these variables start with the prefix \texttt{SLURM}.\\

\noindent Here are some of the most useful environment variables:
%\api{TMPDIR}, \api{SGE\_O\_WORKDIR}, and \api{NSLOTS}:

\begin{itemize}
	\item \api{\$TMPDIR}:
	% TODO: verify temporal existence
	This is the path to the job's temporary space on the node. It \emph{only} exists for the duration of the job.
	If you need the data from this temporary space, ensure you copy it before the job terminates.

	\item \api{\$SLURM\_SUBMIT\_DIR}:
	% TODO: verify if home or current:
	%\api{\$SGE\_O\_WORKDIR}=the path to the job's working directory (likely an
	%NFS-mounted path). If, \texttt{-cwd}, was stipulated, that path is taken; if not, 
	%the path defaults to your home directory.
	The path to the job's working directory (likely an NFS-mounted path).
	If, \option{--chdir}, was stipulated, that path is taken; if not, 
	the path defaults to your home directory.
	
	\item \api{\$SLURM\_JOBID}:
	% TODO: SLURM does not appear to have this
	This variable holds the current job's ID, which is useful for job manipulation and reporting.

	% SLURM_NTASKS
	%\item
	%\api{\$NSLOTS}=the number of cores requested for the job. This variable can 
	%be used in place of hardcoded thread-request declarations. 

	\item \api{\$SLURM\_JOB\_NODELIST}:
	This lists the nodes participating in your job.

	\item \api{\$SLURM\_ARRAY\_TASK\_ID}:
	For array jobs, this variable represents the task ID 
	(refer to \xs{sect:array-jobs} for more details on array jobs).
\end{itemize}

\noindent For a more comprehensive list of environment variables, refer to the SLURM documentation for 
\href{https://slurm.schedmd.com/srun.html#SECTION_INPUT-ENVIRONMENT-VARIABLES}{Input Environment Variables} and
\href{https://slurm.schedmd.com/srun.html#SECTION_OUTPUT-ENVIRONMENT-VARIABLES}{Output Environment Variables}.\\

\noindent Here's an example script (see \xf{fig:tmpdir.sh}) that utilizes some of these environment variables:

\begin{figure}[htpb]
    \lstinputlisting[language=csh,frame=single,basicstyle=\scriptsize\ttfamily]{tmpdir.sh}
    \caption{Source code for \file{tmpdir.sh}}
	\label{fig:tmpdir.sh}
\end{figure}

% -------------- 2.10 SSH Keys for MPI ------------------------
% -------------------------------------------------------------
\subsection{SSH Keys for MPI}
\label{sect:ssh-mpi}

Some programs, such as Fluent, utilize MPI (Message Passing Interface) for parallel processing. 
MPI requires `passwordless login', which is achieved through SSH keys. Here are the steps to set up SSH keys for MPI:

\begin{itemize}
	\item Navigate to the \texttt{.ssh} directory
	\begin{verbatim}cd .ssh\end{verbatim}

	\item Generate a new SSH key pair (Accept the default location and leave the passphrase blank)
	\begin{verbatim}ssh-keygen -t ed25519\end{verbatim}

	\item Authorize the Public Key:
	\begin{verbatim}cat id_ed25519.pub >> authorized_keys\end{verbatim} 
	If the \texttt{\href{https://www.ssh.com/academy/ssh/authorized-keys-file}{authorized\_keys}} file does not exist, use
	\begin{verbatim}cat id_ed25519.pub > authorized_keys\end{verbatim}

	\item Set permissions: ensure the correct permissions are set for the `authorized\_keys' file and your home directory,
	(most users will already have these permissions by default.):
	\begin{verbatim}
	chmod 600 ~/.ssh/authorized_keys
	chmod 700 ~
	\end{verbatim}
\end{itemize}

% -------------- 2.11 Creating Virtual Environments -----------
% -------------------------------------------------------------
\subsection{Creating Virtual Environments}
\label{sect:environments}
\label{sect:examples-venv}

The following documentation is specific to \textbf{Speed} HPC Facility at the
Gina Cody School of Engineering and Computer Science.
%
Virtual environments are typically created using Conda or Python.
Another option is Singularity (detailed in \xs{sect:singularity-containers}).
These environments are usually created once during an interactive session 
before submitting a batch job to the scheduler. 
The job script submitted to the scheduler should:
\begin{enumerate}
	\item Activate the virtual environment.
	\item Use the virtual environment.
	\item Deactivate the virtual environment at the end of the job.
\end{enumerate}

%  2.11.1 Anaconda
% -------------------
\subsubsection{Anaconda}
\label{sect:conda-venv}

To create an Anaconda environment, follow these steps:
\begin{enumerate}
	\item Request an interactive session
	\begin{verbatim}
		salloc -p pg --gpus=1
	\end{verbatim}

	\item Load Anaconda module and create Anaconda environment in your speed-scratch directory by using 
	the \texttt{\-\-prefix} option (without this option, the environment will be created in your home directory by default).
	\begin{verbatim}
		module load anaconda3/2023.03/default
		conda create --prefix /speed-scratch/$USER/myconda
	\end{verbatim}

	\item List environments (to view your conda environment)
	\begin{verbatim}
		conda info --envs
		# conda environments:
		#
		base                  *  /encs/pkg/anaconda3-2023.03/root
                         		/speed-scratch/a_user/myconda
	\end{verbatim}

	\item Activate the environment
	\begin{verbatim}
		conda activate /speed-scratch/$USER/myconda
	\end{verbatim}

	\item Add \tool{pip} to your environment (this will install \tool{pip} and \tool{pip}'s dependencies
	, including python, into the environment.)
	\begin{verbatim}
		conda install pip
	\end{verbatim}
\end{enumerate}   

\noindent A consolidated example using Conda:
\begin{verbatim}
	salloc -p pg --gpus=1 --mem=10GB -A <slurm account name>
	cd /speed-scratch/$USER
	module load python/3.11.0/default
	conda create -p /speed-scratch/$USER/pytorch-env
	conda activate /speed-scratch/$USER/pytorch-env
	conda install python=3.11.0
	pip3 install torch torchvision torchaudio --index-url \ 
		https://download.pytorch.org/whl/cu117
	....
	conda deactivate
	exit # end the salloc session
\end{verbatim}

\noindent If you encounter \textbf{no space left error} while creating Conda environment, refer to our Github 
\href{https://github.com/NAG-DevOps/speed-hpc/tree/master/src#no-space-left-error-when-creating-conda-environment}
{\texttt{HERE}}.\\

\noindent \textbf{Important Note:} \tool{pip} (and \tool{pip3}) are package installers for Python. When you use
\texttt{pip install}, it installs packages from the Python Package Index (PyPI), whereas, 
\texttt{conda install} installs packages from Anaconda's repository.

\paragraph{Conda Env without \tool{--prefix}:}
If you don't want to use the \texttt{--prefix} option every time you create a new environment and 
do not want to use the default home directory, you can create a new directory and set the following 
variables to point to the new created directory, e.g:
\begin{verbatim}
	mkdir -p /speed-scratch/$USER/conda
	setenv CONDA_ENVS_PATH /speed-scratch/$USER/conda
	setenv CONDA_PKGS_DIRS /speed-scratch/$USER/conda/pkg
\end{verbatim}
\noindent If you want to make these changes permanent, add the variables to your \texttt{.tcshrc} 
or \texttt{.bashrc} (depending on the default shell you are using)

%  2.11.2 Python
% -------------------
\subsubsection{Python}
\label{sect:python-venv}
Setting up a Python virtual environment is straightforward.
Here's an example that use a Python virtual environment:

\begin{verbatim}
	salloc -p pg --gpus=1 --mem=10GB -A <slurm account name>
	cd /speed-scratch/$USER
	module load python/3.9.1/default
	mkdir -p /speed-scratch/$USER/tmp 
	setenv TMPDIR /speed-scratch/$USER/tmp
	setenv TMP /speed-scratch/$USER/tmp
	python -m venv $TMPDIR/testenv (testenv=name of the virtualEnv)
	source /speed-scratch/$USER/tmp/testenv/bin/activate.csh
	pip install modules…
	deactivate
	exit
\end{verbatim}

\noindent See, e.g., \href
{https://github.com/NAG-DevOps/speed-hpc/blob/master/src/gurobi-with-python.sh}{\texttt{gurobi-with-python.sh}}\\

\noindent \textbf{Important Note:} partition \texttt{ps} is used for CPU jobs, while \texttt{pg}, \texttt{pt} are used
for GPU jobs. You do not need to use \texttt{--gpus} when preparing environments for CPU jobs.

% -------------- 2.12 Example Job Script: Fluent --------------
% -------------------------------------------------------------
% TMP scheduler-specific section
% TODO: delete the file and move the content here
% ------------------------------------------------------------------------------
\subsection{Example Job Script: Fluent}

\begin{figure}[htpb]
    \lstinputlisting[language=csh,frame=single,basicstyle=\footnotesize\ttfamily]{fluent.sh}
    \caption{Source code for \file{fluent.sh}}
	\label{fig:fluent.sh}
\end{figure}

The job script in \xf{fig:fluent.sh} runs Fluent in parallel over 32 cores. 
Of note, we have requested e-mail notifications (\texttt{-m}), are defining the 
parallel environment for, \tool{fluent}, with, \texttt{-sgepe smp} (\textbf{very 
important}), and are setting \api{\$TMPDIR} as the in-job location for the
``moment'' \file{rfile.out} file (in-job, because the last line of the script 
copies everything from \api{\$TMPDIR} to a directory in the user's NFS-mounted home). 
Job progress can be monitored by examining the standard-out file (e.g.,
\file{flu10000.o249}), and/or by examining the ``moment'' file in 
\texttt{/disk/nobackup/<yourjob>} (hint: it starts with your job-ID) on the node running
the job. \textbf{Caveat:} take care with journal-file file paths.

% ------------------------------------------------------------------------------
\subsection{Example Job: efficientdet}

The following steps describing how to create an efficientdet environment on
\emph{Speed}, were submitted by a member of Dr. Amer's research group.

\begin{itemize}
    \item 
    Enter your ENCS user account's speed-scratch directory 
    \verb!cd /speed-scratch/<encs_username>!
    \item
    load python \verb!module load python/3.8.3!
    create virtual environment \verb!python3 -m venv <env_name>!
    activate virtual environment \verb!source <env_name>/bin/activate.csh!
    install DL packages for Efficientdet
\end{itemize}
\begin{verbatim}
pip install tensorflow==2.7.0
pip install lxml>=4.6.1
pip install absl-py>=0.10.0
pip install matplotlib>=3.0.3
pip install numpy>=1.19.4
pip install Pillow>=6.0.0
pip install PyYAML>=5.1
pip install six>=1.15.0
pip install tensorflow-addons>=0.12
pip install tensorflow-hub>=0.11
pip install neural-structured-learning>=1.3.1
pip install tensorflow-model-optimization>=0.5
pip install Cython>=0.29.13
pip install git+https://github.com/cocodataset/cocoapi.git#subdirectory=PythonAPI
\end{verbatim}

% ------------------------------------------------------------------------------
\subsection{Java Jobs}

Jobs that call \tool{java} have a memory overhead, which needs to be taken 
into account when assigning a value to \api{h\_vmem}. Even the most basic 
\tool{java} call, \texttt{java -Xmx1G -version}, will need to have,
\texttt{-l h\_vmem=5G}, with the 4-GB difference representing the memory overhead. 
Note that this memory overhead grows proportionally with the value of
\texttt{-Xmx}. To give you an idea, when \texttt{-Xmx} has a value of 100G,
\api{h\_vmem} has to be at least 106G; for 200G, at least 211G; for 300G, at least 314G.

% TODO: add a MARF Java job

% ------------------------------------------------------------------------------
\subsection{Scheduling On The GPU Nodes}

The primary cluster has two GPU nodes, each with six Tesla (CUDA-compatible) P6
cards: each card has 2048 cores and 16GB of RAM. Though note that the P6
is mainly a single-precision card, so unless you need the GPU double
precision, double-precision calculations will be faster on a CPU node.

Job scripts for the GPU queue differ in that they do not need these
statements:

\begin{verbatim}
#$ -pe smp <threadcount>
#$ -l h_vmem=<memory>G
\end{verbatim}

But do need this statement, which attaches either a single GPU, or, two
GPUs, to the job:

\begin{verbatim}
#$ -l gpu=[1|2]
\end{verbatim}

Single-GPU jobs are granted 5~CPU cores and 80GB of system memory, and
dual-GPU jobs are granted 10~CPU cores and 160GB of system memory. A
total of \emph{four} GPUs can be actively attached to any one user at any given
time.

Once that your job script is ready, you can submit it to the GPU queue
with:

\begin{verbatim}
qsub -q g.q ./<myscript>.sh
\end{verbatim}

And you can query \tool{nvidia-smi} on the node that is running your job with:

\begin{verbatim}
ssh <username>@speed[-05|-17] nvidia-smi
\end{verbatim}

Status of the GPU queue can be queried with:

\begin{verbatim}
qstat -f -u "*" -q g.q
\end{verbatim}

\textbf{Very important note} regarding TensorFlow and PyTorch: 
if you are planning to run TensorFlow and/or PyTorch multi-GPU jobs, 
do not use the \api{tf.distribute} and/or\\
\api{torch.nn.DataParallel} 
functions, as they will crash the compute node (100\% certainty). 
This appears to be the current hardware's architecture's defect.
%
The workaround is to either
% TODO: Need to link to that example
manually effect GPU parallelisation (TensorFlow has an example on how to
do this), or to run on a single GPU.

\vspace{10pt}
\noindent
\textbf{Important}
\vspace{10pt}

Users without permission to use the GPU nodes can submit jobs to the \texttt{g.q}
queue but those jobs will hang and never run.

There are two GPUs in both \texttt{speed-05} and \texttt{speed-17}, and one 
in \texttt{speed-19}. Their availability is seen with, \texttt{qstat -F g}
(note the capital): 

\small
\begin{verbatim}
queuename                      qtype resv/used/tot. load_avg arch          states
---------------------------------------------------------------------------------
...
---------------------------------------------------------------------------------
g.q@speed-05.encs.concordia.ca BIP   0/0/32         0.04     lx-amd64
        hc:gpu=6
---------------------------------------------------------------------------------
g.q@speed-17.encs.concordia.ca BIP   0/0/32         0.01     lx-amd64
        hc:gpu=6
---------------------------------------------------------------------------------
...
---------------------------------------------------------------------------------
s.q@speed-19.encs.concordia.ca BIP   0/32/32        32.37    lx-amd64
        hc:gpu=1
---------------------------------------------------------------------------------
etc. 
\end{verbatim}
\normalsize

This status demonstrates that all five are available (i.e., have not been 
requested as resources). To specifically request a GPU node, add,
\texttt{-l g=[\#GPUs]}, to your \tool{qsub} (statement/script) or
\tool{qlogin} (statement) request. For example,
\texttt{qsub -l h\_vmem=1G -l g=1 ./count.sh}. You 
will see that this job has been assigned to one of the GPU nodes:

\small
\begin{verbatim}
queuename                      qtype resv/used/tot. load_avg arch          states
--------------------------------------------------------------------------------- 
g.q@speed-05.encs.concordia.ca BIP 0/0/32 0.01 lx-amd64  hc:gpu=6 
--------------------------------------------------------------------------------- 
g.q@speed-17.encs.concordia.ca BIP 0/0/32 0.01 lx-amd64  hc:gpu=6 
--------------------------------------------------------------------------------- 
s.q@speed-19.encs.concordia.ca BIP 0/1/32 0.04 lx-amd64  hc:gpu=0 (haff=1.000000) 
       538 100.00000 count.sh   sbunnell     r     03/07/2019 02:39:39     1
---------------------------------------------------------------------------------
etc. 
\end{verbatim}
\normalsize

And that there are no more GPUs available on that node (\texttt{hc:gpu=0}). Note
that no more than two GPUs can be requested for any one job. 

% ------------------------------------------------------------------------------
\subsubsection{CUDA}

When calling \tool{CUDA} within job scripts, it is important to create a link to
the desired \tool{CUDA} libraries and set the runtime link path to the same libraries. 
For example, to use the \texttt{cuda-11.5} libraries, specify the following in 
your Makefile.

\begin{verbatim}
-L/encs/pkg/cuda-11.5/root/lib64 -Wl,-rpath,/encs/pkg/cuda-11.5/root/lib64
\end{verbatim}

In your job script, specify the version of \texttt{gcc} to use prior to calling 
cuda. For example: 
   \texttt{module load gcc/8.4}
or
   \texttt{module load gcc/9.3}

% ------------------------------------------------------------------------------
\subsubsection{Special Notes for sending CUDA jobs to the GPU Queue}

It is not possible to create a \texttt{qlogin} session on to a node in the 
\textbf{GPU Queue} (\texttt{g.q}). As direct logins to these nodes is not 
available, jobs must be submitted to the \textbf{GPU Queue} in order to compile 
and link.

We have several versions of CUDA installed in:
\begin{verbatim}
/encs/pkg/cuda-11.5/root/
/encs/pkg/cuda-10.2/root/
/encs/pkg/cuda-9.2/root
\end{verbatim}

For CUDA to compile properly for the GPU queue, edit your Makefile 
replacing \option{\/usr\/local\/cuda} with one of the above.


% scheduler-job-examples includes: 
% 2.12 Sample Job Script: fluent
% 2.13 Example Job Script: EfficientDet
% 2.14 Java Jobs
% 2.15 Scheduling on the GPU Nodes
%  	2.15.1 P6 on Multi-GPU, Multi-Node
% 	2.15.2 CUDA
%   2.15.3 Special Notes for Sending CUDA Jobs to the GPU Queue
%   2.15.4 OpenISS Examples
% 2.16 Singularity Containers

% ------------------------------------------------------------------------------
%						3 Conclusion
% ------------------------------------------------------------------------------
\section{Conclusion}
\label{sect:conclusion}

The cluster operates on a ``first-come, first-served'' basis until it reaches full capacity.
After that, job positions in the queue are determined based on past usage.
The scheduler does attempt to fill gaps, so occasionally, a single-core job with lower priority 
may be scheduled before a multi-core job with higher priority.

% -------------- 3.1 Important Limitations --------------------
% -------------------------------------------------------------
\subsection{Important Limitations}
\label{sect:limitations}

While Speed is a powerful tool, it is essential to recognize its limitations to use it effectively:

\begin{itemize}
	\item New users are limited to a total of 32 cores. If you need more cores temporarily 
	(up to 192 cores or six jobs of 32 cores each), please contact \url{rt-ex-hpc@encs.concordia.ca}.

	\item Batch job sessions can run for a maximum of one week. 
	Interactive jobs are limited to 24 hours see \xs{sect:interactive-jobs}.

	\item Scripts can live in your NFS-provided home directory, but substantial data 
	should be stored in your cluster-specific directory (located at \verb+/speed-scratch/<ENCSusername>/+).

	NFS is suitable for short-term activities but not for long-term operations.
	Data that a job will read multiple times should be copied at the start to the scratch disk of a compute node using
	\api{\$TMPDIR} (and possibly \api{\$SLURM\_SUBMIT\_DIR}). 
	Intermediate job data should be produced in \api{\$TMPDIR}, and once a job is near completion,
	this data should be copied to your NFS-mounted home directory (or other NFS-mounted space).
	In other words, IO-intensive operations should be performed locally whenever possible, 
	reserving network activity for the start and end of jobs.

	\item Your current resource allocation is based on past usage, 
	which considers approximately one week's worth of past wall clock time 
	(time spent on the node(s)) and compute activity (on the node(s)).

	\item Jobs must always be run within the scheduler's system. Repeat offenders who 
	run jobs outside the scheduler risk losing cluster access.
\end{itemize}

% -------------- 3.2 Tips/Tricks ------------------------------
% -------------------------------------------------------------
% TMP scheduler-specific section
% TODO: delete the file since it's not needed
% % ------------------------------------------------------------------------------
\subsection{Tips/Tricks}
\label{sect:tips}

\begin{itemize}
\item
Files/scripts must have Linux line breaks in them (not Windows ones).
\item
Use \tool{rsync}, not \tool{scp}, when moving data around. 
\item
If you are going to move many many files between NFS-mounted storage and the 
cluster, \tool{tar} everything up first. 
\item
If you intend to use a different shell (e.g., \tool{bash}~\cite{aosa-book-vol1-bash}),
you will need to source a different scheduler file, and will need to 
change the shell declaration in your script(s).
\item
The load displayed in \tool{qstat} by default is \api{np\_load}, which is
load/\#cores. That means that a load of, ``1'', which represents a fully active 
core, is displayed as $0.03$ on the node in question, as there are 32 cores 
on a node. To display load ``as is'' (such that a node with a fully active 
core displays a load of approximately $1.00$), add the following to your
\file{.tcshrc} file: \texttt{setenv SGE\_LOAD\_AVG load\_avg}

\item
Try to request resources that closely match what your job will use: 
requesting many more cores or much more memory than will be needed makes a 
job more difficult to schedule when resources are scarce.

\item
E-mail, \texttt{rt-ex-hpc AT encs.concordia.ca}, with any concerns/questions.
\end{itemize}

\subsection{Tips/Tricks}
\label{sect:tips}

\begin{itemize}
	\item Ensure that files and scripts have Linux line breaks.
	Use the \tool{file} command to verify and \tool{dos2unix} to convert if necessary.

	\item Use \tool{rsync}, not \tool{scp} for copying or moving large amounts of data.
	
	\item Before transferring a large number of files between NFS-mounted storage and 
	the cluster, compress the files into a \tool{tar} archive.

	\item If you plan to use a different shell (e.g., \tool{bash}~\cite{aosa-book-vol1-bash}), 
	change the shell declaration at the beginning of your script(s).

	\item Request resources (cores, memory) that closely match the actual needs of your job.
	Requesting significantly more than necessary can make your job harder to schedule when resources are limited.

	\item For any concerns or questions, email \texttt{rt-ex-hpc AT encs.concordia.ca}
\end{itemize}

% -------------- 3.3 Use Cases --------------------------------
% -------------------------------------------------------------
\subsection{Use Cases}
\label{sect:cases}

\begin{itemize}
	\item HPC Committee's initial batch about 6 students (end of 2019):
	\begin{itemize}
		\item 10000 iterations job in Fluent finished in $<26$ hours vs. 46 hours in Calcul Quebec
	\end{itemize}

	\item NAG's MAC spoofer analyzer~\cite{mac-spoofer-analyzer-intro-c3s2e2014,mac-spoofer-analyzer-detail-fps2014},
	such as \url{https://github.com/smokhov/atsm/tree/master/examples/flucid}
	\begin{itemize}
		\item compilation of forensic computing reasoning cases about false or true positives of hardware address spoofing in the labs
	\end{itemize}

	\item S4 LAB/GIPSY R\&D Group's:
	\begin{itemize}
		\item MARFCAT and MARFPCAT (OSS signal processing and machine learning tools for 
		vulnerable and weak code analysis and network packet capture
		analysis)~\cite{marfcat-nlp-ai2014,marfcat-sate2010-nist,fingerprinting-mal-traffic}
		\item Web service data conversion and analysis
		\item {\flucid} encoders (translation of large log data into {\flucid}~\cite{mokhov-phd-thesis-2013} for forensic analysis)
		\item Genomic alignment exercises
	\end{itemize}

	\item \textbf{Best Paper award}, \bibentry{job-failure-prediction-compsysarch2024}

	% RT521027
	\item \bibentry{unsteady-wake-ouedraogo_essel_2023}
	\item \bibentry{effects-reynolds-ouedraogo_essel_2024}
 	\item \bibentry{nozzle-effects-APS_2024}
	\item \bibentry{effects-reynolds-APS-ouedraogo_essel_2024}
 
	\item \bibentry{oi-containers-poster-siggraph2023}
 
	\item \bibentry{Gopal2024Sep}
	\item \bibentry{Gopal2023Mob}
	% the next one is not visible (it produces an error)
	%\item \bibentry{roof-mounted-vawt-2023}
	\item \bibentry{root-mounted-vawt-corner-2023}
	\item \bibentry{cfd-modeling-turbine-2023}
	\item \bibentry{small-vaxis-turbine-corner-2022}
	\item \bibentry{cfd-vaxis-turbine-wake-2022}
	\item \bibentry{numerical-turbulence-vawt-2021}
	\item \bibentry{niksirat2020}

	\item The work ``\bibentry{lai-haotao-mcthesis19}'' using TensorFlow and Keras on OpenISS
	adjusted to run on Speed based on the repositories:
	\begin{itemize}
		\item \bibentry{openiss-reid-tfk} and
		\item \bibentry{openiss-yolov3}
	\end{itemize}
	and theirs forks by the team.
\end{itemize}

% ------------------------------------------------------------------------------
\appendix

% ------------------------------------------------------------------------------
%						A History 
% ------------------------------------------------------------------------------
\section{History}

% A.1 Acknowledgments
% -------------------------------------------------------------
\subsection{Acknowledgments}
\label{sect:acks}

\begin{itemize}
	\item 
The first 6 to 6.5 versions of this manual and early UGE job script samples,
Singularity testing and user support were produced/done by Dr.~Scott Bunnell
during his time at Concordia as a part of the NAG/HPC group. We thank
him for his contributions.
	\item 
The HTML version with devcontainer support was contributed by Anh H Nguyen.
	\item 
Dr.~Tariq Daradkeh, was our IT Instructional Specialist from August 2022 to September 2023;
working on the scheduler, scheduling research, end user support, and integration of
examples, such as YOLOv3 in \xs{sect:openiss-yolov3} and other tasks. We have a continued
collaboration on HPC/scheduling research (see~\cite{job-failure-prediction-compsysarch2024}).
\end{itemize}

% A.2 Migration from UGE to SLURM
% -------------------------------------------------------------
\subsection{Migration from UGE to SLURM}
\label{appdx:uge-to-slurm}

For long term users who started off with Grid Engine here are some resources
to make a transition and mapping to the job submission process.

\begin{itemize}
\item
Queues are called ``partitions'' in SLURM. Our mapping from the GE queues
to SLURM partitions is as follows:
\begin{verbatim}
GE  => SLURM
s.q    ps
g.q    pg
a.q    pa
\end{verbatim}
We also have a new partition \texttt{pt} that covers SPEED2 nodes,
which previously did not exist.

\item
Commands and command options mappings are found in \xf{fig:rosetta-mappings} from\\
\url{https://slurm.schedmd.com/rosetta.pdf}\\
\url{https://slurm.schedmd.com/pdfs/summary.pdf}\\
Other related helpful resources from similar organizations who either used
SLURM for a while or also transitioned to it:\\
\url{https://docs.alliancecan.ca/wiki/Running_jobs}\\
\url{https://www.depts.ttu.edu/hpcc/userguides/general_guides/Conversion_Table_1.pdf}\\
\url{https://docs.mpcdf.mpg.de/doc/computing/clusters/aux/migration-from-sge-to-slurm}

\begin{figure}[htpb]
\includegraphics[width=\columnwidth]{images/rosetta-mapping}
\caption{Rosetta Mappings of Scheduler Commands from SchedMD}
\label{fig:rosetta-mappings}
\end{figure}

\item
\noindent
\textbf{NOTE:} If you have used UGE commands in the past you probably still have these
lines there; \textbf{they should now be removed}, as they have no use in SLURM and
will start giving ``command not found'' errors on login when the software is removed:

csh/\tool{tcsh}: sample \file{.tcshrc} file:
\begin{verbatim}
# Speed environment set up 
if ($HOSTNAME == speed-submit.encs.concordia.ca) then
   source /local/pkg/uge-8.6.3/root/default/common/settings.csh
endif
\end{verbatim}

Bourne shell/\tool{bash}: sample \file{.bashrc} file:
\begin{verbatim}
# Speed environment set up 
if [ $HOSTNAME = "speed-submit.encs.concordia.ca" ]; then
    . /local/pkg/uge-8.6.3/root/default/common/settings.sh
    printenv ORGANIZATION | grep -qw ENCS || . /encs/Share/bash/profile
fi
\end{verbatim}

\textbf{IMPORTANT NOTE:} you will need to either log out and back in, or execute a new shell, 
for the environment changes in the updated \file{.tcshrc} or \file{.bashrc} file to be applied.

\end{itemize}

% A.3 Phases
% -------------------------------------------------------------
\subsection{Phases}
\label{sect:phases}

Brief summary of Speed evolution phases.

% ------------------------------------------------------------------------------
\subsubsection{Phase 4}

Phase 4 had 7 SuperMicro servers with 4x A100 80GB GPUs each added,
dubbed as ``SPEED2''. We also moved from Grid Engine to SLURM.

% ------------------------------------------------------------------------------
\subsubsection{Phase 3}

Phase 3 had 4 vidpro nodes added from Dr.~Amer totalling 6x P6 and 6x V100
GPUs added.

% ------------------------------------------------------------------------------
\subsubsection{Phase 2}

Phase 2 saw 6x NVIDIA Tesla P6 added and 8x more compute nodes.
The P6s replaced 4x of FirePro S7150.

% ------------------------------------------------------------------------------
\subsubsection{Phase 1}

Phase 1 of Speed was of the following configuration:

\begin{itemize}
\item
Sixteen, 32-core nodes, each with 512~GB of memory and approximately 1~TB of volatile-scratch disk space. 
\item
Five AMD FirePro S7150 GPUs, with 8~GB of memory (compatible with the Direct X, OpenGL, OpenCL, and Vulkan APIs). 
\end{itemize}

% ------------------------------------------------------------------------------
%						B Frequently Asked Questions 
% ------------------------------------------------------------------------------
% TMP scheduler-specific section
% ------------------------------------------------------------------------------
\section{Frequently Asked Questions}
\label{sect:faqs}

% ------------------------------------------------------------------------------
\subsection{Where do I learn about Linux?}

All Speed users are expected to have a basic understanding of Linux and its commonly used commands.

% ------------------------------------------------------------------------------
\subsubsection*{Software Carpentry}

Software Carpentry provides free resources to learn software, including a workshop on the Unix shell.
\url{https://software-carpentry.org/lessons/} 

% ------------------------------------------------------------------------------
\subsubsection*{Udemy}

There are a number of Udemy courses, including free ones, that will assist 
you in learning Linux. Active Concordia faculty, staff and students have 
access to Udemy courses such as \textbf{Linux Mastery: Master the Linux 
Command Line in 11.5 Hours} is a good starting point for beginners. Visit
\url{https://www.concordia.ca/it/services/udemy.html} to learn how Concordians 
may access Udemy.

% ------------------------------------------------------------------------------
\subsection{How to use the ``bash shell'' on Speed?}

This section describes how to use the ``bash shell'' on Speed. Review
\xs{sect:envsetup} to ensure that your bash environment is set up.

% ------------------------------------------------------------------------------
\subsubsection{How do I set bash as my login shell?}

In order to set your login shell to bash on Speed, your login shell on all GCS servers must be changed to bash.
To make this change, create a ticket with the Service Desk (or email help at concordia.ca) to request that bash become your default login shell for your ENCS user account on all GCS servers.

% ------------------------------------------------------------------------------
\subsubsection{How do I move into a bash shell on Speed?}

To move to the bash shell, type \textbf{bash} at the command prompt.
For example:
\begin{verbatim}
	[speed-submit] [/home/a/a_user] > bash
	bash-4.4$ echo $0
	bash
\end{verbatim}	

Note how the command prompt changed from \verb![speed-submit] [/home/a/a_user] >! to \verb!bash-4.4$! after entering the bash shell.

% ------------------------------------------------------------------------------
\subsubsection{How do I run scripts written in bash on Speed?}

To execute bash scripts on Speed:
\begin{enumerate}
	\item 
Ensure that the shebang of your bash job script is \verb!#!/encs/bin/bash!
	\item 
Use the qsub command to submit your job script to the scheduler.
\end{enumerate}

The Speed GitHub contains a sample \href{https://github.com/NAG-DevOps/speed-hpc/blob/master/src/bash.sh}{bash job script}.  

% ------------------------------------------------------------------------------
\subsection{How to resolve ``Disk quota exceeded'' errors?}

% ------------------------------------------------------------------------------
\subsubsection{Probable Cause}

The \texttt{``Disk quota exceeded''} Error occurs when your application has run out of disk space to write to. On Speed this error can be returned when:
\begin{enumerate}
	\item
The \texttt{/tmp} directory on the speed node your application is running on is full and cannot be written to.
	\item
Your NFS-provided home is full and cannot be written to.
\end{enumerate}

% ------------------------------------------------------------------------------
\subsubsection{Possible Solutions}

\begin{enumerate}
	\item
Use the \textbf{-cwd} job script option to set the directory that the job 
script is submitted from the \texttt{job working directory}. The
\texttt{job working directory} is the directory that the job will write output files in.
 	\item
The use local disk space is generally recommended for IO intensive operations. However, as the size of \texttt{/tmp} on speed nodes 
is \texttt{1GB} it can be necessary for scripts to store temporary data 
elsewhere. 
Review the documentation for each module called within your script to 
determine how to set working directories for that application. 
The basic steps for this solution are:
\begin{itemize}
	\item
	Review the documentation on how to set working directories for 
	each module called by the job script.
	\item
	Create a working directory in speed-scratch for output files. 
	For example, this command will create a subdirectory called \textbf{output}
	 in your \verb!speed-scratch! directory:
	 \begin{verbatim}
		mkdir -m 750 /speed-scratch/$USER/output
	 \end{verbatim}
	\item
	To create a subdirectory for recovery files:
	\begin{verbatim}
		mkdir -m 750 /speed-scratch/$USER/recovery
	\end{verbatim}
	\item
	Update the job script to write output to the subdirectories you created in your \verb!speed-scratch! directory, e.g., \verb!/speed-scratch/$USER/output!.
	\end{itemize}
\end{enumerate}
In the above example, \verb!$USER! is an environment variable containing your ENCS username.

% ------------------------------------------------------------------------------
\subsubsection{Example of setting working directories for \tool{COMSOL}}

\begin{itemize}
	\item 
	Create directories for recovery, temporary, and configuration files. 
	For example, to create these directories for your GCS ENCS user account:
	\begin{verbatim}
	mkdir -m 750 -p /speed-scratch/$USER/comsol/{recovery,tmp,config}
	\end{verbatim}
	\item
	Add the following command switches to the COMSOL command to use the 
	directories created above:
	\begin{verbatim} 
	-recoverydir /speed-scratch/$USER/comsol/recovery 
	-tmpdir /speed-scratch/$USER/comsol/tmp
	-configuration/speed-scratch/$USER/comsol/config
	\end{verbatim}
\end{itemize} 
In the above example, \verb!$USER! is an environment variable containing your ENCS username.

% ------------------------------------------------------------------------------
\subsubsection{Example of setting working directories for \tool{Python Modules}}

By default when adding a python module the \texttt{/tmp} directory is set as the temporary repository for files downloads. 
The size of the \texttt{/tmp} directory on \verb!speed-submit! is too small for pytorch.
To add a python module
\begin{itemize}
    \item 	
	Create your own tmp directory in your \verb!speed-scratch! directory
	\begin{verbatim} 
  mkdir /speed-scratch/$USER/tmp
	\end{verbatim}
	\item
  Use the tmp directory you created
	\begin{verbatim} 
  setenv TMPDIR /speed-scratch/$USER/tmp
	\end{verbatim}
    \item
	Attempt the installation of pytorch
\end{itemize}

In the above example, \verb!$USER! is an environment variable containing your ENCS username.

% ------------------------------------------------------------------------------
\subsection{How do I check my job's status?}

When a job with a job id of 1234 is running, the status of that job can be tracked using \verb!`qstat -j 1234`!.
Likewise, if the job is pending, the \verb!`qstat -j 1234`! command will report as to why the job is not scheduled or running.
Once the job has finished, or has been killed, the \textbf{qacct} command must be used to query the job's status, e.g., \verb!`qaact -j [jobid]`!. 

% ------------------------------------------------------------------------------
\subsection{Why is my job pending when nodes are empty?}

% ------------------------------------------------------------------------------
\subsubsection{Disabled nodes}

It is possible that a (or a number of) the Speed nodes are disabled. Nodes are disabled if they require maintenance. 
To verify if Speed nodes are disabled, request the current list of disabled nodes from qstat.

\begin{verbatim}
qstat -f -qs d
queuename                      qtype resv/used/tot. load_avg arch          states
---------------------------------------------------------------------------------
g.q@speed-05.encs.concordia.ca BIP   0/0/32         0.27     lx-amd64      d
---------------------------------------------------------------------------------
s.q@speed-07.encs.concordia.ca BIP   0/0/32         0.01     lx-amd64      d
---------------------------------------------------------------------------------
s.q@speed-10.encs.concordia.ca BIP   0/0/32         0.01     lx-amd64      d
---------------------------------------------------------------------------------
s.q@speed-16.encs.concordia.ca BIP   0/0/32         0.02     lx-amd64      d
---------------------------------------------------------------------------------
s.q@speed-19.encs.concordia.ca BIP   0/0/32         0.03     lx-amd64      d
---------------------------------------------------------------------------------
s.q@speed-24.encs.concordia.ca BIP   0/0/32         0.01     lx-amd64      d
---------------------------------------------------------------------------------
s.q@speed-36.encs.concordia.ca BIP   0/0/32         0.03     lx-amd64      d
\end{verbatim}

Note how the all of the Speed nodes in the above list have a state of \textbf{d}, or disabled.

Your job will run once the maintenance has been completed and the disabled nodes have been enabled.

% ------------------------------------------------------------------------------
\subsubsection{Error in job submit request.}

It is possible that your job is pending, because the job requested resources that are not available within Speed.
To verify why pending job with job id 1234 is not running, execute \verb!`qstat -j 1234`! 
and review the messages in the \textbf{scheduling info:} section.


% ------------------------------------------------------------------------------
%						C Sister Facilities
% ------------------------------------------------------------------------------
\section{Sister Facilities}

Below is a list of resources and facilities similar to Speed at various capacities.
Depending on your research group and needs, they might be available to you. They
are not managed by HPC/NAG of AITS, so contact their respective representatives.

\begin{itemize}
\item
\texttt{computation.encs} CPU only 3-machine cluster running longer jobs
without a scheduler at the moment
\item
\texttt{apini.encs} cluster for teaching and MPI programming (see the corresponding
course in CSSE)
\item
Computer Science and Software Engineering (CSSE) Virya GPU Cluster. For CSSE 
members only. The cluster has 4 nodes with total of 32 NVIDIA GPUs (a mix of
V100s and A100s). To request access send email to \texttt{virya.help AT concordia.ca}.
\item
Dr. Maria Amer's VidPro group's nodes in Speed (-01, -03, -25, -27) with additional V100 and P6 GPUs.
\item
There are various Lambda Labs other GPU servers and like computers
acquired by individual researchers; if you are member of their
research group, contact them directly. These resources are not
managed by us.
\begin{itemize}
\item
Dr. Amin Hammad's \texttt{construction.encs} Lambda Labs station
\item
Dr. Hassan Rivaz's \texttt{impactlab.encs} Lambda Labs station
\item
Dr. Nizar Bouguila's \texttt{xailab.encs} Lambda Labs station
\item
Dr. Roch Glitho's \texttt{femto.encs} server
\item
Dr. Maria Amer's \texttt{venom.encs} Lambda Labs station
\item
Dr. Leon Wang's \texttt{guerrera.encs} DGX station
\end{itemize}
\item
Dr. Ivan Contreras' servers (managed by AITS)
\item
If you are a member of School of Health (formerly PERFORM Center),
you may have access to their local 
\href
{https://perform-wiki.concordia.ca/mediawiki/index.php/HPC_Cluster}
{PERFORM's High Performance Computing (HPC) Cluster}.
Contact Thomas Beaudry for details and how to obtain access.
\item
All Concordia students have access to the Library's small
\href
{https://library.concordia.ca/technology/sandbox/}
{Technology Sandbox}
testing cluster that also runs Slurm. Email \texttt{sean.cooney AT concordia.ca} for details.
\item
Digital Research Alliance Canada (Compute Canada / Calcul Quebec),\\
\url{https://alliancecan.ca/}. Follow
\href
{https://alliancecan.ca/en/services/advanced-research-computing/account-management/apply-account}
{this link}
on the information how to obtain access (students need to be sponsored
by their supervising faculty members, who should create accounts
first). Their SLURM examples are here: \url{https://docs.alliancecan.ca/wiki/Running_jobs}

\end{itemize}

% ------------------------------------------------------------------------------
%						Software List 
% ------------------------------------------------------------------------------
% -----------------------------------------------------------------------------
% ./generate-software-list.sh
\section{Software Installed On Speed}
\label{sect:software-list}

This is a generated section by a script; last updated on \textit{Thu Sep 18 12:20:09 PM EDT 2025}.
We have two major software trees: Scientific Linux 7 (EL7), which is
outgoing, and AlmaLinux 9 (EL9). After major synchronization of software
packages is complete, we will stop maintaining the EL7 tree and
will migrate the remaining nodes to EL9.

Use \option{--constraint=el7} to select EL7-only installed nodes for their
software packages. Conversely, use \option{--constraint=el9} for the EL9-only
software. These options would be used as a part of your job parameters
in either \api{\#SBATCH} or on the command line.

\noindent
\textbf{NOTE:} this list does not include packages installed directly on the OS (yet).
% -----------------------------------------------------------------------------
\subsection{EL9 (/encs/pkg)}
\label{sect:software-el9}

\scriptsize
\begin{multicols}{3}
\begin{itemize}
\item \verb|a2ps-4.14|
\item \verb|abaqus-2021|
\item \verb|abaqus-2023|
\item \verb|abaqus-2025|
\item \verb|acl-10.1.express|
\item \verb|ADS-2020u1|
\item \verb|alpine-2.24|
\item \verb|alpine-2.25|
\item \verb|anaconda3-2023.03|
\item \verb|ansys-2021R1|
\item \verb|ansys-2021R2|
\item \verb|ansys-2022R1|
\item \verb|ansys-2022R2|
\item \verb|ansys-2023R1|
\item \verb|ansys-2023R2|
\item \verb|ansys-2024R2|
\item \verb|ansys-2025R1|
\item \verb|ansys-2025R2|
\item \verb|ant-1.10.11|
\item \verb|ant-1.10.2|
\item \verb|ANTs-2.3.5|
\item \verb|arduino-1.6.8|
\item \verb|ArgoUML-0.34|
\item \verb|aspectj-1.8.6|
\item \verb|aspell-0.60.8|
\item \verb|automake-1.15.1|
\item \verb|automake-1.17|
\item \verb|bash-4.4|
\item \verb|bazel-0.2.0|
\item \verb|bison-3.7.2|
\item \verb|boost-1.73.0|
\item \verb|buddy-2.4|
\item \verb|build-openocd|
\item \verb|camlp5-6.14|
\item \verb|Check-0.15.2|
\item \verb|cmake-3.18.4|
\item \verb|codelite-18.0.0|
\item \verb|compat-glibc|
\item \verb|comsol-6.0|
\item \verb|comsol-6.1|
\item \verb|comsol-6.2|
\item \verb|comsol-6.3|
\item \verb|cplex-20.1.0|
\item \verb|cplex-22.1.1|
\item \verb|CST-2019|
\item \verb|CST-2020|
\item \verb|cuda-10.2|
\item \verb|cuda-11.5|
\item \verb|cuda-11.8|
\item \verb|cuda-12.8|
\item \verb|cups-2.3.3|
\item \verb|curl-7.86.0|
\item \verb|DbVisualizer-24.1.5|
\item \verb|EasyBuild|
\item \verb|EasyBuild.old|
\item \verb|eclipse-jee.202506|
\item \verb|emacs-27.2|
\item \verb|expect-5.45.4|
\item \verb|feko-2018.2|
\item \verb|ffmpeg-4.1.3|
\item \verb|firefox-102.11.0|
\item \verb|firefox-102.12.0|
\item \verb|firefox-102.13.0|
\item \verb|firefox-102.14.0|
\item \verb|firefox-102.15.0|
\item \verb|firefox-102.15.1|
\item \verb|firefox-115.10.0|
\item \verb|firefox-115.2.1|
\item \verb|firefox-115.3.0|
\item \verb|firefox-91.10.0|
\item \verb|firefox-91.11.0|
\item \verb|firefox-91.8.0|
\item \verb|firefox-91.9.0|
\item \verb|firefox-91.9.1|
\item \verb|firefox_french-102.11.0|
\item \verb|firefox_french-102.12.0|
\item \verb|firefox_french-102.13.0|
\item \verb|firefox_french-102.14.0|
\item \verb|firefox_french-102.15.0|
\item \verb|firefox_french-102.15.1|
\item \verb|firefox_french-115.10.0|
\item \verb|firefox_french-115.2.1|
\item \verb|firefox_french-115.3.0|
\item \verb|firefox_french-91.10.0|
\item \verb|firefox_french-91.11.0|
\item \verb|firefox_french-91.8.0|
\item \verb|firefox_french-91.9.0|
\item \verb|firefox_french-91.9.1|
\item \verb|gcc-11.3.0|
\item \verb|gcc-12.2.0|
\item \verb|gcc-4.9.2|
\item \verb|gcc-5.4.0|
\item \verb|gcc-7.3.0|
\item \verb|gcc-arm-11.2.2022.02|
\item \verb|ghostscript-8.50|
\item \verb|ghostscript-9.50|
\item \verb|glibc-2.28|
\item \verb|gmp-4.3.2|
\item \verb|go-1.12|
\item \verb|go-1.15.6|
\item \verb|go-1.19.3|
\item \verb|gperf-3.0.4|
\item \verb|gurobi-10.0.1|
\item \verb|gurobi-9.1.0|
\item \verb|gv-3.7.4|
\item \verb|httpd-2.4.55|
\item \verb|httpd-2.4.57|
\item \verb|httpd-2.4.63|
\item \verb|httpd-current|
\item \verb|http-parser-2.9.4|
\item \verb|hwloc-2.8.0|
\item \verb|jansson-2.14|
\item \verb|jdk-17|
\item \verb|jdk-17.0.2|
\item \verb|jdk-19|
\item \verb|jdk-19.0.2|
\item \verb|jdk-24|
\item \verb|jdk-24.0.2|
\item \verb|jdk_32b-8u231|
\item \verb|jdk-6|
\item \verb|jdk_64b-6u45|
\item \verb|jdk_64b-7u80|
\item \verb|jdk_64b-8u231|
\item \verb|jdk-7|
\item \verb|jdk-8|
\item \verb|jdk-8_32b|
\item \verb|json-c-0.16|
\item \verb|kicad-4.0.1|
\item \verb|libevent-2.1.12|
\item \verb|libjwt-1.15.2|
\item \verb|LibreOffice-7.1.8|
\item \verb|LibreOffice-7.4.7|
\item \verb|libxml2-2.9.4|
\item \verb|libyaml-0.2.5|
\item \verb|libzip-1.5.1|
\item \verb|lynx-2.8.9|
\item \verb|lz4-1.9.4|
\item \verb|matlab-R2022a|
\item \verb|matlab-R2022b|
\item \verb|matlab-R2023a|
\item \verb|matlab-R2023b|
\item \verb|matlab-R2024a|
\item \verb|matlab-R2024b|
\item \verb|mesa-19.0.3|
\item \verb|modules-3.2.10|
\item \verb|modules-5.3.1|
\item \verb|modules-current|
\item \verb|mpack-1.6|
\item \verb|mpfr-2.4.2|
\item \verb|mpfr-3.1.6|
\item \verb|mpich-4.1.2|
\item \verb|MRtrix-3.0.3|
\item \verb|mysql-5.1.66|
\item \verb|mysql-5.6.43|
\item \verb|mysql-5.7.43|
\item \verb|mysql-8.0.31|
\item \verb|nagtools-2.1.10|
\item \verb|nagtools-2.1.3|
\item \verb|nagtools-2.1.4|
\item \verb|nagtools-2.1.5|
\item \verb|nagtools-2.1.6|
\item \verb|nagtools-2.1.7|
\item \verb|nagtools-2.1.8|
\item \verb|nagtools-2.1.9|
\item \verb|nano-6.2|
\item \verb|nasm-2.15.05|
\item \verb|ncurses-6.4|
\item \verb|net-snmp-5.4.1|
\item \verb|net-snmp-5.9.1|
\item \verb|nmh-1.7.1|
\item \verb|node-v12.18.0|
\item \verb|node-v16.13.0|
\item \verb|nvtop-3.0.1|
\item \verb|ocaml-4.01.0|
\item \verb|oidentd-3.1.0|
\item \verb|oniguruma-6.9.5|
\item \verb|OpenFOAM-11.0|
\item \verb|OpenFOAM-12.0|
\item \verb|OpenFOAM-2.4.0|
\item \verb|OpenFOAM-8.0|
\item \verb|OpenFOAM-v2012|
\item \verb|OpenFOAM-v2306|
\item \verb|openmpi-4.1.6|
\item \verb|openocd|
\item \verb|openocd-0.11.0|
\item \verb|openpmix-5.0.1|
\item \verb|openssl-1.0.2.current|
\item \verb|openssl-1.0.2u|
\item \verb|openssl-1.1.1.current|
\item \verb|openssl-1.1.1n|
\item \verb|openssl-3.0.12|
\item \verb|openssl-3.0.current|
\item \verb|oracle-19c|
\item \verb|os-overrides-1.0|
\item \verb|ParaView-5.11.2|
\item \verb|perl-5.30.3|
\item \verb|pgadmin4-7.6|
\item \verb|php-7.4.33|
\item \verb|postgresql-12|
\item \verb|postgresql-12.3|
\item \verb|postgresql-16|
\item \verb|postgresql-16.9|
\item \verb|postgresql-8.3.18|
\item \verb|print-utils-1.0|
\item \verb|python-2.7.11|
\item \verb|python-3.10.13|
\item \verb|python-3.10.6|
\item \verb|python-3.11.0|
\item \verb|python-3.11.5|
\item \verb|python-3.11.6|
\item \verb|python-3.12.0|
\item \verb|python-3.6.15|
\item \verb|python-3.7.7|
\item \verb|python-3.8.18|
\item \verb|python-3.8.9|
\item \verb|python-3.9.1|
\item \verb|python-3.9.18|
\item \verb|qt-5.14.2|
\item \verb|qt-5.15.10|
\item \verb|qt-5.9|
\item \verb|quota-1.3|
\item \verb|redis-8.0.0|
\item \verb|redis-8.2.1|
\item \verb|ruby-2.7.1|
\item \verb|singularity-3.10.4|
\item \verb|singularity_containers|
\item \verb|sqlite-3.44.2|
\item \verb|StarCCM-15.04.010|
\item \verb|StarCCM-2502.0001_008|
\item \verb|stealfile-1.0|
\item \verb|STSLib-1.0|
\item \verb|tcl-8.4.16|
\item \verb|tcl-8.5.14|
\item \verb|tcl-8.6.13|
\item \verb|tcsh-6.18.01|
\item \verb|tecplot360-2023R1|
\item \verb|texinfo-5.2|
\item \verb|texlive-20220405|
\item \verb|texlive-20230324|
\item \verb|texlive-current|
\item \verb|thunderbird-102.11.1|
\item \verb|thunderbird-102.12.0|
\item \verb|thunderbird-115.1.0|
\item \verb|thunderbird-115.10.2|
\item \verb|thunderbird_french-102.11.1|
\item \verb|thunderbird_french-102.12.0|
\item \verb|thunderbird_french-115.1.0|
\item \verb|thunderbird_french-115.10.2|
\item \verb|tix-8.1.4|
\item \verb|tk-8.4.16|
\item \verb|tk-8.5.14|
\item \verb|tk-8.6.13|
\item \verb|tmux-3.2a|
\item \verb|tomcat-10.1.35|
\item \verb|tomcat10-current|
\item \verb|tomcat-9.0.79|
\item \verb|tomcat-9.0.98|
\item \verb|tomcat9-current|
\item \verb|tomcat_connectors-1.2.46|
\item \verb|user-utils-1.0|
\item \verb|vacation-sendmail-8.16.1|
\item \verb|wxWidgets-3.0.2|
\item \verb|wxWidgets-3.2.8|
\item \verb|xemacs-21.4.24|
\end{itemize}
\end{multicols}
\normalsize

% -----------------------------------------------------------------------------
\subsection{EB (EL9)}
\label{sect:software-eb}

\scriptsize
\begin{multicols}{3}
\begin{itemize}
\item \verb|ant/1.10.11-Java-11|
\item \verb|ant/1.10.12-Java-17|
\item \verb|ant/1.10.14-Java-11|
\item \verb|ASE/3.23.0-gfbf-2024a|
\item \verb|ASE/3.23.0-iimkl-2023a|
\item \verb|ASE/3.24.0-gfbf-2024a|
\item \verb|Autoconf/2.71-GCCcore-12.3.0|
\item \verb|Autoconf/2.72-GCCcore-13.3.0|
\item \verb|Automake/1.16.5-GCCcore-12.3.0|
\item \verb|Automake/1.16.5-GCCcore-13.3.0|
\item \verb|Autotools/20220317-GCCcore-12.3.0|
\item \verb|Autotools/20231222-GCCcore-13.3.0|
\item \verb|binutils/2.36.1|
\item \verb|binutils/2.36.1-GCCcore-10.3.0|
\item \verb|binutils/2.40|
\item \verb|binutils/2.40-GCCcore-12.3.0|
\item \verb|binutils/2.42|
\item \verb|binutils/2.42-GCCcore-13.3.0|
\item \verb|binutils/2.42-GCCcore-14.2.0|
\item \verb|Bison/3.7.6-GCCcore-10.3.0|
\item \verb|Bison/3.8.2|
\item \verb|Bison/3.8.2-GCCcore-12.3.0|
\item \verb|Bison/3.8.2-GCCcore-13.3.0|
\item \verb|Bison/3.8.2-GCCcore-14.2.0|
\item \verb|BLIS/0.9.0-GCC-12.3.0|
\item \verb|BLIS/1.0-GCC-13.3.0|
\item \verb|Boost/1.82.0-GCC-12.3.0|
\item \verb|Boost/1.85.0-GCC-13.3.0|
\item \verb|Brotli/1.0.9-GCCcore-12.3.0|
\item \verb|Brotli/1.1.0-GCCcore-13.3.0|
\item \verb|bzip2/1.0.8-GCCcore-10.3.0|
\item \verb|bzip2/1.0.8-GCCcore-12.3.0|
\item \verb|bzip2/1.0.8-GCCcore-13.3.0|
\item \verb|bzip2/1.0.8-GCCcore-14.2.0|
\item \verb|cairo/1.17.8-GCCcore-12.3.0|
\item \verb|cairo/1.18.0-GCCcore-13.3.0|
\item \verb|Catch2/2.13.10-GCCcore-13.3.0|
\item \verb|Catch2/2.13.9-GCCcore-12.3.0|
\item \verb|cffi/1.15.1-GCCcore-12.3.0|
\item \verb|cffi/1.16.0-GCCcore-13.3.0|
\item \verb|CMake/3.26.3-GCCcore-12.3.0|
\item \verb|CMake/3.29.3-GCCcore-13.3.0|
\item \verb|cppy/1.2.1-GCCcore-12.3.0|
\item \verb|cppy/1.2.1-GCCcore-13.3.0|
\item \verb|cryptography/41.0.1-GCCcore-12.3.0|
\item \verb|cryptography/42.0.8-GCCcore-13.3.0|
\item \verb|cURL/7.76.0-GCCcore-10.3.0|
\item \verb|cURL/7.76.1-GCCcore-10.3.0|
\item \verb|cURL/7.77.0-GCCcore-10.3.0|
\item \verb|cURL/8.0.1-GCCcore-12.3.0|
\item \verb|cURL/8.7.1-GCCcore-13.3.0|
\item \verb|Cython/3.0.10-GCCcore-13.3.0|
\item \verb|Cython/3.0.8-GCCcore-12.3.0|
\item \verb|DBus/1.15.4-GCCcore-12.3.0|
\item \verb|double-conversion/3.3.0-GCCcore-12.3.0|
\item \verb|Doxygen/1.11.0-GCCcore-13.3.0|
\item \verb|Doxygen/1.9.7-GCCcore-12.3.0|
\item \verb|Eigen/3.4.0-GCCcore-12.3.0|
\item \verb|Eigen/3.4.0-GCCcore-13.3.0|
\item \verb|ELPA/2023.05.001-intel-2023a|
\item \verb|ELPA/2024.05.001-foss-2024a|
\item \verb|expat/2.2.9-GCCcore-10.3.0|
\item \verb|expat/2.5.0-GCCcore-12.3.0|
\item \verb|expat/2.6.2-GCCcore-13.3.0|
\item \verb|FFmpeg/6.0-GCCcore-12.3.0|
\item \verb|ffnvcodec/12.0.16.0|
\item \verb|FFTW/3.3.10-GCC-12.3.0|
\item \verb|FFTW/3.3.10-GCC-13.3.0|
\item \verb|FFTW.MPI/3.3.10-gompi-2023a|
\item \verb|FFTW.MPI/3.3.10-gompi-2024a|
\item \verb|Flask/2.3.3-GCCcore-12.3.0|
\item \verb|Flask/3.0.3-GCCcore-13.3.0|
\item \verb|flex/2.6.4|
\item \verb|flex/2.6.4-GCCcore-10.3.0|
\item \verb|flex/2.6.4-GCCcore-12.3.0|
\item \verb|flex/2.6.4-GCCcore-13.3.0|
\item \verb|flex/2.6.4-GCCcore-14.2.0|
\item \verb|FlexiBLAS/3.3.1-GCC-12.3.0|
\item \verb|FlexiBLAS/3.4.4-GCC-13.3.0|
\item \verb|flit/3.9.0-GCCcore-12.3.0|
\item \verb|flit/3.9.0-GCCcore-13.3.0|
\item \verb|fontconfig/2.14.2-GCCcore-12.3.0|
\item \verb|fontconfig/2.15.0-GCCcore-13.3.0|
\item \verb|fonttools/4.53.1-GCCcore-13.3.0|
\item \verb|foss/2023a|
\item \verb|foss/2024a|
\item \verb|freetype/2.13.0-GCCcore-12.3.0|
\item \verb|freetype/2.13.2-GCCcore-13.3.0|
\item \verb|FriBidi/1.0.12-GCCcore-12.3.0|
\item \verb|FriBidi/1.0.15-GCCcore-13.3.0|
\item \verb|FSL/6.0.7.17|
\item \verb|GCC/12.3.0|
\item \verb|GCC/13.3.0|
\item \verb|GCCcore/10.3.0|
\item \verb|GCCcore/12.3.0|
\item \verb|GCCcore/13.2.0|
\item \verb|GCCcore/13.3.0|
\item \verb|GCCcore/14.2.0|
\item \verb|gettext/0.21.1|
\item \verb|gettext/0.21.1-GCCcore-12.3.0|
\item \verb|gettext/0.22.5|
\item \verb|gettext/0.22.5-GCCcore-13.3.0|
\item \verb|gfbf/2023a|
\item \verb|gfbf/2024a|
\item \verb|giflib/5.2.1-GCCcore-12.3.0|
\item \verb|giflib/5.2.1-GCCcore-13.3.0|
\item \verb|git/2.41.0-GCCcore-12.3.0-nodocs|
\item \verb|git/2.45.1-GCCcore-13.3.0|
\item \verb|GLib/2.77.1-GCCcore-12.3.0|
\item \verb|GLib/2.80.4-GCCcore-13.3.0|
\item \verb|GMP/6.3.0-GCCcore-13.3.0|
\item \verb|GObject-Introspection/1.76.1-GCCcore-12.3.0|
\item \verb|GObject-Introspection/1.80.1-GCCcore-13.3.0|
\item \verb|gompi/2023a|
\item \verb|gompi/2024a|
\item \verb|GPAW/24.1.0-intel-2023a|
\item \verb|GPAW/24.6.0-foss-2024a|
\item \verb|GPAW/24.6.0-intel-2023a-ASE-3.23.0|
\item \verb|GPAW/25.1.0-foss-2024a-ASE-3.24.0|
\item \verb|GPAW-setups/24.1.0|
\item \verb|GPAW-setups/24.11.0|
\item \verb|gperf/3.1-GCCcore-10.3.0|
\item \verb|gperf/3.1-GCCcore-12.3.0|
\item \verb|gperf/3.1-GCCcore-13.3.0|
\item \verb|graphite2/1.3.14-GCCcore-12.3.0|
\item \verb|groff/1.22.4-GCCcore-12.3.0|
\item \verb|groff/1.23.0-GCCcore-13.3.0|
\item \verb|gzip/1.12-GCCcore-12.3.0|
\item \verb|gzip/1.13-GCCcore-13.3.0|
\item \verb|HarfBuzz/5.3.1-GCCcore-12.3.0|
\item \verb|HarfBuzz/9.0.0-GCCcore-13.3.0|
\item \verb|hatchling/1.18.0-GCCcore-12.3.0|
\item \verb|hatchling/1.24.2-GCCcore-13.3.0|
\item \verb|HDF5/1.14.0-gompi-2023a|
\item \verb|HDF5/1.14.5-gompi-2024a|
\item \verb|help2man/1.48.3-GCCcore-10.3.0|
\item \verb|help2man/1.49.3-GCCcore-12.3.0|
\item \verb|help2man/1.49.3-GCCcore-13.3.0|
\item \verb|help2man/1.49.3-GCCcore-14.2.0|
\item \verb|hwloc/2.10.0-GCCcore-13.3.0|
\item \verb|hwloc/2.9.1-GCCcore-12.3.0|
\item \verb|hypothesis/6.103.1-GCCcore-13.3.0|
\item \verb|hypothesis/6.82.0-GCCcore-12.3.0|
\item \verb|ICU/73.2-GCCcore-12.3.0|
\item \verb|ICU/75.1-GCCcore-13.3.0|
\item \verb|iimkl/2023a|
\item \verb|iimpi/2023a|
\item \verb|imkl/2023.1.0|
\item \verb|imkl-FFTW/2023.1.0-iimpi-2023a|
\item \verb|impi/2021.9.0-intel-compilers-2023.1.0|
\item \verb|intel/2023a|
\item \verb|intel-compilers/2023.1.0|
\item \verb|intltool/0.51.0-GCCcore-12.3.0|
\item \verb|intltool/0.51.0-GCCcore-13.3.0|
\item \verb|JasPer/4.0.0-GCCcore-12.3.0|
\item \verb|Java/11|
\item \verb|Java/11.0.27|
\item \verb|Java/17|
\item \verb|Java/17.0.15|
\item \verb|jbigkit/2.1-GCCcore-12.3.0|
\item \verb|jbigkit/2.1-GCCcore-13.3.0|
\item \verb|LAME/3.100-GCCcore-12.3.0|
\item \verb|libarchive/3.6.2-GCCcore-12.3.0|
\item \verb|libarchive/3.7.4-GCCcore-13.3.0|
\item \verb|libdeflate/1.18-GCCcore-12.3.0|
\item \verb|libdeflate/1.20-GCCcore-13.3.0|
\item \verb|libdrm/2.4.115-GCCcore-12.3.0|
\item \verb|libdrm/2.4.122-GCCcore-13.3.0|
\item \verb|libevent/2.1.12-GCCcore-12.3.0|
\item \verb|libevent/2.1.12-GCCcore-13.3.0|
\item \verb|libfabric/1.18.0-GCCcore-12.3.0|
\item \verb|libfabric/1.21.0-GCCcore-13.3.0|
\item \verb|libffi/3.4.4-GCCcore-12.3.0|
\item \verb|libffi/3.4.5-GCCcore-13.3.0|
\item \verb|libffi/3.4.5-GCCcore-14.2.0|
\item \verb|libgit2/1.8.1-GCCcore-13.3.0|
\item \verb|libGLU/9.0.3-GCCcore-12.3.0|
\item \verb|libGLU/9.0.3-GCCcore-13.3.0|
\item \verb|libglvnd/1.6.0-GCCcore-12.3.0|
\item \verb|libglvnd/1.7.0-GCCcore-13.3.0|
\item \verb|libiconv/1.17-GCCcore-12.3.0|
\item \verb|libiconv/1.17-GCCcore-13.3.0|
\item \verb|libjpeg-turbo/2.1.5.1-GCCcore-12.3.0|
\item \verb|libjpeg-turbo/3.0.1-GCCcore-13.3.0|
\item \verb|libogg/1.3.5-GCCcore-13.3.0|
\item \verb|libpciaccess/0.17-GCCcore-12.3.0|
\item \verb|libpciaccess/0.18.1-GCCcore-13.3.0|
\item \verb|libpng/1.6.37-GCCcore-10.3.0|
\item \verb|libpng/1.6.39-GCCcore-12.3.0|
\item \verb|libpng/1.6.43-GCCcore-13.3.0|
\item \verb|libreadline/8.1-GCCcore-10.3.0|
\item \verb|libreadline/8.2-GCCcore-12.3.0|
\item \verb|libreadline/8.2-GCCcore-13.3.0|
\item \verb|libreadline/8.2-GCCcore-14.2.0|
\item \verb|LibTIFF/4.5.0-GCCcore-12.3.0|
\item \verb|LibTIFF/4.6.0-GCCcore-13.3.0|
\item \verb|libtool/2.4.7-GCCcore-12.3.0|
\item \verb|libtool/2.4.7-GCCcore-13.3.0|
\item \verb|libunwind/1.6.2-GCCcore-12.3.0|
\item \verb|libunwind/1.8.1-GCCcore-13.3.0|
\item \verb|libvdwxc/0.4.0-foss-2024a|
\item \verb|libwebp/1.3.1-GCCcore-12.3.0|
\item \verb|libwebp/1.4.0-GCCcore-13.3.0|
\item \verb|libxc/6.2.2-GCC-13.3.0|
\item \verb|libxc/6.2.2-intel-compilers-2023.1.0|
\item \verb|libxml2/2.11.4-GCCcore-12.3.0|
\item \verb|libxml2/2.12.7-GCCcore-13.3.0|
\item \verb|libxslt/1.1.42-GCCcore-13.3.0|
\item \verb|libyaml/0.2.5-GCCcore-12.3.0|
\item \verb|libyaml/0.2.5-GCCcore-13.3.0|
\item \verb|lit/18.1.2-GCCcore-12.3.0|
\item \verb|lit/18.1.8-GCCcore-13.3.0|
\item \verb|LittleCMS/2.15-GCCcore-12.3.0|
\item \verb|LittleCMS/2.16-GCCcore-13.3.0|
\item \verb|LLVM/16.0.6-GCCcore-12.3.0|
\item \verb|LLVM/18.1.8-GCCcore-13.3.0|
\item \verb|LLVM/18.1.8-GCCcore-13.3.0-minimal|
\item \verb|lz4/1.9.4-GCCcore-12.3.0|
\item \verb|lz4/1.9.4-GCCcore-13.3.0|
\item \verb|M4/1.4.18-GCCcore-10.3.0|
\item \verb|M4/1.4.19|
\item \verb|M4/1.4.19-GCCcore-12.3.0|
\item \verb|M4/1.4.19-GCCcore-13.3.0|
\item \verb|M4/1.4.19-GCCcore-14.2.0|
\item \verb|make/4.4.1-GCCcore-12.3.0|
\item \verb|make/4.4.1-GCCcore-13.3.0|
\item \verb|Mako/1.2.4-GCCcore-12.3.0|
\item \verb|Mako/1.3.5-GCCcore-13.3.0|
\item \verb|Mamba/23.11.0-0|
\item \verb|matplotlib/3.7.2-iimkl-2023a|
\item \verb|matplotlib/3.9.2-gfbf-2024a|
\item \verb|maturin/1.6.0-GCCcore-13.3.0|
\item \verb|Mesa/23.1.4-GCCcore-12.3.0|
\item \verb|Mesa/24.1.3-GCCcore-13.3.0|
\item \verb|Meson/1.1.1-GCCcore-12.3.0|
\item \verb|Meson/1.4.0-GCCcore-13.3.0|
\item \verb|meson-python/0.13.2-GCCcore-12.3.0|
\item \verb|meson-python/0.16.0-GCCcore-13.3.0|
\item \verb|NASM/2.16.01-GCCcore-12.3.0|
\item \verb|NASM/2.16.03-GCCcore-13.3.0|
\item \verb|ncurses/6.2-GCCcore-10.3.0|
\item \verb|ncurses/6.3|
\item \verb|ncurses/6.4|
\item \verb|ncurses/6.4-GCCcore-12.3.0|
\item \verb|ncurses/6.5|
\item \verb|ncurses/6.5-GCCcore-13.3.0|
\item \verb|ncurses/6.5-GCCcore-14.2.0|
\item \verb|netCDF/4.9.2-gompi-2023a|
\item \verb|nettle/3.10-GCCcore-13.3.0|
\item \verb|Ninja/1.11.1-GCCcore-12.3.0|
\item \verb|Ninja/1.12.1-GCCcore-13.3.0|
\item \verb|NLopt/2.7.1-GCCcore-13.3.0|
\item \verb|nodejs/18.17.1-GCCcore-12.3.0|
\item \verb|nodejs/20.13.1-GCCcore-13.3.0|
\item \verb|NSPR/4.35-GCCcore-12.3.0|
\item \verb|NSS/3.89.1-GCCcore-12.3.0|
\item \verb|numactl/2.0.16-GCCcore-12.3.0|
\item \verb|numactl/2.0.18-GCCcore-13.3.0|
\item \verb|OpenBLAS/0.3.23-GCC-12.3.0|
\item \verb|OpenBLAS/0.3.27-GCC-13.3.0|
\item \verb|OpenJPEG/2.5.0-GCCcore-12.3.0|
\item \verb|OpenJPEG/2.5.2-GCCcore-13.3.0|
\item \verb|OpenMPI/4.1.5-GCC-12.3.0|
\item \verb|OpenMPI/5.0.3-GCC-13.3.0|
\item \verb|OpenSSL/1.1|
\item \verb|OpenSSL/3|
\item \verb|ParaView/5.13.2-foss-2023a|
\item \verb|patchelf/0.18.0-GCCcore-12.3.0|
\item \verb|patchelf/0.18.0-GCCcore-13.3.0|
\item \verb|PCRE2/10.42-GCCcore-12.3.0|
\item \verb|PCRE2/10.43-GCCcore-13.3.0|
\item \verb|Perl/5.36.1-GCCcore-12.3.0|
\item \verb|Perl/5.38.0|
\item \verb|Perl/5.38.2-GCCcore-13.3.0|
\item \verb|Perl-bundle-CPAN/5.36.1-GCCcore-12.3.0|
\item \verb|Perl-bundle-CPAN/5.38.2-GCCcore-13.3.0|
\item \verb|Pillow/10.0.0-GCCcore-12.3.0|
\item \verb|Pillow-SIMD/10.4.0-GCCcore-13.3.0|
\item \verb|pixman/0.42.2-GCCcore-12.3.0|
\item \verb|pixman/0.43.4-GCCcore-13.3.0|
\item \verb|pkgconf/1.8.0|
\item \verb|pkgconf/1.9.5-GCCcore-12.3.0|
\item \verb|pkgconf/2.2.0-GCCcore-13.3.0|
\item \verb|pkgconf/2.3.0-GCCcore-14.2.0|
\item \verb|pkg-config/0.29.2-GCCcore-10.3.0|
\item \verb|PMIx/4.2.4-GCCcore-12.3.0|
\item \verb|PMIx/5.0.2-GCCcore-13.3.0|
\item \verb|poetry/1.5.1-GCCcore-12.3.0|
\item \verb|poetry/1.8.3-GCCcore-13.3.0|
\item \verb|PostgreSQL/16.4-GCCcore-13.3.0|
\item \verb|PRRTE/3.0.5-GCCcore-13.3.0|
\item \verb|psutil/6.0.0-GCCcore-13.3.0|
\item \verb|pybind11/2.11.1-GCCcore-12.3.0|
\item \verb|pybind11/2.12.0-GCC-13.3.0|
\item \verb|Python/3.11.3-GCCcore-12.3.0|
\item \verb|Python/3.12.3-GCCcore-13.3.0|
\item \verb|Python/3.13.1-GCCcore-14.2.0|
\item \verb|Python-bundle-PyPI/2023.06-GCCcore-12.3.0|
\item \verb|Python-bundle-PyPI/2024.06-GCCcore-13.3.0|
\item \verb|PyYAML/6.0.2-GCCcore-13.3.0|
\item \verb|PyYAML/6.0-GCCcore-12.3.0|
\item \verb|Qhull/2020.2-GCCcore-12.3.0|
\item \verb|Qhull/2020.2-GCCcore-13.3.0|
\item \verb|Qt5/5.15.10-GCCcore-12.3.0|
\item \verb|R/4.4.2-gfbf-2024a|
\item \verb|re2c/3.1-GCCcore-12.3.0|
\item \verb|RStudio-Server/1.4.1717-foss-2021a-Java-11-R-4.1.0|
\item \verb|RStudio-Server/2024.12.0+467-foss-2024a-R-4.4.2|
\item \verb|Rust/1.70.0-GCCcore-12.3.0|
\item \verb|Rust/1.78.0-GCCcore-13.3.0|
\item \verb|ScaLAPACK/2.2.0-gompi-2023a-fb|
\item \verb|ScaLAPACK/2.2.0-gompi-2024a-fb|
\item \verb|scikit-build/0.17.6-GCCcore-12.3.0|
\item \verb|scikit-build/0.17.6-GCCcore-13.3.0|
\item \verb|scikit-build-core/0.10.6-GCCcore-13.3.0|
\item \verb|scikit-build-core/0.5.0-GCCcore-12.3.0|
\item \verb|SciPy-bundle/2023.07-gfbf-2023a|
\item \verb|SciPy-bundle/2023.07-iimkl-2023a|
\item \verb|SciPy-bundle/2024.05-gfbf-2024a|
\item \verb|SDL2/2.28.2-GCCcore-12.3.0|
\item \verb|setuptools-rust/1.6.0-GCCcore-12.3.0|
\item \verb|setuptools-rust/1.9.0-GCCcore-13.3.0|
\item \verb|snappy/1.1.10-GCCcore-12.3.0|
\item \verb|SOCI/4.0.3-GCC-13.3.0|
\item \verb|spglib-python/2.1.0-iimkl-2023a|
\item \verb|spglib-python/2.5.0-gfbf-2024a|
\item \verb|SQLite/3.42.0-GCCcore-12.3.0|
\item \verb|SQLite/3.45.3-GCCcore-13.3.0|
\item \verb|SQLite/3.47.2-GCCcore-14.2.0|
\item \verb|Szip/2.1.1-GCCcore-12.3.0|
\item \verb|Szip/2.1.1-GCCcore-13.3.0|
\item \verb|Tcl/8.6.13-GCCcore-12.3.0|
\item \verb|Tcl/8.6.14-GCCcore-13.3.0|
\item \verb|Tcl/8.6.16-GCCcore-14.2.0|
\item \verb|Tk/8.6.13-GCCcore-12.3.0|
\item \verb|Tk/8.6.14-GCCcore-13.3.0|
\item \verb|Tkinter/3.11.3-GCCcore-12.3.0|
\item \verb|Tkinter/3.12.3-GCCcore-13.3.0|
\item \verb|UCC/1.2.0-GCCcore-12.3.0|
\item \verb|UCC/1.3.0-GCCcore-13.3.0|
\item \verb|UCX/1.14.1-GCCcore-12.3.0|
\item \verb|UCX/1.16.0-GCCcore-13.3.0|
\item \verb|UDUNITS/2.2.28-GCCcore-13.3.0|
\item \verb|UnZip/6.0-GCCcore-12.3.0|
\item \verb|UnZip/6.0-GCCcore-13.3.0|
\item \verb|UnZip/6.0-GCCcore-14.2.0|
\item \verb|util-linux/2.39-GCCcore-12.3.0|
\item \verb|util-linux/2.40-GCCcore-13.3.0|
\item \verb|virtualenv/20.23.1-GCCcore-12.3.0|
\item \verb|virtualenv/20.26.2-GCCcore-13.3.0|
\item \verb|Wayland/1.23.0-GCCcore-13.3.0|
\item \verb|X11/20230603-GCCcore-12.3.0|
\item \verb|X11/20240607-GCCcore-13.3.0|
\item \verb|x264/20230226-GCCcore-12.3.0|
\item \verb|x265/3.5-GCCcore-12.3.0|
\item \verb|xorg-macros/1.20.0-GCCcore-12.3.0|
\item \verb|xorg-macros/1.20.1-GCCcore-13.3.0|
\item \verb|Xvfb/21.1.14-GCCcore-13.3.0|
\item \verb|XZ/5.4.2-GCCcore-12.3.0|
\item \verb|XZ/5.4.5-GCCcore-13.3.0|
\item \verb|XZ/5.6.3-GCCcore-14.2.0|
\item \verb|yaml-cpp/0.8.0-GCCcore-13.3.0|
\item \verb|Yasm/1.3.0-GCCcore-12.3.0|
\item \verb|Z3/4.13.0-GCCcore-13.3.0|
\item \verb|zlib/1.2.11|
\item \verb|zlib/1.2.11-GCCcore-10.3.0|
\item \verb|zlib/1.2.13|
\item \verb|zlib/1.2.13-GCCcore-12.3.0|
\item \verb|zlib/1.3.1|
\item \verb|zlib/1.3.1-GCCcore-13.3.0|
\item \verb|zlib/1.3.1-GCCcore-14.2.0|
\item \verb|zstd/1.5.5-GCCcore-12.3.0|
\item \verb|zstd/1.5.6-GCCcore-13.3.0|
\end{itemize}
\end{multicols}
\normalsize

% -----------------------------------------------------------------------------
\subsection{EL7 (legacy)}
\label{sect:software-el7}

Not all packages are intended for HPC, but the common tree is available
on Speed as well as teaching labs' desktops.

\scriptsize
\begin{multicols}{3}
\begin{itemize}
% EOF
\item \verb|a2ps-4.13b|
\item \verb|a2ps-4.14|
\item \verb|abaqus-2019|
\item \verb|abaqus-2020|
\item \verb|abaqus-2021|
\item \verb|abaqus-2023|
\item \verb|acl-10.0.express|
\item \verb|acl-10.1.express|
\item \verb|acroread-9.5.5|
\item \verb|ADS-2016.01|
\item \verb|ADS-2017.01|
\item \verb|ADS-2019|
\item \verb|ADS-2020u1|
\item \verb|adt_bundle-20140702|
\item \verb|alpine-2.00|
\item \verb|alpine-2.25|
\item \verb|anaconda-1.7.0|
\item \verb|anaconda2-2019.07|
\item \verb|anaconda2-5.1.0|
\item \verb|anaconda3-2019.07|
\item \verb|anaconda3-2019.10|
\item \verb|anaconda3-2021.05|
\item \verb|anaconda3-2023.03|
\item \verb|anaconda3-5.1.0|
\item \verb|android_sdk-24.4.1|
\item \verb|android_studio-162.4069837|
\item \verb|android_studio-173.4720617|
\item \verb|ansoft_designer-6.1.2|
\item \verb|ansoft_designer-8.0|
\item \verb|ansys-16.2|
\item \verb|ansys-17.2|
\item \verb|ansys-18.2|
\item \verb|ansys-19.2|
\item \verb|ansys-2019R3|
\item \verb|ansys-2020R2|
\item \verb|ansys-2021R1|
\item \verb|ansys-2021R2|
\item \verb|ansys-2022R1|
\item \verb|ansys-2022R2|
\item \verb|ansys-2023R1|
\item \verb|ansys-2023R2|
\item \verb|AnsysEM-16.2|
\item \verb|ant-1.10.11|
\item \verb|ant-1.10.2|
\item \verb|ant-1.6.2|
\item \verb|ant-1.9.7|
\item \verb|ANTs-2.3.5|
\item \verb|ApacheDirectoryStudio-1.5.3|
\item \verb|apr-1.7.5|
\item \verb|arduino-1.6.8|
\item \verb|ArgoUML-0.34|
\item \verb|arpack-3.0.2|
\item \verb|ARWpost-3.1|
\item \verb|aspectj-1.7.0.M1|
\item \verb|aspectj-1.8.2|
\item \verb|aspectj-1.8.6|
\item \verb|aspell-0.60.6|
\item \verb|aspell-0.60.8|
\item \verb|atanua-1.2.130617|
\item \verb|autoconf-2.68|
\item \verb|autoconf-2.71|
\item \verb|AutoDock-4.2.6|
\item \verb|autogen-5.18.4|
\item \verb|automake-1.13.1|
\item \verb|automake-1.16.1|
\item \verb|autoson-1.4.3|
\item \verb|babl-0.1.12|
\item \verb|bash-4.4|
\item \verb|basilisk-19_3_23|
\item \verb|bazel-0.2.0|
\item \verb|bazel-0.21.0|
\item \verb|bazel-0.24.1|
\item \verb|bazel-0.26.1|
\item \verb|bazel-2.0.0|
\item \verb|bazel-3.7.2|
\item \verb|bazel-4.2.1|
\item \verb|bison-2.7.1|
\item \verb|bison-3.7.2|
\item \verb|blas-3.10.0|
\item \verb|blender-2.66a|
\item \verb|blender-2.78|
\item \verb|blender-2.78c|
\item \verb|bogofilter-1.2.4|
\item \verb|booksim-2.0|
\item \verb|boost-1.51.0|
\item \verb|boost-1.62.0|
\item \verb|boost-1.68.0|
\item \verb|boost-1.69.0|
\item \verb|boost-1.70.0|
\item \verb|boost-1.73.0|
\item \verb|bouncer-2.2|
\item \verb|buddy-2.4|
\item \verb|bzr-2.6.0|
\item \verb|camlp5-6.02.3|
\item \verb|camlp5-6.14|
\item \verb|CGAL-4.3|
\item \verb|cgiwrap-4.1|
\item \verb|Check-0.15.2|
\item \verb|chrome-101.0.4951.54|
\item \verb|chromium-31.0.1650.63|
\item \verb|cilkplus-4.8|
\item \verb|clips-6.23|
\item \verb|clips-6.30|
\item \verb|clojure-1.8.0|
\item \verb|cmake-2.8.12|
\item \verb|cmake-3.18.4|
\item \verb|cmake-3.26.4|
\item \verb|cmake-3.8.2|
\item \verb|CMC_utils-1.0|
\item \verb|codeblocks-13.12|
\item \verb|codelite-9.1|
\item \verb|comsol-3.5a|
\item \verb|comsol-4.0|
\item \verb|comsol-4.1|
\item \verb|comsol-5.6|
\item \verb|comsol-6.0|
\item \verb|comsol-6.1|
\item \verb|concorde-20031219|
\item \verb|coware-2010.1|
\item \verb|cplex-12.10.0|
\item \verb|cplex-12.6.1|
\item \verb|cplex-12.6.2|
\item \verb|cplex-12.6.3|
\item \verb|cplex-12.7.0|
\item \verb|cplex-12.7.1|
\item \verb|cplex-12.8.0|
\item \verb|cplex-12.9.0|
\item \verb|cplex-20.1.0|
\item \verb|cplex-22.1.0|
\item \verb|cplex-22.1.1|
\item \verb|cppunit-1.13.2|
\item \verb|CST-2014|
\item \verb|CST-2019|
\item \verb|CST-2020|
\item \verb|cuda-10.0|
\item \verb|cuda-10.2|
\item \verb|cuda-11.5|
\item \verb|cuda-11.8|
\item \verb|cuda-9.2|
\item \verb|cups-1.3.7|
\item \verb|cups-2.3.3|
\item \verb|curl-7.15.0|
\item \verb|curl-7.38.0|
\item \verb|curl-7.73.0|
\item \verb|curl-7.84.0|
\item \verb|curl-7.86.0|
\item \verb|curl-7.87.0|
\item \verb|curl-8.1.2|
\item \verb|cvs-1.11.23|
\item \verb|cvx-2.1|
\item \verb|DbVisualizer-24.1.5|
\item \verb|DbVisualizer-9.1.9|
\item \verb|ddd-3.3.12|
\item \verb|dejagnu-1.5.3|
\item \verb|dejagnu-1.6.2|
\item \verb|digilent-2.19.2|
\item \verb|digilent-2.8.2|
\item \verb|dmtcp-1.2.8|
\item \verb|dos2unix-7.3|
\item \verb|doxygen-1.8.4|
\item \verb|drush-6.5.0|
\item \verb|ece-eclipse-jee.juno|
\item \verb|ece-eclipse-jee.luna|
\item \verb|ecj-25|
\item \verb|ECLiPSe-6.0_192|
\item \verb|eclipse-jee.201812|
\item \verb|eclipse-jee.202003|
\item \verb|eclipse-jee.202109|
\item \verb|eclipse-jee.202209|
\item \verb|eclipse-jee.202506|
\item \verb|eclipse-jee.indigo|
\item \verb|eclipse-jee.juno|
\item \verb|eclipse-jee.luna|
\item \verb|eclipse-jee.mars|
\item \verb|eclipse-jee.neon|
\item \verb|eclipse-jee.oxygen|
\item \verb|eigen-3.3.7|
\item \verb|electromagnetics_suite-2022R2|
\item \verb|electromagnetics_suite-2023R1|
\item \verb|emacs-24.4|
\item \verb|emacs-25.2|
\item \verb|enscript-1.6.5.2|
\item \verb|exmh-2.7.0|
\item \verb|expat-2.1.0|
\item \verb|expect-5.45|
\item \verb|expect-5.45.4|
\item \verb|fanout-0.6.1|
\item \verb|feko-2017.2.2|
\item \verb|feko-2018.2|
\item \verb|feko-6.1|
\item \verb|feko-7.0.1|
\item \verb|ffmpeg-0.10.2|
\item \verb|ffmpeg-3.3.2|
\item \verb|ffmpeg-4.1.3|
\item \verb|fftw-3.2.2|
\item \verb|fftw-3.3.10|
\item \verb|fftw-3.3.8|
\item \verb|firefox-102.11.0|
\item \verb|firefox-102.12.0|
\item \verb|firefox-102.13.0|
\item \verb|firefox-102.14.0|
\item \verb|firefox-102.15.0|
\item \verb|firefox-102.15.1|
\item \verb|firefox-115.10.0|
\item \verb|firefox-115.2.1|
\item \verb|firefox-115.3.0|
\item \verb|firefox-17.0.11|
\item \verb|firefox-2.0.0.20|
\item \verb|firefox-24.8.1|
\item \verb|firefox-31.7.0|
\item \verb|firefox-38.8.0|
\item \verb|firefox-45.9.0|
\item \verb|firefox-52.9.0|
\item \verb|firefox-60.9.0|
\item \verb|firefox-68.12.0|
\item \verb|firefox-78.15.0|
\item \verb|firefox-91.11.0|
\item \verb|firefox_french-102.11.0|
\item \verb|firefox_french-102.12.0|
\item \verb|firefox_french-102.13.0|
\item \verb|firefox_french-102.14.0|
\item \verb|firefox_french-102.15.0|
\item \verb|firefox_french-102.15.1|
\item \verb|firefox_french-115.10.0|
\item \verb|firefox_french-115.2.1|
\item \verb|firefox_french-115.3.0|
\item \verb|firefox_french-24.8.1|
\item \verb|firefox_french-31.7.0|
\item \verb|firefox_french-38.8.0|
\item \verb|firefox_french-45.9.0|
\item \verb|firefox_french-52.9.0|
\item \verb|firefox_french-60.9.0|
\item \verb|firefox_french-68.12.0|
\item \verb|firefox_french-78.15.0|
\item \verb|firefox_french-91.11.0|
\item \verb|flex-2.5.37|
\item \verb|flex-2.6.4|
\item \verb|fltk-1.3.2|
\item \verb|fltk-1.3.8|
\item \verb|fox-1.6.49|
\item \verb|FreeImage-3.18.0|
\item \verb|freerdp-1.0.2|
\item \verb|freerdp-1.2.0|
\item \verb|freetype-2.4.12|
\item \verb|FSL-6.0.5|
\item \verb|fsl-6.0.6.2|
\item \verb|fsr-1.9|
\item \verb|fxscintilla-2.28.0|
\item \verb|gambit-c-4.7.5|
\item \verb|gate-7.1|
\item \verb|gc-7.2f|
\item \verb|gcc-12.2.0|
\item \verb|gcc-13.1.0|
\item \verb|gcc-3.3.2|
\item \verb|gcc-4.1.2|
\item \verb|gcc-4.4.3|
\item \verb|gcc-4.7.2|
\item \verb|gcc-4.9.2|
\item \verb|gcc-5.1.0|
\item \verb|gcc-5.2.0|
\item \verb|gcc-5.4.0|
\item \verb|gcc-6.1.0|
\item \verb|gcc-7.3.0|
\item \verb|gcc-8.4.0|
\item \verb|gcc-9.2.0|
\item \verb|gcc-9.3.0|
\item \verb|gcc-arm-11.2.2022.02|
\item \verb|gdb-12.1|
\item \verb|gdb-7.7|
\item \verb|geomview-1.9.4|
\item \verb|gerris-131206|
\item \verb|getmail-4.54.0|
\item \verb|gettext-0.18.1.1|
\item \verb|gfsview-121130|
\item \verb|ghc-7.6.3|
\item \verb|ghostscript-10.0.0|
\item \verb|ghostscript-8.50|
\item \verb|ghostscript-9.25|
\item \verb|ghostscript-9.26|
\item \verb|ghostscript-9.50|
\item \verb|ghostscript-9.52|
\item \verb|gifsicle-1.92|
\item \verb|gimp-2.8.14|
\item \verb|gimp-2.8.18|
\item \verb|git-1.8.5.3|
\item \verb|git-2.15.0|
\item \verb|git-2.22.0|
\item \verb|git-2.23.0|
\item \verb|git-2.23.3|
\item \verb|git-2.29.2|
\item \verb|git-2.35.1|
\item \verb|git-2.5.0|
\item \verb|gl2ps-1.3.8|
\item \verb|glade3-3.8.3|
\item \verb|glew-1.10.0|
\item \verb|glew-2.1.0|
\item \verb|glfw-3.0.4|
\item \verb|glfw-3.3|
\item \verb|glfw-3.3.4|
\item \verb|glm-0.9.5.4|
\item \verb|glm-0.9.9.8|
\item \verb|glpk-4.55|
\item \verb|gmp-4.3.2|
\item \verb|gnome_env-20110618|
\item \verb|gnome_env-20130510|
\item \verb|gnupg-2.3.4|
\item \verb|gnuplot-4.6.3|
\item \verb|gnuplot-5.0.1|
\item \verb|gnuplot-5.0.3|
\item \verb|go-1.10.2|
\item \verb|go-1.12|
\item \verb|go-1.15.6|
\item \verb|go-1.19.3|
\item \verb|go-1.3|
\item \verb|go-1.9|
\item \verb|GoldenGate-2020u1|
\item \verb|gperf-3.0.4|
\item \verb|grace-5.1.25|
\item \verb|GraphiteTwo-a5|
\item \verb|GraphiteTwo-Concordia_v1|
\item \verb|graphviz-2.30.0|
\item \verb|graphviz-2.40.1|
\item \verb|gromacs-5.0.7|
\item \verb|gsl-2.6|
\item \verb|gtkwave-3.3.15|
\item \verb|gts-121130|
\item \verb|guile-2.0.11|
\item \verb|gurobi-10.0.0|
\item \verb|gurobi-10.0.1|
\item \verb|gurobi-7.0.1|
\item \verb|gurobi-7.5.0|
\item \verb|gurobi-8.0.0|
\item \verb|gurobi-8.1.0|
\item \verb|gurobi-9.0.0|
\item \verb|gurobi-9.0.2|
\item \verb|gurobi-9.1.0|
\item \verb|gurobi-9.5.0|
\item \verb|gurobi-9.5.2|
\item \verb|gv-3.7.4|
\item \verb|hadoop-1.0.4|
\item \verb|hadoop-1.1.2|
\item \verb|hadoop-2.4.1|
\item \verb|hadoop-2.7.1|
\item \verb|hadoop-2.7.3|
\item \verb|hadoop-2.9.0|
\item \verb|haskell_platform-2013.2.0.0|
\item \verb|hdf5-1.10.5|
\item \verb|hdf5-1.8.11|
\item \verb|hexpert-2.4.1|
\item \verb|hmmer-3.3.2|
\item \verb|html2ps-1.0b5|
\item \verb|httpd-2.2.22|
\item \verb|httpd-2.2.32|
\item \verb|httpd-2.4.27|
\item \verb|httpd-2.4.38|
\item \verb|httpd-2.4.39|
\item \verb|httpd-2.4.52|
\item \verb|httpd-2.4.53|
\item \verb|httpd-2.4.54|
\item \verb|httpd-2.4.55|
\item \verb|httpd-2.4.57|
\item \verb|httpd-2.4.63|
\item \verb|httpd-current|
\item \verb|http-parser-2.9.4|
\item \verb|hunspell-1.7.2|
\item \verb|hwloc-1.11.2|
\item \verb|hwloc-2.8.0|
\item \verb|hxplay-11.0.2|
\item \verb|hyperworks-12.0|
\item \verb|iBioSim-3.0.0.beta|
\item \verb|ImageMagick-6.3.6|
\item \verb|ImageMagick-6.9.4.1|
\item \verb|ImageMagick-7.0.8.42|
\item \verb|imap-2007e|
\item \verb|instantclient-19.3.0.0.0|
\item \verb|IntelliJ-2019.2.3|
\item \verb|intltool-0.50.2|
\item \verb|invtools-2.0|
\item \verb|invtools-2.0.2|
\item \verb|invtools-2.1.0|
\item \verb|invtools-2.2.0|
\item \verb|invtools-2.3.0|
\item \verb|invtools-2.4.0|
\item \verb|invtools-2.5.0|
\item \verb|invtools-2.5.1|
\item \verb|invtools-2.6.0|
\item \verb|invtools-2.7.0|
\item \verb|invtools-2.8.0|
\item \verb|invtools-3.0.0|
\item \verb|invtools-3.1.0|
\item \verb|invtools-3.2.0|
\item \verb|ipe-7.1.3|
\item \verb|iq-8.4.1|
\item \verb|j2sdk_32b-1.5.0_22|
\item \verb|j2sdk_64b-1.5.0_22|
\item \verb|jansson-2.14|
\item \verb|jdk-11|
\item \verb|jdk-11.0.1|
\item \verb|jdk-11.0.2|
\item \verb|jdk-17|
\item \verb|jdk-17.0.2|
\item \verb|jdk-19|
\item \verb|jdk-19.0.2|
\item \verb|jdk_32b-6u45|
\item \verb|jdk_32b-7u80|
\item \verb|jdk_32b-8u101|
\item \verb|jdk_32b-8u111|
\item \verb|jdk_32b-8u121|
\item \verb|jdk_32b-8u131|
\item \verb|jdk_32b-8u141|
\item \verb|jdk_32b-8u151|
\item \verb|jdk_32b-8u161|
\item \verb|jdk_32b-8u171|
\item \verb|jdk_32b-8u181|
\item \verb|jdk_32b-8u191|
\item \verb|jdk_32b-8u201|
\item \verb|jdk_32b-8u211|
\item \verb|jdk_32b-8u231|
\item \verb|jdk_32b-8u91|
\item \verb|jdk-5|
\item \verb|jdk-6|
\item \verb|jdk-6_32b|
\item \verb|jdk_64b-6u45|
\item \verb|jdk_64b-7u80|
\item \verb|jdk_64b-8u101|
\item \verb|jdk_64b-8u111|
\item \verb|jdk_64b-8u121|
\item \verb|jdk_64b-8u131|
\item \verb|jdk_64b-8u141|
\item \verb|jdk_64b-8u151|
\item \verb|jdk_64b-8u161|
\item \verb|jdk_64b-8u171|
\item \verb|jdk_64b-8u181|
\item \verb|jdk_64b-8u191|
\item \verb|jdk_64b-8u201|
\item \verb|jdk_64b-8u211|
\item \verb|jdk_64b-8u231|
\item \verb|jdk_64b-8u91|
\item \verb|jdk-7|
\item \verb|jdk-7_32b|
\item \verb|jdk-8|
\item \verb|jdk-8_32b|
\item \verb|jes-5.02|
\item \verb|jmag-20.1.02zi|
\item \verb|json-c-0.16|
\item \verb|julia-1.7.2|
\item \verb|kaldi-5.5|
\item \verb|kerberos-helpers-1.0|
\item \verb|kicad-4.0.1|
\item \verb|kile-2.0.1|
\item \verb|lam-7.1.3|
\item \verb|lapack-3.10.0|
\item \verb|lapack-3.4.2|
\item \verb|lasso-2.6.0|
\item \verb|lib3ds-1.3.0|
\item \verb|libarchive-3.1.2|
\item \verb|libarchive-3.3.2|
\item \verb|libassuan-2.5.5|
\item \verb|libatomic_ops-7.4.0|
\item \verb|libav-12.3|
\item \verb|libedit-20210910.3.1|
\item \verb|libevent-2.1.12|
\item \verb|libffi-3.2.1|
\item \verb|libgcrypt-1.10.1|
\item \verb|libgd-2.2.5|
\item \verb|libglade-2.6.4|
\item \verb|libgpg-error-1.44|
\item \verb|libidn2-2.3.2|
\item \verb|libjwt-1.15.2|
\item \verb|libksba-1.6.0|
\item \verb|LibreOffice-7.1.8|
\item \verb|LibreOffice-7.3.7|
\item \verb|LibreOffice-7.4.7|
\item \verb|libssh-0.7.3|
\item \verb|libssh-0.7.6|
\item \verb|libtool-2.4.6|
\item \verb|libunistring-0.9.6|
\item \verb|libusb-1.0.8|
\item \verb|libxml2-2.7.6|
\item \verb|libxml2-2.7.8|
\item \verb|libxml2-2.9.4|
\item \verb|libyaml-0.2.5|
\item \verb|libzip-1.5.1|
\item \verb|lispworks-6.1|
\item \verb|lispworks-7.1|
\item \verb|lispworks-8.0|
\item \verb|Logiscope-6.6.0|
\item \verb|Logiscope-7.1|
\item \verb|lshw-B.02.16|
\item \verb|ltct-20060701|
\item \verb|lua-5.1.5|
\item \verb|lua-5.3.4|
\item \verb|lucene-4.4.0|
\item \verb|lucene-6.6.0|
\item \verb|LWkit-2.0|
\item \verb|lynx-2.8.7|
\item \verb|lynx-2.8.9|
\item \verb|lynx-2.8.9dev.11|
\item \verb|lyx-2.0.3|
\item \verb|lz4-1.9.4|
\item \verb|m4-1.4.17|
\item \verb|mairix-0.23|
\item \verb|make-4.4.1|
\item \verb|maple-2019.1|
\item \verb|maple-2022.1|
\item \verb|matlab-R0223b|
\item \verb|matlab-R2020b|
\item \verb|matlab-R2021b|
\item \verb|matlab-R2022a|
\item \verb|matlab-R2022b|
\item \verb|matlab-R2023a|
\item \verb|matlab-R2023b|
\item \verb|maven-3.3.9|
\item \verb|maven-3.6.3|
\item \verb|mcmas-1.0.1|
\item \verb|mcmas-1.3.0|
\item \verb|mercurial-3.0.2|
\item \verb|mercurial-5.6.1|
\item \verb|mesa-19.0.3|
\item \verb|metamail-2.7|
\item \verb|metis-5.1.0|
\item \verb|mininet-2.3.0|
\item \verb|MKL-2021.04|
\item \verb|mod_auth_kerb-5.4|
\item \verb|mod_auth_mellon-0.17.0|
\item \verb|mod_authnz_external-3.2.5|
\item \verb|mod_authnz_external-3.3.2|
\item \verb|mod_authz_unixgroup-1.0.2|
\item \verb|mod_authz_unixgroup-1.1.0|
\item \verb|modeFRONTIER-2017R2|
\item \verb|mod_perl-2.0.10|
\item \verb|mod_perl-2.0.5|
\item \verb|modules-3.2.10|
\item \verb|modules-4.6.1|
\item \verb|modules-5.3.1|
\item \verb|modules-current|
\item \verb|mongodb-2.6.5|
\item \verb|mongodb-3.2.10|
\item \verb|mongodb-3.4.6|
\item \verb|mono-3.0.11|
\item \verb|mono-4.4.2.8|
\item \verb|mono-6.12.0.122|
\item \verb|monodevelop-3.1.1|
\item \verb|monodevelop-5.10.0|
\item \verb|monodevelop-7.8.4|
\item \verb|mosek-7.1.0.54|
\item \verb|mosek-8.1.0.58|
\item \verb|mpack-1.6|
\item \verb|mpc-1.0.1|
\item \verb|mpfr-2.4.2|
\item \verb|mpich-4.1|
\item \verb|MPlayer-1.1.1|
\item \verb|mplayerplugin-3.45|
\item \verb|mpm-itk-2.4.7|
\item \verb|MRtrix-3.0.3|
\item \verb|MUMPS-5.5.0|
\item \verb|mysql-5.1.66|
\item \verb|mysql-5.6.17|
\item \verb|mysql-5.6.37|
\item \verb|mysql-5.6.39|
\item \verb|mysql-5.6.42|
\item \verb|mysql-5.6.43|
\item \verb|mysql-5.7.20|
\item \verb|mysql-5.7.36|
\item \verb|mysql-8.0.16|
\item \verb|mysql-8.0.18|
\item \verb|mysql-8.0.22|
\item \verb|mysql-8.0.23|
\item \verb|mysql-8.0.31|
\item \verb|mysql_workbench-6.3.3|
\item \verb|mysql_workbench-8.0.16|
\item \verb|nagtools-2.1.10|
\item \verb|nagtools-2.1.3|
\item \verb|nagtools-2.1.4|
\item \verb|nagtools-2.1.5|
\item \verb|nagtools-2.1.6|
\item \verb|nagtools-2.1.7|
\item \verb|nagtools-2.1.8|
\item \verb|nagtools-2.1.9|
\item \verb|nano-6.2|
\item \verb|nasm-2.10.07|
\item \verb|nasm-2.14.02|
\item \verb|nasm-2.15.05|
\item \verb|ncl-6.6.2|
\item \verb|ncurses-6.4|
\item \verb|netbeans-7.3.1|
\item \verb|netbeans-8.0.2|
\item \verb|netbeans-8.1|
\item \verb|netcdf-4.1.3|
\item \verb|netcdf-4.3.0|
\item \verb|netcdf_c-4.7.2|
\item \verb|netcdf_fortran-4.5.2|
\item \verb|netpbm-10.35.95|
\item \verb|net-snmp-5.4.1|
\item \verb|net-snmp-5.9.1|
\item \verb|nettle-2.7.1|
\item \verb|ninja-1.10.2|
\item \verb|nltk-3.0|
\item \verb|nltk-3.3|
\item \verb|nmh-1.1rc3|
\item \verb|nmh-1.5|
\item \verb|nmh-1.6|
\item \verb|nmh-1.7.1|
\item \verb|node-v0.10.33|
\item \verb|node-v12.13.0|
\item \verb|node-v12.16.1|
\item \verb|node-v12.18.0|
\item \verb|node-v16.13.0|
\item \verb|node-v7.8.0|
\item \verb|node-v9.5.0|
\item \verb|nph-1.2.2|
\item \verb|npth-1.6|
\item \verb|ns-3.26|
\item \verb|ntbtls-0.3.0|
\item \verb|nvtop-3.0.1|
\item \verb|ocaml-3.12.1|
\item \verb|ocaml-4.01.0|
\item \verb|octave-3.8.2|
\item \verb|octave-4.0.3|
\item \verb|octave-7.3.0|
\item \verb|octave-8.2.0|
\item \verb|omnetpp-4.2.2|
\item \verb|omnetpp-5.6.2|
\item \verb|oniguruma-6.9.5|
\item \verb|Open3D-0.11.1|
\item \verb|opencv-3.0.0|
\item \verb|opencv-3.3.0|
\item \verb|opencv-3.4.5|
\item \verb|opencv-4.0.1|
\item \verb|opencv-4.5.0|
\item \verb|opencv-4.5.4|
\item \verb|openesb-2.3.1|
\item \verb|openesb-3.0.5|
\item \verb|OpenFOAM-10.0|
\item \verb|OpenFOAM-11.0|
\item \verb|OpenFOAM-1.7.1|
\item \verb|OpenFOAM-2.3.1|
\item \verb|OpenFOAM-2.4.0|
\item \verb|OpenFOAM-3.0.1|
\item \verb|OpenFOAM-5.0|
\item \verb|OpenFOAM-6.0|
\item \verb|OpenFOAM-7.0|
\item \verb|OpenFOAM-8.0|
\item \verb|OpenFOAM-9.0|
\item \verb|OpenFOAM-v2012|
\item \verb|OpenFOAM-v2106|
\item \verb|OpenFOAM-v2206|
\item \verb|OpenFOAM-v2212|
\item \verb|OpenFOAM-v2306|
\item \verb|openjpeg-2.3.1|
\item \verb|openmotif-2.2.4|
\item \verb|openmpi-1.6.3|
\item \verb|openmpi-1.8.3|
\item \verb|openmpi-3.1.3|
\item \verb|openmpi-4.0.1|
\item \verb|openocd-0.11.0|
\item \verb|openpmix-5.0.1|
\item \verb|OpenSees-3.4.0|
\item \verb|openssl-1.0.2.current|
\item \verb|openssl-1.0.2l|
\item \verb|openssl-1.0.2o|
\item \verb|openssl-1.0.2u|
\item \verb|openssl-1.1.1a|
\item \verb|openssl-1.1.1.current|
\item \verb|openssl-1.1.1g|
\item \verb|openssl-1.1.1j|
\item \verb|openssl-1.1.1n|
\item \verb|openssl-1.1.1o|
\item \verb|openssl-1.1.1p|
\item \verb|openssl-1.1.1q|
\item \verb|openssl-1.1.1s|
\item \verb|openssl-1.1.1t|
\item \verb|openssl-1.1.1u|
\item \verb|openssl-3.0.10|
\item \verb|openssl-3.0.11|
\item \verb|openssl-3.0.12|
\item \verb|openssl-3.0.4|
\item \verb|openssl-3.0.5|
\item \verb|openssl-3.0.6|
\item \verb|openssl-3.0.7|
\item \verb|openssl-3.0.8|
\item \verb|openssl-3.0.9|
\item \verb|openssl-3.0.current|
\item \verb|openssl-3.1.0|
\item \verb|openssl-3.1.current|
\item \verb|opera-12.16|
\item \verb|oracle-19c|
\item \verb|os-overrides-1.0|
\item \verb|otp-20.1|
\item \verb|p11_kit-0.22.1|
\item \verb|p7zip-16.02|
\item \verb|p7zip-9.20.1|
\item \verb|parallel-20200322|
\item \verb|ParaView-5.11.2|
\item \verb|parmetis-4.0.3|
\item \verb|pc2-9.3.1|
\item \verb|pcre-8.21|
\item \verb|pdfedit-0.4.5|
\item \verb|perl-5.28.1|
\item \verb|perl-5.28.1_mt|
\item \verb|perl-5.30.3|
\item \verb|perl-5.30.3_mt|
\item \verb|perl-5.8.3|
\item \verb|perl-5.8.3_mt|
\item \verb|pgadmin3-1.10.2|
\item \verb|pgadmin4-5.0|
\item \verb|pgadmin4-7.6|
\item \verb|php-5.3.15|
\item \verb|php-5.5.11|
\item \verb|php-5.5.12|
\item \verb|php-5.5.38|
\item \verb|php-5.6.33|
\item \verb|php-5.6.40|
\item \verb|php-7.2.11|
\item \verb|php-7.2.4|
\item \verb|php-7.2.5|
\item \verb|php-7.4.11|
\item \verb|php-7.4.14|
\item \verb|php-7.4.27|
\item \verb|php-7.4.33|
\item \verb|php-8.2.6|
\item \verb|ph-substitute-1.0|
\item \verb|pinentry-1.2.0|
\item \verb|poppler-0.81.0|
\item \verb|postgresql-11.7|
\item \verb|postgresql-12|
\item \verb|postgresql-12.2|
\item \verb|postgresql-12.3|
\item \verb|postgresql-16|
\item \verb|postgresql-16.9|
\item \verb|postgresql-8.0.15|
\item \verb|postgresql-8.3.18|
\item \verb|postgresql-9.6.8|
\item \verb|praat-6.3.06|
\item \verb|Precision-2010aU1|
\item \verb|print-utils-1.0|
\item \verb|probreg-0.3.1|
\item \verb|proj-6.0.0|
\item \verb|protege-4.0|
\item \verb|protobuf-2.5.0|
\item \verb|pwauth-2.3.11|
\item \verb|pwauth-2.3.8|
\item \verb|python-2.5.6|
\item \verb|python-2.7.10|
\item \verb|python-2.7.11|
\item \verb|python-3.10.13|
\item \verb|python-3.10.6|
\item \verb|python-3.11.0|
\item \verb|python-3.11.5|
\item \verb|python-3.11.6|
\item \verb|python-3.12.0|
\item \verb|python-3.2.3|
\item \verb|python-3.3.0|
\item \verb|python-3.4.3|
\item \verb|python-3.5.1|
\item \verb|python-3.6.15|
\item \verb|python-3.7.3|
\item \verb|python-3.7.7|
\item \verb|python-3.8.18|
\item \verb|python-3.8.3|
\item \verb|python-3.8.9|
\item \verb|python-3.9.1|
\item \verb|python-3.9.18|
\item \verb|pytorch-1.1.0|
\item \verb|pytorch-1.10.0|
\item \verb|pytorch-1.6.0|
\item \verb|qemu-6.1.1|
\item \verb|qhull-2012.1|
\item \verb|qhull-2020.2|
\item \verb|qrupdate-1.1.2|
\item \verb|QScintilla-2.8.4|
\item \verb|qt-4.8.6|
\item \verb|qt-5.14.2|
\item \verb|qt-5.9.6|
\item \verb|qt_sdk-1.2.1|
\item \verb|quota-1.3|
\item \verb|R-3.4.1|
\item \verb|R-3.4.2|
\item \verb|R-3.5.0|
\item \verb|R-3.6.1|
\item \verb|R-4.1.3|
\item \verb|R-4.2.2|
\item \verb|racket-8.3|
\item \verb|rapidjson-1.1.0|
\item \verb|rational_rose_realtime-7.0|
\item \verb|rdesktop-1.8.3|
\item \verb|rdesktop-1.8.4|
\item \verb|readline-5.2|
\item \verb|redis-4.0.10|
\item \verb|redis-4.0.8|
\item \verb|RSA-8.5|
\item \verb|ruby-2.1.1|
\item \verb|ruby-2.3.1|
\item \verb|ruby-2.5.0|
\item \verb|ruby-2.5.3|
\item \verb|ruby-2.6.3|
\item \verb|ruby-2.6.5|
\item \verb|ruby-2.7.1|
\item \verb|ruby-3.2.1|
\item \verb|rust-1.62.1|
\item \verb|rust-1.65.0|
\item \verb|sag-dbtools-1.0|
\item \verb|sage-8.9|
\item \verb|salome-9.10.0|
\item \verb|sbt-1.1.1|
\item \verb|scala-2.12.1|
\item \verb|scala-2.12.4|
\item \verb|scalapack-2.2.0|
\item \verb|SciTE-3.4.0|
\item \verb|scons-2.3.1|
\item \verb|scotch-6.0.0|
\item \verb|scotch-6.1.3|
\item \verb|SDL2-2.0.9|
\item \verb|sep-offprint-1.11|
\item \verb|serf-1.3.6|
\item \verb|serscis-access-modeller-0.16|
\item \verb|sfold-2.2|
\item \verb|sharutils-4.6.3|
\item \verb|shells-1.1|
\item \verb|singularity-2.6.1|
\item \verb|singularity-3.10.4|
\item \verb|singularity-3.4.2|
\item \verb|singularity-3.7.0|
\item \verb|slicer-4.11.20210226|
\item \verb|Smarty-2.6.26|
\item \verb|smpeg-2.0.0|
\item \verb|spacetools-1.0.0|
\item \verb|spacetools-1.1.0|
\item \verb|spacetools-1.2.0|
\item \verb|spack-0.16.0|
\item \verb|spark-2.2.0|
\item \verb|spin-5.2.4|
\item \verb|spin-6.5.1|
\item \verb|spm-12|
\item \verb|sqlite-3.44.2|
\item \verb|sqlite-3.8.4.3|
\item \verb|squashfs-4.3|
\item \verb|StarCCM-12.06.011|
\item \verb|StarCCM-13.06.012|
\item \verb|StarCCM-14.06.013|
\item \verb|StarCCM-15.04.010|
\item \verb|stealfile-1.0|
\item \verb|STSLib-1.0|
\item \verb|subversion-1.14.2|
\item \verb|subversion-1.6.17|
\item \verb|subversion-1.8.9|
\item \verb|subversion-1.9.5|
\item \verb|SuiteSparse-4.2.1|
\item \verb|sundials-6.4.1|
\item \verb|SuperLU-4.3|
\item \verb|SWI-Prolog-7.2.3|
\item \verb|SWI-Prolog-8.4.0|
\item \verb|tcl-8.4.16|
\item \verb|tcl-8.5.14|
\item \verb|tcl-8.6.13|
\item \verb|tcmalloc-1.0|
\item \verb|tcsh-6.18.01|
\item \verb|teams-1.5.00|
\item \verb|tecplot360-2021R2|
\item \verb|tecplot360-2022R1|
\item \verb|tecplot360-2022R2|
\item \verb|tecplot360-2023R1|
\item \verb|TensorFlow-1.15.2|
\item \verb|TensorFlow-1.9|
\item \verb|TensorFlow-2.2.0|
\item \verb|TensorFlow-2.2.1|
\item \verb|TensorFlow-2.4.1|
\item \verb|TensorFlow-r0.7|
\item \verb|texinfo-5.2|
\item \verb|texinfo-7.0.2|
\item \verb|texlive-20220405|
\item \verb|texlive-20230324|
\item \verb|texlive-current|
\item \verb|texstudio-2.12.16|
\item \verb|texstudio-4.5.1|
\item \verb|tgif-4.2.5|
\item \verb|tgrid-5.0.6|
\item \verb|thttpd-2.25b|
\item \verb|thunderbird-102.10.0|
\item \verb|thunderbird-102.11.1|
\item \verb|thunderbird-102.12.0|
\item \verb|thunderbird-115.1.0|
\item \verb|thunderbird-115.10.2|
\item \verb|thunderbird_french-102.10.0|
\item \verb|thunderbird_french-102.11.1|
\item \verb|thunderbird_french-102.12.0|
\item \verb|thunderbird_french-115.1.0|
\item \verb|thunderbird_french-115.10.2|
\item \verb|tidy-5.7.28|
\item \verb|tix-8.1.4|
\item \verb|tk-8.4.16|
\item \verb|tk-8.5.14|
\item \verb|tk-8.6.13|
\item \verb|tmux-3.2a|
\item \verb|tnef-1.4.4|
\item \verb|tomcat-5.5.23|
\item \verb|tomcat-6.0.26|
\item \verb|tomcat-6.0.32|
\item \verb|tomcat-6.0.33|
\item \verb|tomcat-6.0.35|
\item \verb|tomcat-6.0.36|
\item \verb|tomcat-6.0.41|
\item \verb|tomcat-7.0.100|
\item \verb|tomcat-7.0.104|
\item \verb|tomcat-7.0.107|
\item \verb|tomcat-7.0.2|
\item \verb|tomcat-7.0.57|
\item \verb|tomcat-7.0.67|
\item \verb|tomcat-7.0.70|
\item \verb|tomcat-7.0.78|
\item \verb|tomcat-7.0.85|
\item \verb|tomcat-7.0.90|
\item \verb|tomcat-7.0.91|
\item \verb|tomcat-9.0.79|
\item \verb|tomcat-9.0.98|
\item \verb|tomcat9-current|
\item \verb|tomcat_connectors-1.2.42|
\item \verb|tomcat_connectors-1.2.46|
\item \verb|transfig-3.2.5|
\item \verb|unafold-3.8|
\item \verb|unrar-3.8.2|
\item \verb|unzip-6.0|
\item \verb|user-utils-1.0|
\item \verb|uudeview-0.5.20|
\item \verb|uuid-1.6.2|
\item \verb|vacation-sendmail-8.12.11|
\item \verb|vacation-sendmail-8.16.1|
\item \verb|valgrind-3.15.0|
\item \verb|verilog-0.9.3|
\item \verb|vmd-1.9.3|
\item \verb|vnc-4.1.3|
\item \verb|VNLB-1.0|
\item \verb|VSCode-1570750623|
\item \verb|web-helpers-1.3|
\item \verb|websphinx-0.5|
\item \verb|weka-3.8.0|
\item \verb|wget-1.16|
\item \verb|wget-1.19|
\item \verb|wget-1.20.3|
\item \verb|wine-1.1.20|
\item \verb|wireshark-2.0.4|
\item \verb|WRF-4.1.2|
\item \verb|wxGTK-2.8.12|
\item \verb|wxGTK-2.8.9|
\item \verb|wxWidgets-3.0.2|
\item \verb|xcal-4.1.18.2|
\item \verb|xcalendar-4.0|
\item \verb|xemacs-21.4.22|
\item \verb|xemacs-21.4.24|
\item \verb|xemacs-21.5.34|
\item \verb|xfig-3.2.5|
\item \verb|xmgr-4.1.2|
\item \verb|xmlsec1-1.2.31|
\item \verb|xpdf-3.04|
\item \verb|xsb-3.3.7|
\item \verb|xv-3.10a|
\item \verb|xz-5.0.0|
\item \verb|xz-5.2.2|
\item \verb|yasm-1.3.0|
\item \verb|zip-3.0|
\item \verb|zlib-1.2.13|
\end{itemize}
\end{multicols}
\normalsize


% ------------------------------------------------------------------------------
% Refs:
%
\nocite{aosa-book-vol1}
\label{sect:bib}
%\bibliographystyle{IEEEtran}
\bibliographystyle{plain}
%\bibliographystyle{alpha}
%\bibliographystyle{unsrt}
%\bibliographystyle{abbrv}
% Create a section for references otherwise it appears to be part of the "Sister Facilities" Appendix
\clearpage
\addcontentsline{toc}{section}{Annotated Bibliography} 
\bibliography{speed-manual}

%------------------------------------------------------------------------------
\end{document}
