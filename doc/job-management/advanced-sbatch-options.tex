% Advanced sbatch Options
% -------------------------------------------------------------
\subsection{Advanced \tool{sbatch} Options}
\label{sect:submit-options}

In addition to the basic sbatch options presented earlier,
there are several advanced options that are generally useful:

\begin{itemize}
	\item E-mail notifications: requests the scheduler to send an email when the job changes state.
	\begin{verbatim}
		--mail-type=<TYPE>
	\end{verbatim}
	\texttt{<TYPE>} can be \texttt{ALL}, \texttt{BEGIN}, \texttt{END}, or \texttt{FAIL}.

	Mail is sent to the default address of \tool{<ENCSusername>@encs.concordia.ca}

	Which you can consult via \url{webmail.encs.concordia.ca} (use VPN from off-campus)
	unless a different address is supplied (see, \option{--mail-user}). The report sent when
    a job ends includes job runtime, as well as the maximum memory value hit (\api{maxvmem}).

    To specify a different email address for notifications rather than the default, use
    \begin{verbatim}
		--mail-user email@domain.com
	\end{verbatim}

	\item Export environment variables used by the script:
	\begin{verbatim}
		--export=ALL
		--export=NONE
		--export=VARIABLES
	\end{verbatim}

	\item Job runtime: sets a job runtime of min or HH:MM:SS. Note that if you give a single number,
	that represents \emph{minutes}, not hours. The set runtime should not exceed
	the default maximums of 24h for interactive jobs and 7 days for batch jobs.
	\begin{verbatim}
		-t <MINUTES> or DAYS-HH:MM:SS
	\end{verbatim}

	\item Job Dependencies: Runs the job only when the specified job \verb|<job-ID>| finishes.
    This is useful for creating job chains where subsequent jobs depend on the completion of previous ones.
	\begin{verbatim}
		--depend=<state:job-ID>
	\end{verbatim}
\end{itemize}

\noindent \textbf{Note:} \tool{sbatch} options can be specified during the job-submission command,
and these \emph{override} existing script options (if present). The syntax is
\begin{verbatim}
	sbatch [options] path/to/script
\end{verbatim}
but unlike in the script, the options are specified without the leading \verb+#SBATCH+
e.g.:
\begin{verbatim}
	sbatch -J sub-test --chdir=./ --mem=1G ./env.sh
\end{verbatim}