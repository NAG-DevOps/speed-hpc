% Getting Started
% -------------------------------------------------------------
\subsection{Getting Started}
\label{sect:getting-started}

Before getting started, please review the ``What Speed is'' (\xs{sect:speed-is-for})
and ``What Speed is Not'' (\xs{sect:speed-is-not}).
Once your GCS ENCS account has been granted access to ``Speed'',
use your GCS ENCS account credentials to create an SSH connection to
\texttt{speed} (an alias for \texttt{speed-submit.encs.concordia.ca}).

All users are expected to have a basic understanding of
Linux and its commonly used commands (see \xa{sect:faqs} for resources).

% SSH Connection
% -----------------------
\subsubsection{SSH Connections}
\label{sect:ssh-connection}

Requirements to create SSH connection to ``Speed'':
\begin{enumerate}
	\item \textbf{Active GCS ENCS user account:} Ensure you have an active GCS ENCS user account with
	permission to connect to Speed (see \xs{sect:access-requests}).
	\item \textbf{VPN Connection} (for off-campus access): If you are off-campus, you wil need to establish an active connection to Concordia's VPN,
	which requires a Concordia netname.
	\item \textbf{Terminal Emulator for Windows:} Windows systems use a terminal emulator such as PuTTY, Cygwin, or MobaXterm.
	\item \textbf{Terminal for macOS:} macOS systems have a built-in Terminal app or \tool{xterm} that comes with XQuartz.
\end{enumerate}

\noindent To create an SSH connection to Speed, open a terminal window and type the following command, replacing \verb!<ENCSusername>! with your ENCS account's username:
\begin{verbatim}
    ssh <ENCSusername>@speed.encs.concordia.ca
\end{verbatim}

\noindent For detailed instructions on securely connecting to a GCS server, refer to the AITS FAQ:
\href{https://www.concordia.ca/ginacody/aits/support/faq/ssh-to-gcs.html}{How do I securely connect to a GCS server?}

% Environment Set Up
% --------------------------
\subsubsection{Environment Set Up}
\label{sect:envsetup}

After creating an SSH connection to Speed, you will need to make sure the \tool{srun}, \tool{sbatch}, and \tool{salloc}
commands are available to you. To check this, type each command at the prompt and press Enter.
If ``command not found'' is returned, you need to make sure your \api{\$PATH} includes \texttt{/local/bin}.
You can check your path by typing:
\begin{verbatim}
    echo $PATH
\end{verbatim}

\noindent The next step is to set up your cluster-specific storage ``speed-scratch'', to do so, execute the following command from within your
home directory.
\begin{verbatim}
    mkdir -p /speed-scratch/$USER && cd /speed-scratch/$USER
\end{verbatim}

\noindent Next, copy a job template to your cluster-specific storage
\begin{itemize}
    \item From Windows drive G: to Speed:\\
    \verb|cp /winhome/<1st letter of $USER>/$USER/<script>.sh /speed-scratch/$USER/|
    \item From Linux drive U: to Speed:\\
    \verb|cp ~/<script>.sh /speed-scratch/$USER/|
\end{itemize}

\noindent \textbf{Tip:} the default shell for GCS ENCS users is \tool{tcsh}.
If you would like to use \tool{bash}, please contact \texttt{rt-ex-hpc AT encs.concordia.ca}.

\noindent \textbf{Note:} If you encounter a ``command not found'' error after logging in to Speed,
your user account may have defunct Grid Engine environment commands.
See \xa{appdx:uge-to-slurm} for instructions on how to resolve this issue.