% Array Jobs
% -------------------------------------------------------------
\subsection{Array Jobs}
\label{sect:array-jobs}

Array jobs are those that start a batch job or a parallel job multiple times.
Each iteration of the job array is called a task and receives a unique job ID.
Array jobs are particularly useful for running a large number of similar tasks with slight variations.

To submit an array job (Only supported for batch jobs),
use the \option{--array} option of the \tool{sbatch} command as follows:

\begin{verbatim}
	sbatch --array=n-m[:s] <script>
\end{verbatim}

\textbf{where}
\begin{itemize}
	\item \texttt{n}: indicates the start-id.
	\item \texttt{m}: indicates the max-id.
	\item \texttt{s}: indicates the step size.
\end{itemize}

\textbf{Examples:}
\begin{itemize}
	\item Submit a job with 1 task where the task-id is 10.
	\begin{verbatim}
		sbatch --array=10 array.sh
	\end{verbatim}

	\item Submit a job with 10 tasks numbered consecutively from 1 to 10.
	\begin{verbatim}
		sbatch --array=1-10 array.sh
	\end{verbatim}

	\item Submit a job with 5 tasks numbered consecutively with a step size of 3\\
	(task-ids 3,6,9,12,15).
	\begin{verbatim}
		sbatch --array=3-15:3 array.sh
	\end{verbatim}

	\item Submit a job with 50000 elements, where \%a maps to the task-id between 1 and 50K.
	\begin{verbatim}
		sbatch --array=1-50000 -N1 -i my_in_%a -o my_out_%a array.sh
	\end{verbatim}
\end{itemize}

\textbf{Output files for Array Jobs:} The default output and error-files are\\
\texttt{slurm-job\_id\_task\_id.out}.
This means that Speed creates an output and an error-file for each task generated by the array-job,
as well as one for the super-ordinate array-job.
To alter this behavior use the \option{-o} and \option{-e} options of \tool{sbatch}.

For more details about Array Job options, please review the manual pages for \tool{sbatch}
by executing the following at the command line on \tool{speed-submit}.
\begin{verbatim}
    man sbatch
\end{verbatim}