% Singularity Containers
% -------------------------------------------------------------
\subsection{Singularity Containers}
\label{sect:singularity-containers}

Singularity is a container platform designed to execute applications in a portable, 
reproducible, and secure manner. Unlike Docker, Singularity does not require root privileges, 
making it more suitable for HPC environments. If the \tool{/encs} software tree does not have 
the required software available, another option is to run Singularity containers. 
We run EL7 and EL9 flavors of Linux, and if some projects require Ubuntu or 
other distributions, it is possible to run that software as a container, 
including those converted from Docker. The currently recommended version of Singularity 
is \texttt{singularity/3.10.4/default}.\\

The example
\href{https://github.com/NAG-DevOps/speed-hpc/blob/master/src/lambdal-singularity.sh}{lambdal-singularity.sh}
showcases an immediate use of a container built for the Ubuntu-based LambdaLabs software stack, 
originally built as a Docker image then pulled in as a Singularity container. The source material
used for the docker image was our fork of their official repository: 
\url{https://github.com/NAG-DevOps/lambda-stack-dockerfiles}.\\

\noindent \textbf{Note}: If you make your own containers or pull from DockerHub,
use your \verb+/speed-scratch/$USER+ directory, as these images may easily 
consume gigabytes of space in your home directory, quickly exhausting your quota.\\

\noindent \textbf{Tip}: To check your quota and find big files, 
see \xs{sect:quota-exceeded} and
\href{https://www.concordia.ca/ginacody/aits/encs-data-storage.html}{ENCS Data Storage}.\\

We have also built equivalent OpenISS (\xs{sect:openiss-examples}) containers from their Docker 
counterparts for teaching and research purposes~\cite{oi-containers-poster-siggraph2023}. 
The images from \url{https://github.com/NAG-DevOps/openiss-dockerfiles}
and their DockerHub equivalents \url{https://hub.docker.com/u/openiss} can be found in 
\verb+/speed-scratch/nag-public+ with a `\texttt{.sif}' extension.
Some can be run in both batch and interactive modes, covering basics with CUDA, OpenGL rendering, 
and computer vision tasks. Examples include Jupyter notebooks with Conda support.

\begin{verbatim}
  /speed-scratch/nag-public:
  openiss-cuda-conda-jupyter.sif
  openiss-cuda-devicequery.sif
  openiss-opengl-base.sif
  openiss-opengl-cubes.sif
  openiss-opengl-triangle.sif
  openiss-reid.sif
  openiss-xeyes.sif
\end{verbatim}

This section introduces working with Singularity, its containers, and what can and cannot 
be done with Singularity on the ENCS infrastructure. For comprehensive documentation, 
refer to the authors' guide: \url{https://www.sylabs.io/docs/}.\\

Singularity containers are either built from an existing container, or from scratch. 
Building from scratch requires a recipe file (think of like a Dockerfile) and
must be done with root permissions, which are not available on the ENCS infrastructure. 
Therefore, built-from-scratch containers must be created on a user-managed/personal system. 
There are three types of Singularity containers:
% with one exception (see, Building A Container From An Existing Container).

\begin{itemize}
  \item File-system containers: built around the ext3 file system and are read-write ``file'', but cannot be resized once built.
  \item Sandbox containers: essentially a directory in an existing read-write space and are also read-write.
  \item Squashfs containers: read-only compressed ``file'' and are read-only. It is the default build type.
\end{itemize}

\noindent
``A common workflow is to use the ``sandbox'' mode for container development and then build it as a 
default (squashfs) Singularity image when done.'' says the Singularity's authors about builds.
File-system containers are considered legacy and are not commonly used.\\

For many workflows, a Docker container might already exist. In this case, you can use Singularity's 
docker pull function as part of your virtual environment setup in an interactive job allocation:

\scriptsize
\begin{verbatim}
  salloc --gpus=1 -n8 --mem=4Gb -t60
  cd /speed-scratch/$USER/
  singularity pull openiss-cuda-devicequery.sif docker://openiss/openiss-cuda-devicequery
  INFO:    Converting OCI blobs to SIF format
  INFO:    Starting build...
\end{verbatim}
\normalsize

\noindent
This method can be used for converting Docker containers directly on Speed.
On GPU nodes, make sure to pass on the \option{--nv} flag to Singularity so its containers 
could access the GPUs. See the linked example for more details.
