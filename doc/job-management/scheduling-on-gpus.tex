% Scheduling on the GPU Nodes
% -------------------------------------------------------------
\subsection{Scheduling on the GPU Nodes}
\label{sect:gpu-scheduling}

Speed has various GPU types in various subclusters of its nodes.

\begin{itemize}
	\item \texttt{speed-05} and \texttt{speed-17}:
The primary SPEED1 cluster has two GPU nodes, each with six Tesla (CUDA-compatible) P6
cards. Each card has 2048 cores and 16GB of RAM. Note that the P6
is mainly a single-precision card, so unless you need GPU double precision,
double-precision calculations will be faster on a CPU node.
	\item \texttt{speed-01}:
This \texttt{vidpro} node (see \xf{fig:speed-architecture}, contact Dr.~Maria Amer) is identical
to 05 and 17 in its GPU configuration, but managed by the priority
for the vidpro group, that is a \texttt{pg} job scheduled there
is a subject for preemption.
	\item \texttt{speed-03}, \texttt{speed-25}, \texttt{speed-25}:
These \texttt{vidpro} nodes feature NVIDIA V100 cards with 32GB of RAM.
Like \texttt{speed-01}, the priority is of the vidpro group, who
purchased the nodes, and others' jobs are a subject from preemption
within \texttt{pg}, \texttt{pt}, and \texttt{cl} partitions.
	\item \texttt{speed-37}~--~\texttt{speed-43}:
SPEED2 nodes, the main backbone of the teaching partition \texttt{pt},
have 4x A100 80GB GPUs each, partitioned into average 4x MIGs of 20GB
each, with exceptions.
	\item \texttt{nebulae}:
A member of the Nebular subcluster (contact Dr.~Jun Yan), has 2x 48GB
RTX Ada 6000 cards. This node is in the \texttt{pn} partition.
	\item \texttt{speed-19}:
Has an AMD GPU, Tonga, 16GB of GPU ram.
This node along with the majority of the NVIDIA GPU nodes are in the
\texttt{cl} partition (with restrictions) to run OpenCL, Vulkan,
and HIP jobs.
\end{itemize}

\noindent
Job scripts for the GPU queues differ in that they need these statements,
which attach either a single GPU or more GPUs to the job with the
appropriate partition:
\begin{verbatim}
  #SBATCH --gpus=[1|x]
  #SBATCH -p [pg|pt|cl|pa]
\end{verbatim}
The default max quota for $x$ is 4.

\noindent
Once your job script is ready, submit it to the GPU partition (queue) with:
\begin{verbatim}
  sbatch --mem=<MEMORY> -p pg ./<myscript>.sh
\end{verbatim}
\option{--mem} and \option{-p} can reside in the script.

\noindent
You can query \tool{nvidia-smi} on the node \textbf{running your job} with:
\begin{verbatim}
  ssh <ENCSusername>@speed-[01|03|05|17|25|27|37-43]|nebulae nvidia-smi
\end{verbatim}

\noindent The status of the GPU queues can be queried e.g. with:
\begin{verbatim}
  sinfo -p pg --long --Node
  sinfo -p pt --long --Node
  sinfo -p cl --long --Node
  sinfo -p pa --long --Node
  sinfo -p pn --long --Node
\end{verbatim}

\noindent
You can query \tool{rocm-smi} on the AMD GPU node running your job with:
\begin{verbatim}
  ssh <ENCSusername>@speed-19 rocm-smi
\end{verbatim}

\noindent
\textbf{Important note for TensorFlow and PyTorch users}:
if you are planning to run TensorFlow and/or PyTorch multi-GPU jobs, please
\textbf{do not use} the \api{tf.distribute} and/or \api{torch.nn.DataParallel} functions 
on \textbf{speed-01, speed-05, or speed-17}, as they will crash the compute node (100\% certainty). 
This appears to be a defect in the current hardware architecture.
%
% TODO: Need to link to that example
The workaround is to either manually effect GPU parallelisation (see \xs{sect:multi-node-gpu})
(TensorFlow provides an example on how to do this), or to run on a single GPU,
which is now the default for those nodes.\\

\noindent \textbf{Important}:
Users without permission to use the GPU nodes can submit jobs to the various GPU
partitions, but those jobs will hang and never run.
Their availability can be seen with:
%
\scriptsize
\begin{verbatim}
[serguei@speed-submit src] % sinfo -p pg --long --Node
Thu Oct 19 22:31:04 2023
NODELIST   NODES PARTITION       STATE CPUS    S:C:T MEMORY TMP_DISK WEIGHT AVAIL_FE REASON
speed-05       1        pg        idle 32     2:16:1 515490        0      1    gpu16 none
speed-17       1        pg     drained 32     2:16:1 515490        0      1    gpu16 UGE
speed-25       1        pg        idle 32     2:16:1 257458        0      1    gpu32 none
speed-27       1        pg        idle 32     2:16:1 257458        0      1    gpu32 none
[serguei@speed-submit src] % sinfo -p pt --long --Node
Thu Oct 19 22:32:39 2023
NODELIST   NODES PARTITION       STATE CPUS    S:C:T MEMORY TMP_DISK WEIGHT AVAIL_FE REASON
speed-37       1        pt        idle 256    2:64:2 980275        0      1 gpu20,mi none
speed-38       1        pt        idle 256    2:64:2 980275        0      1 gpu20,mi none
speed-39       1        pt        idle 256    2:64:2 980275        0      1 gpu20,mi none
speed-40       1        pt        idle 256    2:64:2 980275        0      1 gpu20,mi none
speed-41       1        pt        idle 256    2:64:2 980275        0      1 gpu20,mi none
speed-42       1        pt        idle 256    2:64:2 980275        0      1 gpu20,mi none
speed-43       1        pt        idle 256    2:64:2 980275        0      1 gpu20,mi none
\end{verbatim}
\normalsize

\noindent
To specifically request a GPU node, add, \texttt{--gpus=[\#GPUs]},
to your \tool{sbatch} statement/script or \tool{salloc} statement request.
For example:
\begin{verbatim}
  sbatch -t 10 --mem=1G --gpus=1 -p pg ./tcsh.sh
\end{verbatim}
The request can be further specified to a specific node using \option{-w}
or a GPU type or feature.

\scriptsize
\begin{verbatim}
[serguei@speed-submit src] % squeue -p pg -o "%15N %.6D %7P %.11T %.4c %.8z %.6m %.8d %.6w %.8f %20G %20E"
NODELIST         NODES PARTITI       STATE MIN_    S:C:T MIN_ME MIN_TMP_  WCKEY FEATURES GROUP DEPENDENCY
speed-05             1 pg          RUNNING    1    *:*:*     1G        0 (null)   (null) 11929     (null)
[serguei@speed-submit src] % sinfo -p pg -o "%15N %.6D %7P %.11T %.4c %.8z %.6m %.8d %.6w %.8f %20G %20E"
NODELIST         NODES PARTITI       STATE CPUS    S:C:T MEMORY TMP_DISK WEIGHT AVAIL_FE GRES      REASON
speed-17             1 pg          drained   32   2:16:1 515490        0      1    gpu16 gpu:6        UGE
speed-05             1 pg            mixed   32   2:16:1 515490        0      1    gpu16 gpu:6       none
speed-[25,27]        2 pg             idle   32   2:16:1 257458        0      1    gpu32 gpu:2       none
\end{verbatim}
\normalsize

% P6 on Multi-GPU, Multi-Node
% -------------------
\subsubsection{P6 on Multi-GPU, Multi-Node}
\label{sect:multi-node-gpu}

As described earlier, P6 cards are not compatible with \api{Distribute} and \api{DataParallel} functions
(\textbf{PyTorch}, \textbf{Tensorflow}) when running on multiple GPUs.
One workaround is to run the job in Multi-node, single GPU per node
(this applies to P6 nodes: speed-05, speed-17, speed-01):
\begin{verbatim}
  #SBATCH --nodes=2
  #SBATCH --gpus-per-node=1
\end{verbatim}

\noindent An example script for training on multiple nodes with multiple GPUs is provided in 
\href
  {https://github.com/NAG-DevOps/speed-hpc/blob/master/src/pytorch-multinode-multigpu.sh}
	{pytorch-multinode-multigpu.sh}
illustrates a job for training on Multi-Nodes, Multi-GPUs

% CUDA
% -------------------
\subsubsection{CUDA}
\label{sect:cuda}

When calling \textbf{CUDA} within job scripts, it is important to link to the desired
the desired \textbf{CUDA} libraries and set the runtime link path to the same libraries. 
For example, to use the \texttt{cuda-11.5} libraries, specify the following in your \texttt{Makefile}.
\begin{verbatim}
  -L/encs/pkg/cuda-11.5/root/lib64 -Wl,-rpath,/encs/pkg/cuda-11.5/root/lib64
\end{verbatim}

\noindent In your job script, specify the version of \texttt{GCC} to use prior to calling CUDA:
\begin{verbatim}
  module load gcc/9.3
\end{verbatim}

% Special Notes for Sending CUDA Jobs to the GPU Queue
% -------------------
\subsubsection{Special Notes for Sending CUDA Jobs to the GPU Queues}

Interactive jobs (\xs{sect:interactive-jobs}) must be submitted to the GPU partition to compile and link.
Several versions of CUDA are installed in:
\begin{verbatim}
  /encs/pkg/cuda-11.5/root/
  /encs/pkg/cuda-10.2/root/
  /encs/pkg/cuda-9.2/root
\end{verbatim}

For CUDA to compile properly for the GPU partition, edit your \texttt{Makefile} replacing\\
\texttt{\/usr\/local\/cuda} with one of the above.

% OpenISS Examples
% -------------------
\subsubsection{OpenISS Examples}
\label{sect:openiss-examples}

These examples represent more comprehensive research-like jobs
for computer vision and other tasks with longer runtime (subject to the number of epochs and other parameters).
They derive from the actual research works of students and their theses and require the use of CUDA and GPUs.
These examples are available as ``native'' jobs on Speed and as Singularity containers.

\noindent Examples include:
\paragraph{OpenISS and REID}
\label{sect:openiss-reid}

A computer-vision-based person re-identification 
(e.g., motion capture-based tracking for stage performance) part of the OpenISS
project by Haotao Lai~\cite{lai-haotao-mcthesis19} using TensorFlow and Keras.
The script is available here:
\href{https://github.com/NAG-DevOps/speed-hpc/blob/master/src/openiss-reid-speed.sh}{openiss-reid-speed.sh}.
The fork of the original repo~\cite{openiss-reid-tfk} adjusted to run on Speed is available here:
\href{https://github.com/NAG-DevOps/openiss-reid-tfk}{openiss-reid-tfk}.
Detailed instructions on how to run it on Speed are in the README:
\url{https://github.com/NAG-DevOps/speed-hpc/tree/master/src#openiss-reid-tfk}

\paragraph{OpenISS and YOLOv3}
\label{sect:openiss-yolov3}

The related code using YOLOv3 framework is in the
the fork of the original repo~\cite{openiss-yolov3} adjusted
to to run on Speed is available here: \href{https://github.com/NAG-DevOps/openiss-yolov3}{openiss-yolov3}.\\

\noindent Example job scripts can run on both CPUs and GPUs, as well as interactively using TensorFlow:

\begin{itemize}
	\item Interactive mode:
  \href{https://github.com/NAG-DevOps/speed-hpc/blob/master/src/openiss-yolo-interactive.sh}
  {openiss-yolo-interactive.sh}
	\item CPU-based job:
  \href{https://github.com/NAG-DevOps/speed-hpc/blob/master/src/openiss-yolo-cpu.sh}
  {openiss-yolo-cpu.sh}
	\item GPU-based job:
  \href{https://github.com/NAG-DevOps/speed-hpc/blob/master/src/openiss-yolo-gpu.sh}
  {openiss-yolo-gpu.sh}
\end{itemize}

\noindent Detailed instructions on how to run these on Speed are in the README: 
\url{https://github.com/NAG-DevOps/speed-hpc/tree/master/src#openiss-yolov3}