% Requesting Multiple Cores (i.e., Multithreading Jobs)
% -------------------------------------------------------------
\subsection{Requesting Multiple Cores (i.e., Multithreading Jobs)}
\label{sect:multicore-jobs}

For jobs that can take advantage of multiple machine cores, you can
request up to 32 cores (per job) in your script using the following options:

\begin{verbatim}
	#SBATCH -n      #cores for processes>
	#SBATCH -n 1
	#SBATCH -c      #cores for threads of a single process
\end{verbatim}

\noindent Both \tool{sbatch} and \tool{salloc} support \option{-n} on the command line,
and it should always be used either in the script or on the command line as the
default $n=1$.

\textbf{Important Considerations}:
\begin{itemize}
	\item Do not request more cores than you think will be useful,
	as larger-core jobs are more difficult to schedule.

	\item If you are running a program that scales out to the maximum single-machine core count available,
    please request 32 cores to avoid node oversubscription (i.e., overloading the CPUs).
\end{itemize}

\textbf{Note:}
\begin{itemize}
    \item\option{--ntasks} or \option{--ntasks-per-node}
    (\option{-n}) refers to processes (usually the ones run with \tool{srun}).
    \item \option{--cpus-per-task} (\option{-c}) corresponds to threads per process.
\end{itemize}

\noindent Some programs consider them equivalent, while others do not.
For example, Fluent uses \option{--ntasks-per-node=8} and \option{--cpus-per-task=1},
whereas others may set \option{--cpus-per-task=8} and \option{--ntasks-per-node=1}.
If one of these is not 1, some applications need to be configured to use \texttt{n * c} total cores.

\noindent Core count associated with a job appears under,``AllocCPUS'', in the, \texttt{sacct -j <job-id>}, output.

\scriptsize
\begin{verbatim}
	[serguei@speed-submit src] % squeue -l
	Thu Oct 19 20:32:32 2023
	JOBID PARTITION     NAME     USER    STATE       TIME TIME_LIMI  NODES NODELIST(REASON)
	2652        ps interact   a_user  RUNNING   9:35:18 1-00:00:00      1 speed-07
	[serguei@speed-submit src] % sacct -j 2652
	JobID           JobName  Partition    Account  AllocCPUS      State ExitCode
	------------ ---------- ---------- ---------- ---------- ---------- --------
	2652         interacti+         ps     speed1         20    RUNNING      0:0
	2652.intera+ interacti+                speed1         20    RUNNING      0:0
	2652.extern      extern                speed1         20    RUNNING      0:0
	2652.0       gydra_pmi+                speed1         20  COMPLETED      0:0
	2652.1       gydra_pmi+                speed1         20  COMPLETED      0:0
	2652.2       gydra_pmi+                speed1         20     FAILED      7:0
	2652.3       gydra_pmi+                speed1         20     FAILED      7:0
	2652.4       gydra_pmi+                speed1         20  COMPLETED      0:0
	2652.5       gydra_pmi+                speed1         20  COMPLETED      0:0
	2652.6       gydra_pmi+                speed1         20  COMPLETED      0:0
	2652.7       gydra_pmi+                speed1         20  COMPLETED      0:0
\end{verbatim}
\normalsize