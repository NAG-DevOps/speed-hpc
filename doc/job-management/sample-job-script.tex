% 2.3 Sample Job Script
% -------------------------------------------------------------
\subsection{Sample Job Script}
\label{sect:sample-job-script}

Here's a basic job script, \file{env.sh} shown in \xf{fig:env.sh}.
You can copy it from our \href{https://github.com/NAG-DevOps/speed-hpc}{GitHub repository}.

\small
\begin{figure}[htpb]
	\lstinputlisting[language=csh,frame=single,basicstyle=\ttfamily\scriptsize]{env.sh}
	\caption{Source code for \file{env.sh}}
	\label{fig:env.sh}
\end{figure}
\normalsize

\noindent The first line is the shell declaration (also know as a shebang) and sets the shell to \emph{tcsh}.
The lines that begin with \texttt{\#SBATCH} are directives for the scheduler.

\begin{itemize}
	\item \option{-J} (or \option{--job-name}) sets \emph{envs} as the job name.
	\item \option{--mail-type} sets the type of notifications.
	\item \option{--chdir} sets the working directory.
	\item \option{--nodes} specifies the number of required nodes.
	\item \option{--ntasks} specifies the number of tasks.
	\item \option{--cpus-per-task} requests 1 cpus.
	\item \option{--mem=} requests memory.

    \textbf{Note:} Jobs require the \option{--mem} option to be set either in the script
	or on the command line; \textbf{job submission will be rejected if it's missing.}
\end{itemize}

\noindent The script then:
\begin{enumerate}
    \item Creates a directory.
	\item Sets TMPDIR to a larger storage.
	\item Prints current date.
	\item Prints env variables.
	\item Prints current date again.
\end{enumerate}

\noindent The scheduler command, \tool{sbatch}, is used to submit (non-interactive) jobs.
From an ssh session on ``speed-submit'', submit this job with
\begin{verbatim}
    sbatch ./env.sh
\end{verbatim}

\noindent You will see, \texttt{Submitted batch job <JOB ID>}.
The commands \tool{squeue} and \tool{sinfo} can be used to look at the queue and the status of the cluster
%\texttt{squeue -l} and \texttt{sinfo -la}.

\small
\begin{verbatim}
[serguei@speed-submit src] % squeue -l
Thu Oct 19 11:38:54 2023
JOBID PARTITION     NAME     USER    STATE       TIME TIME_LIMI  NODES NODELIST(REASON)
 2641        ps interact   b_user   RUNNING   19:16:09 1-00:00:00      1 speed-07
 2652        ps interact   a_user   RUNNING      41:40 1-00:00:00      1 speed-07
 2654        ps envs       serguei  RUNNING       0:01 7-00:00:00      1 speed-07

[serguei@speed-submit src] % sinfo
PARTITION AVAIL  TIMELIMIT  NODES  STATE NODELIST
ps*          up 7-00:00:00      8  drng@ speed-[09-11,15-16,20-21,36]
ps*          up 7-00:00:00      3   drng speed-[38,42-43]
ps*          up 7-00:00:00      2  drain magic-node-[04,08]
ps*          up 7-00:00:00      4    mix magic-node-07,salus,speed-[07,37]
ps*          up 7-00:00:00      7  alloc magic-node-06,speed-[08,12,22-24,29]
ps*          up 7-00:00:00     13   idle magic-node-[05,09-10],speed-[19,30-35,39-41]
pg           up 7-00:00:00      1 drain* speed-05
pg           up 7-00:00:00      2    mix speed-[01,17]
pt           up 7-00:00:00      4   drng speed-[27,38,42-43]
pt           up 7-00:00:00      2    mix speed-[17,37]
pt           up 7-00:00:00      3   idle speed-[39-41]
pa           up 7-00:00:00      1   drng speed-27
pa           up 7-00:00:00      1    mix speed-01
pa           up 7-00:00:00      2   idle speed-[03,25]
cl           up 7-00:00:00      1 drain* speed-05
cl           up 7-00:00:00      4   drng speed-[27,38,42-43]
cl           up 7-00:00:00      3    mix speed-[01,17,37]
cl           up 7-00:00:00      6   idle speed-[03,19,25,39-41]
pc           up 7-00:00:00      1    mix salus
pm           up 7-00:00:00      2  drain magic-node-[04,08]
pm           up 7-00:00:00      1    mix magic-node-07
pm           up 7-00:00:00      1  alloc magic-node-06
pm           up 7-00:00:00      3   idle magic-node-[05,09-10]
pn           up 7-00:00:00      1  down* stellar
pn           up 7-00:00:00      2   idle matrix,nebulae
\end{verbatim}
\normalsize

Once the job finishes, there will be a new file in the directory that the job was started from,
with the syntax of, \texttt{slurm-<job id>.out}. This file represents the standard output
(and error, if there is any) of the job in question.
Important information is often written to this file.
%
%Congratulations on your first job!