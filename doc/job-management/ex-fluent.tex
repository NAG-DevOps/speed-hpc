% Example Job Script: Fluent
% -------------------------------------------------------------
\subsection{Example Job Script: Fluent}
\label{sect:example-fluent}

\begin{figure}[htpb]
  \lstinputlisting[language=csh,frame=single,basicstyle=\footnotesize\ttfamily]{fluent.sh}
  \caption{Source code for \texttt{fluent.sh}}
  \label{fig:fluent.sh}
\end{figure}

The job script in \xf{fig:fluent.sh} runs Fluent in parallel over 32 cores.
Notable aspects of this script include requesting e-mail notifications (\option{--mail-type}),
defining the parallel environment for Fluent with \option{-t\$SLURM\_NTASKS} and \option{-g-cnf=\$FLUENTNODES},
and setting \api{\$TMPDIR} as the in-job location for the ``moment'' \file{rfile.out} file.
The script also copies everything from \api{\$TMPDIR} to a directory in the user's NFS-mounted home after the job completes.
Job progress can be monitored by examining the standard-out file (e.g.,
\texttt{slurm-249.out}), and/or by examining the ``moment'' file in TMPDIR (usually
\texttt{/disk/nobackup/<yourjob>} (it starts with your job-ID)) on the node running
the job. Be cautious with journal-file paths.